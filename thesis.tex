\documentclass[12 pt,oneside]{book}
\usepackage[top=1 in,left=1 in,right=1 in,bottom=1 in]{geometry}
\usepackage[subpreambles=true]{standalone}
\usepackage{import}
\usepackage[final]{showlabels}
\usepackage{listings}
\usepackage{xcolor}
\usepackage{courier}
\usepackage{fancyhdr}
\usepackage{pbox}
\renewcommand{\showlabelfont}{\tiny\slshape\color{red}}

\pagestyle{fancy}
\fancyhf{}
\lhead{Schrecengost}
\rhead{Doctoral Thesis}
\cfoot{\thepage}


\begin{document}
\frontmatter
\begin{center}
	\thispagestyle{empty}
	\vspace*{1 in}
	Finite Generation in RGD Systems with Exceptional Rank-2 Residues\\[1 in]
	Mark Allen Schrecengost Jr.\\
	Apollo, Pennsylvania\\[0.5 in]
	Bachelor of Science, Grove City College, 2014\\[1.5 in]
	A Thesis presented to the Graduate Faculty\\
	of the University of Virginia in Candidacy for the Degree of\\
	Doctor of Philosophy\\[0.5 in]
	Department of Mathematics\\[0.5 in]
	University of Virginia\\
	May, 2020
\end{center}
\vfill
\hfill\pbox{\textwidth}{
Committee Members:\\
Peter Abramenko\\
Mikhail Ershov\\
Thomas Koberda\\
}
\clearpage
\thispagestyle{empty}
\begin{center}
	\textbf{Abstract}
\end{center}
Let $(G,(U_\alpha),T)$ be an RGD system. The most prominent examples are Kac-Moody groups, which are infinite dimensional analogs of semisimple Lie groups. These groups have an associated twin building $\Delta$ on which the group $G$ will act strongly transitively. We say that the building $\Delta$ satisfies condition (co) of the collection of chambers opposite any chamber is gallery connected. It is known that if $\Delta$ satisfies (co), the the subgroup $U_+$ of $G$ is generated by some finite set of fundamental root groups, and thus is finitely generated if these root groups are finitely generated.

We will help close the gap in the literature relying on condition (co) by proving when RGD systems associated to rank-3 buildings, are and are not finitely generated when the buildings do not satisfy condition (co). Most of the time, the group $U_+$ will not be finitely generated, and we will give sufficient conditions to guarantee the infinite generation of $U_+.$ We will then modify this aproach to see that another group not covered by the conditions is also not finitely generated. Our main strategy will be to produce a large family of surjective homomorphisms from $U_+$ which send relatively few $U_\alpha$ to non-identity elements, implying that some of these $U_\alpha$ must be in any generating set.

Finally, we will show that there are two cases where $U_+$ remains finitely despite $\Delta$ not satisfying (co). We will use an aproach which relies on defining a distance between root groups, and showing that most root groups can be expressed in terms of those closer to the fundamental chamber. This approach can also give another proof of the finite generation of $U_+$ with condition (co).
\clearpage
\chapter*{Acknowledgements}
First of all I would like to thank my adviser, Peter Abramenko, for his guidance through the Ph.D. process. Without him this thesis would not exist. It has been a pleasure working in the world of combinatorial group theory, and his attention to detail have made me a better mathematician.

I would also like to thank my fellow advisees, Zach Gates and Ted Williams. Many a discussion about Buildings, amongst other topics, have pushed me to think, do, and learn more as a mathematician. Having someone there to understand what it is like to draw so many triangles is something that cannot be overlooked.

My fellow graduate students, espcially those from my cohort, have made the long and difficult process of graduate school one of the best experiences of my life. Without them to share tennis, boardgames, or spring barbeques, I would have long ago lost my mind and probably quit.

I want to thank my parents, Mark and Regina, for supporting me on this journey since the only math I could do was multiply numbers with the help of some stacks of pennies. Their love and guidance through every stage of my life has shaped me to where I am today.

Finally, I most of all want to thank my wife Robin and our pet bunny Kaladin. Without her love, support, encouragement, and occasional kicks in the butt, I almost certainly would not have finished this thesis, and it surely would have been a much more painful process. Her unwavering confidence has been the most wonderful gift, especially when I doubted myself the most. Kaladin also provided some much needed help when I needed a break from writing to give me some bunny pets, as well as all around cuteness. I tried to employ him as a proofreader, but he prefered to simply eat my thesis instead of reading it.




\tableofcontents
\mainmatter

\setcounter{chapter}{-1}
\import{chapters/intro/}{intro}

\import{chapters/coxeter/}{coxeter}

\import{chapters/buildings/}{buildings}

\import{chapters/rgd/}{rgd}

\import{chapters/knownresults/}{knownresults}

\import{chapters/generalcase/}{generalcase}

\import{chapters/exceptionalcase/}{exceptionalcase}


\lstset{ 
  backgroundcolor=\color{lightgray},   % choose the background color; you must add \usepackage{color} or \usepackage{xcolor}; should come as last argument
  basicstyle=\scriptsize\ttfamily,        % the size of the fonts that are used for the code
  breakatwhitespace=false,         % sets if automatic breaks should only happen at whitespace
  breaklines=true,                 % sets automatic line breaking
  captionpos=b,                    % sets the caption-position to bottom
  commentstyle=\color{olive},    % comment style
  deletekeywords={...},            % if you want to delete keywords from the given language
%  escapeinside={\%*}{*)},          % if you want to add LaTeX within your code
  extendedchars=true,              % lets you use non-ASCII characters; for 8-bits encodings only, does not work with UTF-8
  firstnumber=1,                % start line enumeration with line 1000
  frame=single,	                   % adds a frame around the code
  keepspaces=true,                 % keeps spaces in text, useful for keeping indentation of code (possibly needs columns=flexible)
  keywordstyle=\color{blue},       % keyword style
  language=Python,                 % the language of the code
  morekeywords={*,...},            % if you want to add more keywords to the set
  numbers=left,                    % where to put the line-numbers; possible values are (none, left, right)
  numbersep=5pt,                   % how far the line-numbers are from the code
  numberstyle=\tiny\color{gray}, % the style that is used for the line-numbers
  rulecolor=\color{black},         % if not set, the frame-color may be changed on line-breaks within not-black text (e.g. comments (green here))
  showspaces=false,                % show spaces everywhere adding particular underscores; it overrides 'showstringspaces'
  showstringspaces=false,          % underline spaces within strings only
  showtabs=false,                  % show tabs within strings adding particular underscores
  stepnumber=1,                    % the step between two line-numbers. If it's 1, each line will be numbered
  stringstyle=\color{red},     % string literal style
  tabsize=2	                   % sets default tabsize to 2 spacesF.
}
\appendix
\chapter{Code for Diagram Generation}
\label{ch:app}
Included in Appendix \ref{ch:app} is the Python code used to generate the various figures. All files should be saved in the same folder, and only hyperbolictilinggenerator.py needs to be run. The code will create (or modify) a document named texcode.tex. This document uses the standalone package, and should be able to be compiled as is, imported into another document, or simply gutted for the tikz code. A word of warning, the code can generate exceptionally large tikz files. There is no check to determine how long the program will run with given inputs, and there is certainly no way of telling how long your \LaTeX compiler will take to actually compile the document.\\

\lstinputlisting[caption=simplex.py]{/home/mark/Documents/gradschool/research/hyperbolictilinggenerator/simplex.py}
\lstinputlisting[caption=hyptransformations.py]{/home/mark/Documents/gradschool/research/hyperbolictilinggenerator/hyptransformations.py}
\lstinputlisting[caption=coxetergroup.py]{/home/mark/Documents/gradschool/research/hyperbolictilinggenerator/coxetergroup.py}
\lstinputlisting[caption=hyperbolictilinggenerator.py]{/home/mark/Documents/gradschool/research/hyperbolictilinggenerator/hyperbolictilinggenerator.py}


\bibliographystyle{unsrt}
\bibliography{bibliography.bib}

\end{document}
