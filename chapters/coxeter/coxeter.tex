\documentclass[class=book, crop=false,12 pt]{standalone}
\usepackage[subpreambles=true]{standalone}
\usepackage{import}
\usepackage{/home/mark/Documents/gradschool/research/thesis/preamble}

%\newcommand{\leqnomode}{\tagsleft@true}
\newcommand{\cop}{\ensuremath{C^{\text{op}}}}

\begin{document}
\chapter{Coxeter Groups and Coxeter Complexes}
\label{ch:coxeter}
The theory of RGD systems was developed \textcolor{red}{for some reason.} These systems describe groups with an incredible ammound of geometric structure, which allows us to say a great deal about the group theory. While we will ultimately use some geometric properties to prove results about finite generation in RGD systems, before we can do that we will need to understand the underlying geometry. This geometry starts with Coxeter groups.

\section{Coxeter Groups}
\label{sec:coxgrp}
\begin{defn}
	\label{def:coxgrp}
	A Coxeter system is a pair $(W,S)$ such that $S$ is a finite set, and $W$ is a group with a presentation
	\[
		W=\langle s\in S|(st)^{m(s,t)}=1\text{ for all }s,t\in S\rangle
	\]
	subject to the conditions that $m(s,t)\in \mathbb{N}\cup \{\infty\},$ $m(s,s)=1,$ and $m(s,t)=m(t,s)\ge 2$ if $s\neq t.$ If $m(s,t)=\infty$ then we simply discard the relation $(st)^{m(s,t)}=1.$
\end{defn}

Through slight abuse of terminology we will refer to $W$ as a Coxeter group, but we will always have a specifice generating set $S$ for the Coxeter system in mind. Coxeter groups have many nice properties, and far too many to discuss here, but we will mention a few which will be of use later. The first of which is the length function. If $(W,S)$ is a Coxeter system then we can define a function $\ell:W\to \mathbb{N}$ by $\ell(w)$ is the minimum number $n$ such that $w$ can be written as $w=s_1s_2\cdots s_n$ with $s_i\in S$ for all $i.$ This length function is standard in group theory, and can be defined on any group with any generating set. However, in Coxeter groups this length function takes on a much richer structure which we will describe in more detail.


\subsection{M-Operations}
We say that $(s_1,s_2,\dots,s_n)$ is a decomposition of $w$ if $w=s_1s_2\cdots s_n,$ and that it is a reduced decomposition if $n=\ell(w).$ Certainly decompositions, and even reduced decompositions need not be unique, and we will see some ways that we can generate new decompositions. By definition, $m(s,s)=1$ so $s^2=1$ for all $s\in S.$ Thus if we ever have an element of $s$ repeated twice in a row in a decompostion, we can simply delete the copies to get another decomposition with smaller length. If $s\neq t\in S$ then $(st)^{m(s,t)}=1$ and thus we can say
\[
	\underbrace{sts\cdots t(s)}_{m(s,t)}=\underbrace{tst\cdots s(t)}_{m(s,t)}
\]
This again means if we have any alternating string of $s$ and $t$ of the right length, then we can replace it with the swapped alternating string, and get another decompostion of the same length. These two decompostion operations are immediate consequences of the definition of a Coxeter system, but as we will see in the following theorem from \cite{buildings}, these give us everything we need.

\begin{theorem}
	\label{thm:Mop}
	If $(W,S)$ is a coxeter system and $(s_1,s_2,\dots,s_n)$ is a decompostion of $w,$ then we can obtain a new decomposition of $W$ by deleting a substring of the form $(s,s),$ or relpacing a substring of length $m(s,t)$ of the form $(s,t,\dots,s(,t))$ with a substring of the form $(t,s,\dots,t(,s)).$ We will call these two operations \emph{M-Operations} of type 1 and 2 respectively. Furthermore, any decomposition of $w$ can be transformed into a reduced decompostion by repeated application of M-Operations of type 1 and 2, and any two reduced decompostions of $w$ can be transformed into one another by applications of M-Operations of type 2.
\end{theorem}

There are many consequences of Theorem \ref{thm:Mop} but one of the most notable is this, we have a simple algorithm to obtain a reduced decomposition of any $w,$ and we can always check if a decomposition is reduced. In either case we repeatedly apply any possible M-Operations, and applying those of type 1 if possible or noting if none are possible in the case of a decomposition which is already reduced. It also gives us some facts about the length function. For example, if we can write $w=s_{i_1}\cdots s_{i_k}$ then $\ell(w)$ and $k$ are either both even, or both odd, as application of type 1 operations will always reduce the length of a decomposition by 2.

\subsection{Standard Subgroups and Standard Cosets}
Coxeter groups also have a nice subgroup structure will give rise to the rich geometry we will use later. If $(W,S)$ is a Coxeter system then by definition $W$ is generated by $S.$ For any $J\subset S$ we can form a subgroup $W_J=\langle s|s\in J\rangle\le W_S.$ For example, $W_S=W$ and $W_{\emptyset}=\{1\}.$ We will also define a standard coset to be any coset of the form $wW_J$ for any $w\in W$ and $J\subset S.$ Standard cosets also have a type function and it is the type of the associated standard subgroup.

As before, there is nothing special about these definitions, as similar definitions hold for any group, but what is special is the structure on standard subgroups. The map which sends $J\to W_J$ is a bijection from subsets of $S$ to standard subgroups. If $H$ is a standard subgroup, then its $J$ can be recovered as $H\cap S.$ We can also check that $(W_J,J)$ is also a Coxeter system.

We can use Theorem \ref{thm:Mop} to derive some basic consequences about standard subgroups. For example, we can show that $W_J\cap W_J'=W_{J\cap J'}.$ One inclusion is clear, and if we take $w\in W_J\cap W_J'$ we can write two reduced decompositions of $w,$ one of which only uses letters from $J$ and the other only uses letters from $J'.$ These reduced decompositions can be transformed into one another by M-Operations of type 2, but M-Operations cannot introduce new letters into a reduced decomposition, only change the order. Thus every letter in the intial decompositions must be in $J$ and $J'.$

One situation which will be very useful later is when the group $W$ is finite. We say that a Coxeter Group or Coxeter System is \emph{sperical} if $W$ is finite. If $(W,S)$ is spherical then we can prove several facts. First of all, $W$ has a unique element of maximal length, which is usually denoted $w_0.$ It has the property that $\ell(ww_0)=\ell(w_0)-\ell(w)$ for every $w\in W.$ One consequence of this fact is that for any $w\in W$ a reduced decomposition of $w$ can be extended to a reduced decomposition of $w_0.$ This element of maximal length will be of some interest in the geometry of $W$ as well. In a similar fashion, we say that $J$ is a spherical subset of $S$ if $W_J$ is spherical. We also say that $W$ is 2-spherical if every subset of $S$ of size 2 is a spherical subset. This is equivalent to saying that $m(s,t)<\infty$ for every $s,t\in S.$

Let $\Delta$ be the set of all standard subgroups of $W,$ with a partial order given by reverse inclusion, so that $W_J\le W_{J'}$ if and only if $J'\subset J.$ Using the fact from the previous paragraph, one can check that $\Delta$ is isomorphic as a poset to the subsets of $S$ under reverse inclusion. This fact is the basis for our definition of the Coxeter Complex.

\section{Coxeter Complex}
\begin{defn}
	If $(W,S)$ is a Coxeter system, let $\Sigma$ be the collection of standard cosets of $W,$ ordered by reverse inclusion. Then $\Sigma$ is a simplical complex called the Coxeter Complex of $W.$
\end{defn}

In the standard terminology of simplicial complexs, we will refer to each standard coset as a simplex, and $A$ and $B$ are simplices with $A\le B$ then we say $A$ is a face of $B.$ One can check that the dimension of any simplex $wW_J$ will be $|S|-|J|-1$ because the ordering is by reverse inclusion. For this reason, it is sometimes more useful to refer to the rank of a simplex which is one more than the dimension, so that the rank of $wW_J=|S|-|J|.$ We can also draw several conclusions from this fact. First of all, every maximal simplex of $\Sigma$ has the same dimension, $|S|-1,$ and they will correspond exactly to the elements of $W$ by $w\mapsto wW_\emptyset.$ We can also see that the standard subgroup $W=W_S$ is a simplex of dimension $-1$ and of rank 0 which is a face of every other simplex.

The Coxeter complex is also equiped with a type function $\tau:\Sigma \to \mathcal{P}(S)$ by $\tau(wW_J)=S\setminus J.$ However, for convinience, we will more often refer to the \emph{cotype} of a simplex which is $S\setminus \tau(wW_J)=J.$ For example, maximial dimensional simplices will have cotype $\emptyset,$ and co-dimension 1 simplices will have type $\{s\}$ for some $s\in S.$ This convention is also convinient as simplices of cotype $J$ will have rank $|J|$ and dimension $|J|-1.$

We will call the maximal simplices of $\Sigma$ \emph{chambers} and the co-dimension 1 simplices of $\Sigma$ will be called \emph{panels}. A panel will have cotype $\{s\}$ for some $s\in S,$ or just cotype $s$ for short. If we take a look at a panel of cotype $s,$ we see that it is a standard subgroup of the form $wW_{\{s\}}=w\{1,s\}=\{w,ws\}.$ Thus each panel will contain exactly two chambers, corresponding to $w$ and $ws,$ and we will say that the chambers $w$ and $ws$ are $s$-adjacent. We say that two chambers are ajacent if they are $s$-adjacent for some $s\in S.$ We will also note that there is an obvious chamber which can be distinguished, namely the chamber $W_\emptyset=\{1\}.$ We will call this the \emph{fundamental chamber} of $\Sigma$ and denote it as $C.$

A \emph{gallery} in $\Sigma$ is a sequence of chambers $D_0,D_2,\dots,D_n$ such that $D_i$ and $D_{i+1}$ are ajacent for every $i.$ We will say that a subset $\D$ of chambers of $\Sigma$ is gallery connected if for all chambers $D,E\in \D,$ there is a gallery $D_0,\dots,D_n$ in $\D$ such that $D_0=D$ and $D_n=E.$ Using this definition we have the following result about $\Sigma.$
\begin{prop}
	\label{prop:gallerycon}
	If $\Sigma$ is the Coxeter complex of a Coxeter system $(W,S)$ then $\Sigma$ is gallery connected.
\end{prop}
\begin{proof}
	If $D_1,\dots,D_n$ is a gallery then $D_n,\dots,D_1$ will also be a gallery. We can also see that if $D_1,\dots,D_n$ and $E_1,\dots,E_m$ are galleries with $D_n=E_1$ then $D_1,\dots,D_n=E_1,\dots,E_m$ will also be a gallery. Thus it will suffice to show that for any chamber $D,$ there is a gallery between $D$ and the fundamental chamber $C.$

	If $D$ is a chamber of $\Sigma$ then $D$ is a standard coset $wW_\emptyset=\{w\}$ for some $w\in W.$ We will induct on $\ell(w).$ If $\ell(w)=0$ then $w=1$ and $D=C$ so the result is immediate. Otherwise, let $(s_{i_1},s_{i_2},\dots,s_{i_n})$ be a minimal decomposition of $w$ with $n\ge 1$ so that $w=s_{i_1}s_{i_2}\cdots s_{i_n}$ and $n=\ell(w).$ Then $w'=s_{i_1}s_{i_2}\cdots s_{i_{n-1}}\in W$ with $\ell(w')=n-1<\ell(w).$ If $D'$ is the chamber corresponding to $w'$ then inductively there is a gallery $C,\dots,D'.$ If we examine the panel of $D'$ of cotype $s_{i_n}$ then we will see it must be $w'\{1,s_{i_n}\}=\{w',w\}$ and thus $D'$ and $D$ are adjacent. Thus we can extend our gallery $C,\dots,D',D$ to get a gallery from $C$ to $D$ as desired.
\end{proof}

It turns out that $\Sigma$ is sufficiently nice that the geometry of the lower dimension simplices can be recovered from the chambers of $\Sigma$ and from the $s$-ajacency relations. Thus we will rarely need to make arguments using simplices other than chambers or panels. This also means when considering subset of $\Sigma,$ we will instead use the chambers of $\Sigma,$ $\mathcal{C}(S)$ almost exclusively.

If $D$ and $E$ are chambers then a minimal gallery between $D$ and $E$ is a gallery of minimal length, that is, any other gallery between $D$ and $E$ is at least as long. Then we can turn $\mathcal{C}(\Sigma)$ into a metric space where $d(D,E)$ is the length of a minimal gallery between $D$ and $E.$ It is not so surprising that there is a direct link between galleries in $\Sigma$ and decompositions in $W.$ In fact, we have the following facts which can be found in \cite{buildings}. If $D=w$ and $E=w'$ are chambers of $\Sigma,$ then $d(D,E)=\ell(w^{-1}w').$ Furthermore, if $(s_{i_1},\dots,s_{i_n})$ is any decomposition of $w^{-1}w'$ then there is a gallery $D_0,\dots,D_n$ from $D$ to $E$ where $D_j$ is $s_{i_j}$ adjacent to $D_{j+1}$ for all $j.$ In this case the minimal galleries will correspond to reduced decompositions.

\subsection{Links and Stars}
We saw before that if $J\subset S$ then $(W_J,J)$ is also a Coxeter system. This structure will also carry over into the coxeter complexes. Before giving the details, we need to define a few more terms. In any simplical complex, we say that two simplices $A$ and $B$ are joinable if they are contained in a common maximal simplex. In term of the coxeter complex $\Sigma,$ two simplices $A=wW_J$ and $B=w'W_{J'}$ are joinable if they share a common element $w.$ We can now make two more definitions which we will use extensively through the rest of the paper.

\begin{defn}
	If $A$ is a simplex of $\Sigma,$ then the star of $A$, $\st(A),$ is all of the simplices of $\Sigma$ which are joinable to $A.$ In terms of chambers $\mathcal{C}(\st(A))=\{w\in W|w\in A=w'W_J\}.$ We can also define the link of $A$, $\lk(A),$ as the set of all simplices of $\Sigma$ which are joinable to $A,$ but do not contain $A.$
\end{defn}

Now we can see how the subgroup structure of $W$ translates to the geometry of $\Sigma.$
\begin{prop}
	\label{prop:link}
	If $A$ is a simplex of $\Sigma$ of cotype $J,$ then $\lk(A)$ is isomorphic as simplical complexes to the Coxeter complex $\Sigma_J$ of $(W_J,J).$
\end{prop}

We can define $\Sigma_{\ge A}$ to be the set of simplices in $\Sigma$ which contain $A.$ There is a bijection from $\lk(A)$ to $\Sigma_{\ge A}$ given by $B\mapsto B\cup A$ which is also an isomorphism as posets. Using this fact we can check that the chambers of $\st(A)$ will be in 1-1 correspondence with the maximal simplices of $\lk(A)$ which are also the chambers of $\Sigma_J.$ For a simplex $A,$ the star and link of $A$ will give more or less the same combinatorial information, and thus which one we use will be somewhat a matter of convinience.

Stars and links have other nice properties which we will take advantage of later. First of all $\mathcal{C}(\st(A))$ is gallery connected, and the galleries in $\st(A)$ correspond exactly to galleries in $\Sigma_J.$ Furthermore, suppose that $D_0,\dots,D_n$ is a minimal gallery between two chambers in $\st(A)$ where $A$ has cotype $J.$ Then we know that $D_i$ and $D_{i+1}$ are $s_i$ adjacent for some $s_i\in S.$ But in fact, $s_i\subset J$ for every $i.$ In fact, the types of these adjacencies is exactly the same as those in the minimal gallery of $\Sigma_J.$ 

We say that a Coxeter complex $\Sigma$ is spherical or 2-spherical if $W$ is spherical or 2-spherical. If $\Sigma$ is spherical then we will define $C^{\text{op}}$ to be the chamber of $\Sigma$ corresponding to $w_0.$ Then $\cop$ is the unique chamber of $\Sigma$ at maximal distance from $C,$ and it has the property that every chamber of $\Sigma$ is part of a minimal gallery from $C$ to $\cop.$ 

Now suppose that $\Sigma$ is a 2-spherical coxeter complex, and let $A$ be a simplex of $\Sigma$ of co-dimension 2. Then $A$ is a simplex of cotype $J=\{s,t\}$ for some $s,t\in S.$ By definition of 2-spherical, this means $W_J$ is spherical and thus there are finitely many chambers in $\st(A).$ Every chamber in $\st(A)$ also has a unique chamber at maximal distance away in $\st(A)$ which we will call opposite in $\st(A).$ If we examine the structure of $W_J$ we can even see that it is the dihedral group of order $2m(s,t),$ and the simplical compelex $\Sigma_J$ will be a $2m(s,t)$-gon with edges as chambers and vertices as panels. Translating to $\Sigma$ this means that $\st(A)$ consists of $2m(s,t)$ chambers arranged in a circular patern around $A,$ and opposite chambers in $\st(A)$ will be at distance $m(s,t)$ away from each other. 

\subsection{Projections}
Another useful tool for studying the geometry of $\Sigma$ is the concept of projections. 

\begin{theorem}
	If $A$ is a simplex of $\Sigma,$ and $D$ is a chamber of $\Sigma,$ then there is a chamber $E\in \st(A)$ such that $d(D,E)\le d(D,E')$ for all $E'\in \st(A).$ Additionally, the chamber $E$ is unique and we define the projection of $D$ on to $A,$ or $\proj_A(D)$ to be the chamber $E.$ The projection $E$ is also characterized by the property that $d(D,E')=d(D,E)+d(E,E')$ for all $E'\in \st(A).$
\end{theorem}

The property $d(D,E')=d(D,E)+d(E,E')$ is known as the gate property because it means for any $E'\in \st(A),$ there is a minimal gallery from $D$ to $E'$ which passes through $E.$ Projections also allow us to define a notion of convexity in a Coxeter complex.

\begin{defn}
	\label{defn:convex}
	We say that a subcomplex $\Delta$ of $\Sigma$ is convex, if $\proj_{A}(D)\in \Delta$ whenever $A$ is a simplex of $\Delta$ and $D$ is a chamber of $\Delta.$
\end{defn}

Convexity also has another interpretation, which can be taken as the definition if desired. A chamber subcomplex $\Delta$ of $\Sigma$ is convex if for any chambers $D,E$ of $\Delta,$ any minimial gallery from $D$ to $E$ in $\Sigma$ is contained in $\Delta.$ This means that we can look for minimal gallerys in a convex chamber subcomplex of $\Sigma,$ and still be sure that it will be minimal in all of $\Sigma.$ One of the most common uses for this is to apply the result to the convex chamber subcomplex $\st(A)$ for some simplex $A.$ If $D,E\in \st(A)$ for some simplex $A,$ then any minimal gallery from $D$ to $E$ will be contained in $\st(A),$ which is very easy to understand based on our earlier remarks.

\subsection{Roots}
Intuitively we should think of Coxeter groups as reflection groups in some space. A reflection should divide a space into two halves, which are sswitched by a reflection. We will formalize this notion with the concept of roots.

\begin{defn}
	\label{defn:root}
	For any adjacent chambers $D,D',$ let $\alpha_{D,D'}$ be the subcomplex of $\Sigma$ defined by $\mathcal{C}(\alpha_{D,D'})=\{E\in \Sigma|d(E,D)<d(E,D')\}.$ Then $\alpha_{D,D'}$ is called a root, and the collection of all $\alpha_{D,D'}$ for adjacent chambers $D$ and $D'$ are called the roots of $\Sigma.$
\end{defn}

We will denote the set of all roots of $\Sigma$ by $\Phi.$ A consequence of Theorem \ref{Mop} is that $d(E,D)\neq d(E,D')$ for every chamber $E$ of $\Sigma,$ so we cna think about $\alpha_{D,D'}$ as the chambers which are closer to $D$ than to $D'.$ This also means that for any chamber $E,$ we have either $d(E,D)>d(E,D')$ or $d(E,D)<d(E,D').$ If $D$ and $D'$ are adjacent chambers then both $\alpha_{D,D'}$ and $\alpha_{D',D}$ will be roots, and we will have $\mathcal{C}(\alpha_{D,D'})\cap \mathcal{C}(\alpha_{D',D})=\emptyset$ and $\mathcal{C}(\alpha_{D,D'})\cup \mathcal{C}(\alpha_{D',D})=\mathcal{C}(\Sigma).$

The roots $\alpha_{D,D'}$ and $\alpha_{D',D}$ are very closely related, and roughly correspond to the two half spaces defined by a reflection. To differentiate between these roots, we say a root is \emph{positive} if it contains the fundamental chamber $C.$ This choice is of course arbitrary, but the chamber $C$ is a convinient choice. Similarly, we say a root is negative if it does not contain $C,$ and we say that $\alpha_{D,D'}$ and $\alpha_{D',D}$ are opposite roots. We will also denote this with the notation $\alpha_{D',D}=-\alpha_{D,D'}.$

If roots are roughly analagous to the half spaces defined by a reflection, then we should also have some notion of the reflection line. If $\alpha$ is a root of $\Sigma$ then we define the \emph{wall} of $\alpha,$ denoted by $\partial\alpha$ or $\mathcal{H}_\alpha,$ to be $\alpha\cap (-\alpha).$ Then certainly $\partial \alpha$ will contain no chambers, but will not be non-empty, as the panel contained in $D$ and $D'$ will be in $\partial\alpha$ if $\alpha=\alpha_{D,D'}.$

There are several facts about roots and walls which we will use later. Every root is gallery connected, and is also a convex chamber subcomplex of $\Sigma.$ What is even more interesting is the interaction between roots and links. Suppose $A$ is a simplex of $\Sigma$ of cotype $J.$ Then we can recall that $\lk(A)\cong \Sigma_J$ where $\Sigma_J$ is is a Coxeter complex for $(W_J,J).$ Then there is a natual corespondence between roots in $\Sigma$ to roots in $\lk(A).$ The map $\alpha \to \alpha \cap \lk(A)$ is a bijection between the roots of $\Sigma$ such that $A\in \alpha,$ and the roods of $\lk(A)$ viewed as a Coxeter complex in its own right. Furthermore, this map is also a bijection between walls as well. These results further reiterate the fact that when working in $\lk(A),$ we can essentially forget about the rest of $\Sigma$ and consider only the Coxeter complex for $(W_J,J).$ This will be especially useful when discussing links of co-dimension 2 simplices.


Thus far we have discussed many properties and attributes of $\Sigma,$ but we have not really described how the group theory of $W$ interacts with $\Sigma$ besides in the notion of galleries. In the next section we will see that we can say much more about the interaction of the group $W$ and the Coxeter complex $\Sigma.$

\section{W-Action}
\begin{prop}
	\label{prop:wact}
	There is a well defined action of $W$ on $\Sigma$ by $w'(wW_J)=w'wW_J.$ Then each $w\in W$ induces an isomorphism of $\Sigma$ which also preserves (co)type of each simplex.
\end{prop}

As $\Sigma$ is built directly from $W,$ it is unsurprising that this $W$ action plays very nicely with the geometry of $\Sigma,$ and we will briefly collect the more relevant facts. The $W$-action sents galleries to galleries, minimal galleries to minimal galleries, and thus $d(D,E)=d(wD,wE)$ for all $D,E\in \mathcal{C}(\Sigma)$ and $w\in W.$ 

Because of how natural our definition of the $W$ action is, we can also check relatively easily that $W$ interacts nicely with all of the concepts we have defined so far. If $A$ is a simplex and $D$ is a chamber then we have $\proj_{wA}(wD)=w\proj_{A}(D)$ for all $w\in W.$ If $\alpha$ is a root of $\Sigma$ then $w\alpha$ is also a root with wall $w\partial\alpha.$ Furthermore, if $\partial\alpha$ is a wall which separates $D$ and $D'$ then $w\partial\alpha$ will separate $wD$ and $wD'.$ This also means $w\alpha_{D,D'}=\alpha_{wD,wD'}.$ 

It will also be useful to provie some properties of this action.
\begin{theorem}
	\label{thm:stabW}
	The action of $W$ is transitive on simplices of $\Sigma$ of cotype $J.$ Furthermore, suppose $A$ is a simplex of $W$ of cotype $J$ which is a face of $w=wW_{\emptyset}.$ Then $\stab_W(A)=wW_Jw^{-1}.$
\end{theorem}

An immediate result is that $W$ acts simply transitively on the chambers of $\Sigma,$ which is no surprise given the definition of the action. An application is that when working with links or galleries, it is almost always good enough to assume that a simplex $A$ is a face of the fundamental chamber $C.$


\end{document}
