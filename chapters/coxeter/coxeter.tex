\documentclass[class=book, crop=false,12 pt]{standalone}
\usepackage[subpreambles=true]{standalone}
\usepackage{import}
\RequirePackage{/home/mark/Documents/gradschool/research/thesis/preamble}

%\newcommand{\leqnomode}{\tagsleft@true}
\newcommand{\cop}{\ensuremath{C^{\text{op}}}}

\begin{document}
\chapter{Coxeter Groups and Coxeter Complexes}
\label{ch:Coxeter}
The RGD systems and Kac-Moody groups discussed in the introduction have a large amount of geometric structure which we can use to study them. The foundations of this geometry come from the idea of reflection groups studied by H.S.M. Coxeter in the 1930's. These reflection groups, and their associate geometry, have particularly nice properties, and they provide the foundation for the study of RGD systems.

\section{Coxeter Groups}
\label{sec:coxgrp}
\begin{defn}
	\label{def:coxgrp}
	A Coxeter system is a pair $(W,S)$ such that $S$ is a finite set, and $W$ is a group with a presentation
	\[
		W=\langle s\in S|(st)^{m(s,t)}=1\text{ for all }s,t\in S\rangle
	\]
	subject to the conditions that $m(s,t)\in \mathbb{N}\cup \{\infty\},$ $m(s,s)=1,$ and $m(s,t)=m(t,s)\ge 2$ if $s\neq t.$ If $m(s,t)=\infty$ then we simply discard the relation $(st)^{m(s,t)}=1.$
\end{defn}

Through slight abuse of terminology we will refer to $W$ as a Coxeter group, but we will always have a specific generating set $S$ for the Coxeter system in mind. Coxeter groups have many nice properties, and far too many to discuss here, but we will mention a few which will be of use later. The first of which is the length function. If $(W,S)$ is a Coxeter system then we can define a function $\ell:W\to \mathbb{N}$ by $\ell(w)$ is the minimum number $n$ such that $w$ can be written as $w=s_1s_2\cdots s_n$ with $s_i\in S$ for all $i.$ This length function is standard in group theory, and can be defined on any group with any generating set. However, in Coxeter groups this length function takes on a much richer structure which we will describe in more detail.


\subsection{M-Operations}
We say that $(s_1,s_2,\dots,s_n)$ is a decomposition of $w$ if $w=s_1s_2\cdots s_n,$ and that it is a reduced decomposition if $n=\ell(w).$ Certainly decompositions, and even reduced decompositions need not be unique, and we will see some ways that we can generate new decompositions. By definition, $m(s,s)=1$ so $s^2=1$ for all $s\in S.$ Thus if we ever have an element of $s$ repeated twice in a row in a decomposition, we can simply delete the copies to get another decomposition with smaller length. If $s\neq t\in S$ then $(st)^{m(s,t)}=1$ and thus we can say
\[
	\underbrace{sts\cdots t(s)}_{m(s,t)}=\underbrace{tst\cdots s(t)}_{m(s,t)}
\]
This again means if we have any alternating string of $s$ and $t$ of the right length, then we can replace it with the swapped alternating string, and get another decomposition of the same length. These operations on decompositions are automatic from the presentation, and we will repeat Theorem 2.33 from \cite{buildings}, which shows these operations are all we need.

\begin{theorem}
	\label{thm:Mop}
	If $(W,S)$ is a Coxeter system and $(s_1,s_2,\dots,s_n)$ is a decomposition of $w,$ then we can obtain a new decomposition of $W$ by deleting a sub-string of the form $(s,s),$ or replacing a sub-string of length $m(s,t)$ of the form $(s,t,\dots,s(,t))$ with a sub-string of the form $(t,s,\dots,t(,s)).$ We will call these two operations \emph{M-Operations} of type 1 and 2 respectively. Furthermore, any decomposition of $w$ can be transformed into a reduced decomposition by repeated application of M-Operations of type 1 and 2, and any two reduced decompositions of $w$ can be transformed into one another by applications of M-Operations of type 2.
\end{theorem}

There are many consequences of Theorem \ref{thm:Mop} but one of the most notable is this, we have a simple algorithm to obtain a reduced decomposition of any $w,$ and we can always check if a decomposition is reduced. In either case we repeatedly apply any possible M-Operations, and applying those of type 1 if possible or noting if none are possible in the case of a decomposition which is already reduced. It also gives us some facts about the length function. For example, if we can write $w=s_{i_1}\cdots s_{i_k}$ then $\ell(w)$ and $k$ are either both even, or both odd, as application of type 1 operations will always reduce the length of a decomposition by 2.

\subsection{Standard Subgroups and Standard Cosets}
Coxeter groups also have a nice subgroup structure will give rise to the rich geometry we will use later. If $(W,S)$ is a Coxeter system then by definition $W$ is generated by $S.$ For any $J\subset S$ we can form a subgroup $W_J=\langle s|s\in J\rangle\le W.$ For example, $W_S=W$ and $W_{\emptyset}=\{1\}.$ We will also define a standard coset to be any coset of the form $wW_J$ for any $w\in W$ and $J\subset S.$

As before, there is nothing special about these definitions, as similar definitions hold for any group, but what is special is the structure on standard subgroups. Proposition 2.13 from \cite{buildings} tells us that the map which sends $J\to W_J$ is a bijection from subsets of $S$ to standard subgroups. Furthermore, if $H$ is a standard subgroup, then its $J$ can be recovered as $H\cap S.$ We can also check that $(W_J,J)$ is also a Coxeter system.

We can use Theorem \ref{thm:Mop} to derive some basic consequences about standard subgroups. For example, we can show that $W_J\cap W_J'=W_{J\cap J'}.$ One inclusion is clear, and if we take $w\in W_J\cap W_J'$ we can write two reduced decompositions of $w,$ one of which only uses letters from $J$ and the other only uses letters from $J'.$ These reduced decompositions can be transformed into one another by M-Operations of type 2, but M-Operations cannot introduce new letters into a reduced decomposition, only change the order. Thus every letter in the initial decompositions must be in $J$ and $J'.$

One situation which will be very useful later is when the group $W$ is finite. We say that a Coxeter Group or Coxeter System is \emph{spherical} if $W$ is finite. If $(W,S)$ is spherical then we can prove several facts. By Corollary 2.19 in \cite{buildings}, $W$ has a unique element of maximal length, which is usually denoted $w_0.$ It has the property that $\ell(ww_0)=\ell(w_0)-\ell(w)$ for every $w\in W.$ One consequence of this fact is that for any $w\in W$ a reduced decomposition of $w$ can be extended to a reduced decomposition of $w_0.$ This element of maximal length will be of some interest in the geometry of $W$ as well. In a similar fashion, we say that $J$ is a spherical subset of $S$ if $W_J$ is spherical. We also say that $W$ is 2-spherical if every subset of $S$ of size 2 is a spherical subset. This is equivalent to saying that $m(s,t)<\infty$ for every $s,t\in S.$

Let $\Delta$ be the set of all standard subgroups of $W,$ with a partial order given by reverse inclusion, so that $W_J\le W_{J'}$ if and only if $J'\subset J.$ Using the fact from the previous paragraph, one can check that $\Delta$ is isomorphic as a poset to the subsets of $S$ under reverse inclusion. This fact is the basis for our definition of the Coxeter Complex.

\section{Coxeter Complex}
\begin{defn}
	If $(W,S)$ is a Coxeter system, let $\Sigma$ be the collection of standard cosets of $W,$ ordered by reverse inclusion. Then $\Sigma$ is a simplicial complex called the Coxeter Complex of $W.$
\end{defn}

Before proceeding we should clarify what is meant by the following definition. A simplicial complex should be a collection of subsets $\mathcal{S}\subset \mathcal{P}(V)$ of some set $V$ such that $B\in \mathcal{S}$ whenever $A\in \mathcal{S}$ and $B\subset A.$ For any simplicial complex we can form a poset given by $\mathcal{S}$ ordered by inclusion. As described in Section A.1.1 of \cite{buildings}, posets satisfying certain properties can also be considered as simplicial complexes, where the vertices are the elements which are greater than exactly one other element, and the other faces are determined by which vertices they contain. Theorem 3.5 in \cite{buildings} proves that the poset $\Sigma$ satisfies the needed properties, and thus can be viewed as a simplicial complex.

In the standard terminology of simplicial complexes, we will refer to each standard coset as a simplex, and $A$ and $B$ are simplices with $A\le B$ then we say $A$ is a face of $B.$ One can check that the dimension of any simplex $wW_J$ will be $|S|-|J|-1$ because the ordering is by reverse inclusion. For this reason, it is sometimes more useful to refer to the rank of a simplex which is the number of vertices, and is also one more than the dimension, so that the rank of $wW_J=|S|-|J|.$ We can also draw several conclusions from this fact. First of all, every maximal simplex of $\Sigma$ has the same dimension, $|S|-1,$ and they will correspond exactly to the elements of $W$ by $w\mapsto wW_\emptyset.$ We can also see that the standard subgroup $W=W_S$ is a simplex of dimension $-1$ and of rank 0 which is a face of every other simplex.

Let $\Delta$ be a simplicial complex with vertices $V(\Delta),$ and let $I$ be any set. We say that $\tau:V(\Delta)\to I$ is a type function if the vertices of each maximal simplex map bijectively onto $I.$ If $\tau$ is a type function then by definition, each maximal simplex has the same rank and dimension. For any simplex $A,$ we can also extend the type function $\tau:\Delta \to \mathcal{P}(I)$ where $\tau(A)=\{\tau(v)|v\text{ is a face of }A\}.$ 

By Theorem 3.5 in \cite{buildings},the Coxeter complex is also equipped with a type function $\tau:\Sigma \to \mathcal{P}(S)$ by $\tau(wW_J)=S\setminus J.$ However, for convenience, we will more often refer to the \emph{cotype} of a simplex which is $S\setminus \tau(wW_J)=J.$ For example, maximal dimensional simplices will have cotype $\emptyset,$ and co-dimension 1 simplices will have cotype $\{s\}$ for some $s\in S.$ This convention is also convenient as simplices of cotype $J$ will have rank $|J|$ and dimension $|J|-1.$

We will call the maximal simplices of $\Sigma$ \emph{chambers} and the co-dimension 1 simplices of $\Sigma$ will be called \emph{panels}. A panel will have cotype $\{s\}$ for some $s\in S,$ or just cotype $s$ for short. If we take a look at a panel of cotype $s,$ we see that it is a standard subgroup of the form $wW_{\{s\}}=w\{1,s\}=\{w,ws\}.$ Thus each panel will be contained in exactly two chambers, corresponding to $w$ and $ws,$ and we will say that the chambers $w$ and $ws$ are $s$-adjacent. We say that two chambers are adjacent if they are $s$-adjacent for some $s\in S.$ We will also note that there is an obvious chamber which can be distinguished, namely the chamber $W_\emptyset=\{1\}.$ We will call this the \emph{fundamental chamber} of $\Sigma$ and denote it as $C.$

A \emph{gallery} in $\Sigma$ is a sequence of chambers $D_0,D_2,\dots,D_n$ such that $D_i$ and $D_{i+1}$ are adjacent for every $i.$ We will say that a subset $\D$ of chambers of $\Sigma$ is gallery connected if for all chambers $D,E\in \D,$ there is a gallery $D_0,\dots,D_n$ in $\D$ such that $D_0=D$ and $D_n=E.$ Since every chamber $E$ can be written as $wC$ for some $w\in W,$ we can make a gallery $D_0,\dots,D_n$ from $1$ to $w$ where $D_i=s_1\dots s_i$ for any decomposition $(s_1,\dots,s_n)$ of $w.$ This leads to the following proposition
\begin{prop}
	\label{prop:gallerycon}
	If $\Sigma$ is a Coxeter complex then it is gallery connected.
\end{prop}

It turns out that $\Sigma$ is sufficiently nice that the geometry of the lower dimension simplices can be recovered from the chambers of $\Sigma$ and from the $s$-adjacency relations. Thus we will rarely need to make arguments using simplices other than chambers or panels. This also means when considering subset of $\Sigma,$ we will instead use the chambers of $\Sigma,$ which we will denote $\mathcal{C}(\Sigma).$ In fact, for any sub-complex $\Delta',$ we will refer to the set of chambers of $\Delta'$ as $\mathcal{C}(\Delta).$ In the next chapter we will see how the chambers and notion of $s$-adjacency will be ``enough'' when discussing Coxeter complexes.

If $D$ and $E$ are chambers then a minimal gallery between $D$ and $E$ is a gallery of minimal length, that is, any other gallery between $D$ and $E$ is at least as long. Then we can turn $\mathcal{C}(\Sigma)$ into a metric space where $d(D,E)$ is the length of a minimal gallery between $D$ and $E.$ It is not so surprising that there is a direct link between galleries in $\Sigma$ and decompositions in $W.$ In fact, we have the following facts which can be found in \cite{buildings}. If $D=w$ and $E=w'$ are chambers of $\Sigma,$ then $d(D,E)=\ell(w^{-1}w').$ Furthermore, if $(s_{i_1},\dots,s_{i_n})$ is any decomposition of $w^{-1}w'$ then there is a gallery $D_0,\dots,D_n$ from $D$ to $E$ where $D_j$ is $s_{i_j}$ adjacent to $D_{j+1}$ for all $j.$ In this case the minimal galleries will correspond to reduced decompositions.

\subsection{Links and Stars}
We saw before that if $J\subset S$ then $(W_J,J)$ is also a Coxeter system. This structure will also carry over into the Coxeter complexes. Before giving the details, we need to define a few more terms. In any simplicial complex, we say that two simplices $A$ and $B$ are joinable if they are contained in a common maximal simplex. In term of the Coxeter complex $\Sigma,$ two simplices $A=wW_J$ and $B=w'W_{J'}$ are joinable if they share a common element $w.$ We can now make two more definitions which we will use extensively through the rest of the paper.

\begin{defn}
	If $A$ is a simplex of $\Sigma,$ then the star of $A$, $\st(A),$ is all of the simplices of $\Sigma$ which are joinable to $A.$ In terms of chambers $\mathcal{C}(\st(A))=\{w\in W|w\in A=w'W_J\}.$ We can also define the link of $A$, $\lk(A),$ as the set of all simplices of $\Sigma$ which are joinable to $A,$ but do not contain $A.$
\end{defn}

Previously we saw for an $J\subset S,$ that not only could we form the standard subgroup $W_J,$ but that $(W_J,J)$ was also a Coxeter system in it's own right. Proposition 3.16 in \cite{buildings} translates this concept in to Coxeter complexes as well.
\begin{prop}
	\label{prop:link}
	If $A$ is a simplex of $\Sigma$ of cotype $J,$ then $\lk(A)$ is isomorphic as a simplicial complex to the Coxeter complex $\Sigma_J$ of $(W_J,J).$
\end{prop}

We can define $\Sigma_{\ge A}$ to be the set of simplices in $\Sigma$ which contain $A.$ There is a bijection from $\lk(A)$ to $\Sigma_{\ge A}$ given by $B\mapsto B\cup A$ which is also an isomorphism as posets. Using this fact we can check that the chambers of $\st(A)$ will be in 1-1 correspondence with the maximal simplices of $\lk(A)$ which are also the chambers of $\Sigma_J.$ For a simplex $A,$ the star and link of $A$ will give more or less the same combinatorial information, and thus which one we use will be somewhat a matter of convenience.

Stars and links have other nice properties which we will take advantage of later. First of all $\mathcal{C}(\st(A))$ is gallery connected, and the galleries in $\st(A)$ correspond exactly to galleries in $\Sigma_J.$ Furthermore, suppose that $D_0,\dots,D_n$ is a minimal gallery between two chambers in $\st(A)$ where $A$ has cotype $J.$ Then we know that $D_i$ and $D_{i+1}$ are $s_i$ adjacent for some $s_i\in S.$ But in fact, $s_i\subset J$ for every $i.$ In fact, the types of these adjacencies is exactly the same as those in the minimal gallery of $\Sigma_J.$ 

We say that a Coxeter complex $\Sigma$ is spherical or 2-spherical if $W$ is spherical or 2-spherical. If $\Sigma$ is spherical then we will define $C^{\text{op}}$ to be the chamber of $\Sigma$ corresponding to $w_0.$ Then $\cop$ is the unique chamber of $\Sigma$ at maximal distance from $C,$ and it has the property that every chamber of $\Sigma$ is part of a minimal gallery from $C$ to $\cop.$ 

Now suppose that $\Sigma$ is a 2-spherical Coxeter complex, and let $A$ be a simplex of $\Sigma$ of co-dimension 2. Then $A$ is a simplex of cotype $J=\{s,t\}$ for some $s,t\in S.$ By definition of 2-spherical, this means $W_J$ is spherical and thus there are finitely many chambers in $\st(A).$ Every chamber in $\st(A)$ also has a unique chamber at maximal distance away in $\st(A)$ which we will call opposite in $\st(A).$ If we examine the structure of $W_J$ we can even see that it is the dihedral group of order $2m(s,t),$ and the simplicial complex $\Sigma_J$ will be a $2m(s,t)$-gon with edges as chambers and vertices as panels. Translating to $\Sigma$ this means that $\st(A)$ consists of $2m(s,t)$ chambers arranged in a circular pattern around $A,$ and opposite chambers in $\st(A)$ will be at distance $m(s,t)$ away from each other. 

\subsection{Projections}
Another useful tool for studying the geometry of $\Sigma$ is the concept of projections. Proposition 3.105 from \cite{buildings} yields the following theorem.

\begin{theorem}
	If $A$ is a simplex of $\Sigma,$ and $D$ is a chamber of $\Sigma,$ then there is a chamber $E\in \st(A)$ such that $d(D,E)\le d(D,E')$ for all $E'\in \st(A).$ Additionally, the chamber $E$ is unique and we define the projection of $D$ on to $A,$ or $\proj_A(D)$ to be the chamber $E.$ The projection $E$ is also characterized by the property that $d(D,E')=d(D,E)+d(E,E')$ for all $E'\in \st(A).$
\end{theorem}

The property $d(D,E')=d(D,E)+d(E,E')$ is known as the gate property because it means for any $E'\in \st(A),$ there is a minimal gallery from $D$ to $E'$ which passes through $E.$ Projections also allow us to define a notion of convexity in a Coxeter complex.

\begin{defn}
	\label{defn:convex}
	We say that a sub-complex $\Delta$ of $\Sigma$ is convex, if $\proj_{A}(D)\in \Delta$ whenever $A$ is a simplex of $\Delta$ and $D$ is a chamber of $\Delta.$
\end{defn}

Convexity also has another interpretation, which can be taken as the definition if desired. A chamber sub-complex $\Delta$ of $\Sigma$ is convex if for any chambers $D,E$ of $\Delta,$ any minimal gallery from $D$ to $E$ in $\Sigma$ is contained in $\Delta.$ This means that we can look for minimal galleries in a convex chamber sub-complex of $\Sigma,$ and still be sure that it will be minimal in all of $\Sigma.$ One of the most common uses for this is to apply the result to the convex chamber sub-complex $\st(A)$ for some simplex $A.$ In particular, if $A$ is a simplex of cotype $\{s,t\}$ then we can find minimal galleries from $D$ to $E$ by examining the Dihedral group $D_{2m(s,t)}$ which is not too challenging.

\subsection{Roots}
Intuitively we should think of Coxeter groups as reflection groups in some space, which is how they were originally considered. If we think about a reflection in Euclidean space, then there should be a few properties that are satisfied. A reflection should divide our space into two halves, which are interchanged by the reflection. Furthermore, there should be some set of fixed points of the reflection which would ideally have co-dimension 1. While it is natural to view some Coxeter groups as reflections in Euclidean space, in this section we will show how the notion of roots will generalize these ideas to any Coxeter group.

\begin{defn}
	\label{defn:root}
	For any adjacent chambers $D,D',$ let $\alpha_{D,D'}$ be the sub-complex of $\Sigma$ defined by $\mathcal{C}(\alpha_{D,D'})=\{E\in \Sigma|d(E,D)<d(E,D')\}.$ Then $\alpha_{D,D'}$ is called a root, and the collection of all $\alpha_{D,D'}$ for adjacent chambers $D$ and $D'$ are called the roots of $\Sigma.$
\end{defn}

We will denote the set of all roots of $\Sigma$ by $\Phi.$ If $D$ and $D'$ are adjacent then for any gallery from $E$ to $D,$ there is a gallery from $E$ to $D'$ of length 1 more. As a consequence of Theorem \ref{thm:Mop}, it is impossible for $d(E,D)=d(E,D')$ for any chamber $E,$ and thus the condition $d(E,D)<d(E,D')$ is equivalent to $d(E,D)\le d(E,D').$ If $D$ and $D'$ are adjacent chambers then both $\alpha_{D,D'}$ and $\alpha_{D',D}$ will be roots, and we will have $\mathcal{C}(\alpha_{D,D'})\cap \mathcal{C}(\alpha_{D',D})=\emptyset$ and $\mathcal{C}(\alpha_{D,D'})\cup \mathcal{C}(\alpha_{D',D})=\mathcal{C}(\Sigma).$

The roots $\alpha_{D,D'}$ and $\alpha_{D',D}$ are very closely related, and roughly correspond to the two half spaces defined by a reflection. To differentiate between these roots, we say a root is \emph{positive} if it contains the fundamental chamber $C.$ This choice is of course arbitrary, but the chamber $C$ is a convenient choice. Similarly, we say a root is negative if it does not contain $C,$ and we say that $\alpha_{D,D'}$ and $\alpha_{D',D}$ are opposite roots. We will also denote this with the notation $\alpha_{D',D}=-\alpha_{D,D'}.$

If roots are roughly analogous to the half spaces defined by a reflection, then we should also have some notion of the reflection line. If $\alpha$ is a root of $\Sigma$ then we define the \emph{wall} of $\alpha,$ denoted by $\partial\alpha$ or $\mathcal{H}_\alpha,$ to be $\alpha\cap (-\alpha).$ Then certainly $\partial \alpha$ will contain no chambers, but will not be non-empty, as the panel contained in $D$ and $D'$ will be in $\partial\alpha$ if $\alpha=\alpha_{D,D'}.$ Since the maximal simplices of $\partial\alpha$ will be co-dimension 1 in $\Sigma,$ this is consistent with our intuition that reflections should have a co-dimension 1 set of fixed points.

There are several facts about roots and walls which we will use later. By Proposition 3.94 from \cite{buildings}, every root is gallery connected, and is also a convex chamber sub-complex of $\Sigma.$ Furthermore, if $\D$ is a collection of chambers of $\Sigma,$ then $\D$ is convex if and only if $\D$ is the intersection of $\mathcal{C}(\alpha)$ for some collection of roots $\{\alpha\}.$ 

What is even more interesting is the interaction between roots and links. Suppose $A$ is a simplex of $\Sigma$ of cotype $J.$ Then we can recall that $\lk(A)\cong \Sigma_J$ where $\Sigma_J$ is is a Coxeter complex for $(W_J,J).$ Then there is a natural correspondence between roots in $\Sigma$ to roots in $\lk(A)$ as described in Proposition 3.79 of \cite{buildings}. The map $\alpha \to \alpha \cap \lk(A)$ is a bijection between the roots of $\Sigma$ such that $A\in \alpha,$ and the roods of $\lk(A)$ viewed as a Coxeter complex in its own right. Furthermore, this map is also a bijection between walls as well. These results further reiterate the fact that when working in $\lk(A),$ we can essentially forget about the rest of $\Sigma$ and consider only the Coxeter complex for $(W_J,J).$ This will be especially useful when discussing links of co-dimension 2 simplices.

There is also a connection between roots and the distance function $d.$ By Proposition 3.78 in \cite{buildings}, for any two chambers $D,E,$ the distance $d(D,E)$ is equal to the number of roots $\alpha$ such that $D\in \mathcal{C}(\alpha)$ and $E\not\in \mathcal{C}(\alpha).$ Equivalently, $d(D,E)$ is equal to the number of walls which separate $D$ and $E.$

While discussing roots we should also discuss intervals of roots which will become important later on when discussing RGD systems. We say that two roots $\alpha$ and $\beta$ are pre-nilpotent if both $\alpha\cap \beta$ and $(-\alpha)\cap (-\beta)$ contain a chamber. In this case we define $[\alpha,\beta]=\{\gamma\in \Phi|\alpha\cap \beta \subset \gamma \text{ and } (-\alpha)\cap (-\beta)\subset -\gamma\}$ and $(\alpha,\beta)=[\alpha,\beta]\setminus {\alpha,\beta}.$ While these definitions seem arbitrary, they will be very important in later chapters, and they also have very nice interpretations in the context of links. Suppose $\alpha$ and $\beta$ form a pre-nilpotent pair. Then Lemma 8.42 of \cite{buildings} says that either $\partial \alpha$ and $\partial \beta$ will meet, or $\alpha$ and $\beta$ are nested, meaning $\alpha\subset \beta$ or $\beta\subset \alpha.$

If $\alpha$ and $\beta$ are nested then we can think of $\partial \alpha$ and $\partial\beta$ as parallel hyperplanes in some space. Then the closed interval $[\alpha,\beta]$ is the set of roots whose walls lie between $\partial\alpha$ and $\partial\beta.$ If $\partial \alpha$ and $\partial\beta$ meet then we can choose a maximal simplex $A$ of $\partial\alpha\cap\partial\beta.$ As described in Example 8.44 of \cite{buildings}, the simplex $A$ will have have co-dimension 2, and $\lk(A)$ will be a rank 2 buildings, which is a $2n$-gon. In this case, intervals of roots can be described by enumerating the roots around a $2n$-gon in clockwise or counterclockwise order, and then lifting those roots to $\Sigma.$ It is also worth noting that the condition on pre-nilpotence is natural in the context of spherical Coxeter complexes, where two roots $\alpha$ and $\beta$ are pre-nilpotent as long as $\beta \neq -\alpha.$

Thus far we have discussed many properties and attributes of $\Sigma,$ but we have not really described how the group theory of $W$ interacts with $\Sigma$ besides in the notion of galleries. In the next section we will see that we can say much more about the interaction of the group $W$ and the Coxeter complex $\Sigma.$

\section{W-Action}
\begin{prop}
	\label{prop:wact}
	There is a well defined action of $W$ on $\Sigma$ by $w'(wW_J)=w'wW_J.$ Then each $w\in W$ induces an isomorphism of $\Sigma$ which also preserves (co)type of each simplex.
\end{prop}

As $\Sigma$ is built directly from $W,$ it is unsurprising that this $W$ action plays very nicely with the geometry of $\Sigma,$ and we will briefly collect the more relevant facts. The $W$-action sends galleries to galleries, minimal galleries to minimal galleries, and thus $d(D,E)=d(wD,wE)$ for all $D,E\in \mathcal{C}(\Sigma)$ and $w\in W.$ 

Because of how natural our definition of the $W$ action is, we can also check relatively easily that $W$ interacts nicely with all of the concepts we have defined so far. If $A$ is a simplex and $D$ is a chamber then we have $\proj_{wA}(wD)=w\proj_{A}(D)$ for all $w\in W.$ If $\alpha$ is a root of $\Sigma$ then $w\alpha$ is also a root with wall $w\partial\alpha.$ Furthermore, if $\partial\alpha$ is a wall which separates $D$ and $D'$ then $w\partial\alpha$ will separate $wD$ and $wD'.$ This also means $w\alpha_{D,D'}=\alpha_{wD,wD'}.$ 

It will also be useful to provide some properties of this action. The first result is immediate from the definition of the action.
\begin{theorem}
	\label{thm:stabW}
	The action of $W$ is transitive on simplices of $\Sigma$ of cotype $J.$ If $A=wW_J$ is a simplex in $\Sigma$ of cotype $J,$ then $\stab_W(A)=wW_Jw^{-1}.$ In particular, $W$ acts simply transitively on the chambers of $\Sigma.$ 
\end{theorem}

The previous theorem also shows, among other things, that $W$ will send links to links of the same type. There is also nice interplay between this $w$ action and the roots of $\Sigma.$ In light of Theorem \ref{thm:stabW} we can identify the chambers of $\Sigma$ with the set of all $wC$ for $w\in W$ where $C$ is the fundamental chamber. Note, this identification also agrees with the identification $wW_\emptyset\mapsto w.$ We can check that the chambers $wC$ and $wsC$ are $s$-adjacent. We can define a root $\alpha_s=\{D\in \mathcal{C}(\Sigma)|d(D,C)<d(D,sC)\}=\{w\in W|\ell(w)<\ell(sw)\}.$ The root $\alpha_s$ is positive, and the wall $\partial\alpha_s$ is also the set of all simplices fixed by $s.$ 

For any $w\in W$ and $s\in S,$ $w\alpha_s$ will also be a root which contains $w$ but not $ws.$ Since any adjacent chambers $D$ and $D'$ must be $s$ adjacent for some $s\in S,$ we can also say that $D=wC$ and $D'=wsC,$ and thus $\alpha_{D,D'}=w\alpha_s.$ This shows every root is a $W$ translate of $\alpha_s$ for some $s\in S,$ and similarly every wall is a $W$ translate of some $\partial\alpha_s.$ 

\end{document}
