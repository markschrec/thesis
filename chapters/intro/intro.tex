\documentclass[class=book, crop=false,12 pt]{standalone}
\usepackage[subpreambles=true]{standalone}
\usepackage{import}
\RequirePackage{/home/mark/Documents/gradschool/research/thesis/preamble}

%\newcommand{\leqnomode}{\tagsleft@true}

\begin{document}
\chapter{Introduction}
\label{ch:intro}
The classification of finite simple groups is one of the foundational problems in group theory, and central to this problem is the study of finite Lie groups. Jacques Tits introduced and developed the theory of buildings in the 1950's and 1960's as a way to study and classify these groups. In the 1980's, these ideas were extended by Tits and Mark Ronan to study Kac-Moody groups, with the introduction of twin buildings. Kac-Moody groups can be described by some group functor $\mathcal{G}$ and a field $k,$ which give the Kac-Moody group $\mathcal{G}(k).$ These groups act as an infinite dimensional analogy of semisimple Lie groups and share many nice properties. 

All of the motivating examples in the previous paragraph can be generalized with the notion of RGD systems and Moufang twin buildings. RGD systems are equipped with a Coxeter group $W,$ which is the Weyl group, and a collection $\{U_\alpha\}$ of subgroups associated to every root of $W.$ If $(G,(U_\alpha)_{\alpha\in \Phi},T)$ is an RGD system, then there are several natural subgroups of interest. We can define the subgroup $U_+$ of $G$ to be the subgroup generated by the positive root groups $U_\alpha$ for all the positive roots of $W.$ A consequence of the RGD axioms is that $U_+$ admits a nice presentation in terms of the geometry of $W.$ When the Weyl group $W$ is infinite then this presentation will be defined in terms of infinitely many generators, but this does not preclude the possibility that $U_+$ is still finitely generated. Our main goal in the following will be to show when $U_+$ is in fact finitely generated and when it is not.

For a twin building $\Delta,$ we will say that $\Delta$ satisfies condition (co) if the chambers opposite $C_-,$ form a gallery connected subset of the positive half of the twin building. Results in \cite{cop} and \cite{jtco} show that $U_+$ is finitely generated, and even gives an explicit generating set when that building associated to $U_+$ satisfies condition (co). Furthermore, \cite{globalco} shows that a twin building will satisfy condition (co) when all of its rank 2 residues satisfy condition (co). Finally, \cite{cop} proves that every rank 2 residue will satisfy condition (co), unless it is the Moufang polygon associated to one of the groups $C_2(2),G_2(2),G_2(3)$ or ${}^2F_4(2).$ Much of the general theory regarding RGD systems, and thus Kac-Moody groups, relies on the (co) condition, and makes assumptions like $|U_\alpha|\ge 4$ for all $\alpha$ to insist it will be satisfied. In this paper we will consider the cases where the rank 2 residues of $\Delta$ do not satisfy (co), and thus we cannot rely on (co) to be satisfied in $\Delta.$ 

Our main approach will be to use the properties of these exceptional rank 2 residues to show that $U_+$ is not finitely generated. A consequence of the failure of condition (co) is that for each vertex $v$ where (co) fails, the subgroup $U_v\le U_+$ generated by roots at $v$ will have a proper normal subgroup of small index. We can use these normal subgroups to define homomorphisms which are surjective and where the kernel is well understood. Since we have a presentation of $U_+,$ we will attempt to extend these homomorphisms to all of $U_+$ in a way that does not help with surjectivity, namely if all other root groups $U_\alpha$ are sent to the identity. If this extension is well defined, and remains surjective, then we should be able to say roughly that any generating set of $U_+$ will need to contain a root group $U_\alpha$ for some positive root $\alpha$ which goes through $v.$ If we have enough vertices where we can do this, we will be able to show that $U_+$ is not finitely generated.

In Chapters \ref{ch:Coxeter} and \ref{ch:buildings}, we will introduce the necessary theory of Coxeter groups, Coxeter complexes, and buildings that we will use in the remainder of the paper. In particular, results about roots, projections and links will be used extensively in the proof of Theorem \ref{thm:notfg} and its preceding lemmas. One result of particular interest will be that concerning M-Operations. We will see that we can write elements of a Coxeter group $W$ in a mostly canonical way, which will allow us to determine when any word in a Coxeter group is trivial, or contained in special subgroups. We will use this in the proof of Lemma \ref{lem:infmany} and \ref{lem:samewall}, when we translate geometric properties of Coxeter complexes, into group theoretic properties of $W.$

In Chapter \ref{ch:rgd}, we will discuss the theory of RGD systems and BN-Pairs, which explain the connections between buildings and the groups of interest. It will show how we can construct buildings from RGD systems and give properties of the action. These connections will allow us to use the geometry from Chapters \ref{ch:Coxeter} and \ref{ch:buildings} to study the group theory of $G.$ We will also see some common examples of RGD systems with Lie groups and Kac-Moody groups.

Chapter \ref{ch:known} will explore the properties of the exceptional rank 2 residues, and will construct the main tools we use when proving the main results. Specifically we give presentations for the exceptional groups $U_v$ and use them to define the homomorphisms $\phi_v$ to be extended to all of $U_+.$ These homomorphisms will interact nicely with the geometry and group theory of Weyl group $W.$ We will also state the triangle condition, which is a result about the geometry of certain Coxeter complexes showing that triangles formed by the walls of a Coxeter complex, must be single chambers. This will be used when we define our extension maps as we will need to prove properties about the intersections of roots.

In the last two chapters we will record the main results about finite generation in RGD systems. Chapter \ref{ch:general} will introduce condition \eqref{assume} on certain RGD systems, and discuss how \eqref{assume} is satisfied by canonical examples like Kac-Moody groups. Lemma \ref{lem:Dexists} will give us a way to construct extension maps for a large family of vertices. Then we can give a necessary condition for the group $U_+$ to not be finitely generated as described in Theorem \ref{thm:notfg}.

Chapter \ref{exceptional} starts by examining the rank 3 cases not covered by Chapter \ref{ch:general} and explaining why the same approach will not work. There are three remaining cases which will be enumerated in this chapter. The first one will not be finitely generated, and the arguments there will be adaptations of those in Chapter \ref{ch:general} with slight modification. In the rest of the Chapter we will use the presentation of $U_+$ and the presentations given in Chapter \ref{ch:known} to prove that the remaining cases will yield a $U_+$ which is in fact finitely generated. We will end by discussing future work including minimal generation in the finitely generated cases, as well as possibilities to extend these results to RGD systems not satisfying \eqref{assume}, especially to higher rank cases.

Before starting, we will try to motivate our curiosity in the group $U_+.$ Finite generation is a fundamental problem in group theory, and so determining answers to questions about finite generation are always interesting from a group theoretic point of view. This question is also specifically of interest because the general theory leaves such a specific gap, and it would be nice to complete the characterization of when the group $U_+$ is finitely generated. The group $U_+$ is equipped with a nice presentation that makes it easy to work with, but it also allows us to extend our results found here to other special subgroups of $G.$

If $(G,(U_\alpha)_{\alpha\in \Phi},T)$ is an RGD system then the subgroup $B_+=TU_+$ is the stabilizer of the fundamental chamber $C_+$ in one half of the twin building. This subgroup is known as the Borel subgroup, and can be seen in classical examples like $\mathrm{SL}_n(k)$ as the upper triangular matrices. If $T$ is finite, as in the case for Kac-Moody groups over finite fields, then $[B_+:U_+]<\infty,$ and thus $B_+$ is finitely generated if and only if $U_+$ is finitely generated. Knowing stabilizers for certain group actions is an important tool for understanding other finiteness properties of $B_+$ and $G.$ 

We can also define parabolic subgroups $P_J$ for every standard subgroup $W_J$ of the Weyl group $W.$ These subgroups will correspond to stabilizers of lower dimensional simplices in $\Delta.$ These subgroups have a very nice decomposition, but without going into too many details here it will suffice to say that each parabolic subgroup $P_J$ has a finite set $\{g_i\}\subset G$ such that $P_J=\langle Bg_iB\rangle.$ In particular, if $B_+$ is finitely generated then so is $P_J$ for every proper parabolic subgroup $P_J$ of $G$ such that $W_J$ is finite.


\end{document}
