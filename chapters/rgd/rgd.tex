\documentclass[class=book, crop=false,12 pt]{standalone}
\usepackage[subpreambles=true]{standalone}
\usepackage{import}
\usepackage{/home/mark/Documents/gradschool/research/thesis/preamble}

%\newcommand{\leqnomode}{\tagsleft@true}

\begin{document}
\chapter{Group Actions on Buildings}
\label{ch:rgd}
In the first chapter we saw the interplay between the group theory of $W$ and the geometry of $\Sigma.$ In the previous chapter we developed the geometry of buildings, and we will now explore the the group theoretic consequences of groups acting on a building. 

Some arbitrary group action on a building $\Delta$ will not be enough to say much, so we neeed to restrict our attention to stronger group actions. Throughout this chapter we will assume that we have a group $G$ acting on a building $\Delta$ and the action is both simplicial and type preserving. We will also assume that $\mathcal{A}$ is a system of apartments for $\Delta$ such that $g\Sigma\in \mathcal{A}$ for each $g\in G$ and $\Sigma\in \mathcal{A}.$ We will not discuss the details of apartment systems, but it should be noted that this is always possible using the complete system of apartments as described in Theorem 4.54 of \cite{buildings}. 

\section{BN-Pairs}
We say that the group $G$ acts \emph{strongly transitively} if $G$ acts transitively on pairs $(\Sigma,C)$ where $\Sigma$ is an apartment of $\Delta,$ and $C$ is a chamber of $\Delta$ contained in $\Sigma.$ Equivalently, $G$ acts strongly transitively if $G$ acts transitively on chambers, and for any chamber $C,$ the stabilizer of $C$ acts transitively on apartments containing $C,$ or if $G$ acts transitively on apartments, and for any apartment $\Sigma,$ the stabilizer of $\Sigma$ acts transitively on the chambers of $\Sigma.$ Assume for the rest of the chapter that any group action on a building is strongly transitive.

Choose a chamber $C$ and an apartment $\Sigma$ containing $\Sigma$ which we will fix and call the fundamental chamber and fundamental apartment respectively. We can define several subgroups of $G$ which will be the basis of most of the section. Define subgroups
\begin{align*}
	B&=\{g\in G|gC=C\}
	N&=\{g\in G|g\Sigma=\Sigma\}
\end{align*}

We can make a few remarks which can all be found in section 6.1.1 of \cite{buildings}. First of all, there is a natual, type preserving action of $N$ on $\Sigma,$ and since the type preserving automorphisms of $\Sigma$ are exactly those induced by $W,$ we get a homomorphism $\phi:N\to W$ which is surjective by strong transitivity. Let $T=\ker \phi$ be the elements of $N$ which fix $\Sigma$ pointwise. Then $N/T\cong W,$ and furthermore, $N/T$ has a canonical choice of generators by taking representatives in $N$ which send the fundamental chamber $C$ to adjacent chambers. Since the action of $G$ is type preserving, we can check that any element of $G$ which stabilizes $\Sigma$ and $C$ will fix $\Sigma$ pointwise, and thus $T=B\cap N.$ Finally, $G$ is generated by $B$ and $N,$ and we can even show $G=BNB.$

Similar to the case with Coxeter groups and Coxeter complexes, we would like to be able to move between the group theory of $G$ and the geometry of $\Delta.$ Before we can do this we need to make a few more remarks. As previously stated, $G=BNB.$ We also know that there is a surjection of $N$ onto $W$ with kernel $T=B\cap N.$ This means that for any $w\in W,$ there is some lift $\tilde{w}\in N$ which is sent to $w,$ and any other lift will differ by an element of $T\subset B.$ This allows us to unambiguously write expressions of the form $BwB$ which is understood to mean the double coset $B\tilde{w}B$ for any lift of $w.$ Furthermore, the map $W\to B\backslash G/B$ by $w\mapsto BwB$ is a bijection.

Proposition 6.27 in \cite{buildings} also says that we have a nice description of stabilizers of lower dimensonal simplices. If $A$ is a face of $C$ of cotype $J,$ then the stabilizer of $A$ is $P_J=\cup_{w\in W_J}BwB.$ In particular, $P_J$ is a subgroup of $G$ for all $J\subset S,$ and we will refer to them as standard parabolic subgroups, while cosets of the form $gP_J$ will be called standard parabolic cosets. Corollary 6.29 of \cite{buildings} says, similar to Coxeter complexes, the building $\Delta$ can be recovered as the poset of standard parabolic cosets of $G,$ ordered by reverse inclusion. While the result is similar to that for Coxeter complexes, it is worth nothing that this result goes in the opposite direction, with Coxeter complexes we defined a simplicial complex from the coset data, while here we already had the simplical complex data, and simply recovered it from the group theory. Before moving on we will explore the conditions necessary to actually construct a building from group theoretic data.

\begin{defn}
	\label{defn:bnpair}
	A pair of subgroups $B$ and $N$ of a group $G$ is a \emph{BN-Pair} if $G=\langle B,N\rangle,$ $T=B\cap N$ is normal in $N,$ and the quotient $W=N/T$ admits a set of generators $S$ such that $sBw\subset BswB\cup BwB$ and $sBs^{-1}\not\subset B$ for all $s\in S$ and $w\in W.$ In this case we also say that the tuple $(G,B,N,S)$ is a \emph{Tits System}.
\end{defn}

Despite the suggestive notation, we do not assume that the elements of $S$ have order 2, or even that $W$ is a Coxeter group. These are results which follow from the axioms, as well as others which can be found in Theorem 6.56 of \cite{buildings}. If $(G,B,N,S)$ is a Tits system, then $(W,S)$ is a Coxeter system, and there is a thick building $\Delta(G,B)$ on which $G$ acts strongly transitively with $B$ the stabilizer of the fundamental chamber, and $N$ the stabilizer of the fundamental apartment. Furthermore, if $G$ acts strongly transitively on $\Delta,$ then $B$ and $N$ as defined before form a BN-pair, and the building $\Delta$ is canonically isomorphic to $\Delta(G,B).$

Before moving on it is worth giving at least one example of a BN-Pair. Let $G=\mathrm{GL}_n(k)$ where $n\ge 2$ and $k$ is a field. Then one can let $B$ be the set of upper triangular matrices, and $N$ the set of permutation matrices, which are matrices with exactly one non-zero element in each row and column. The elements $S$ will be the permutation matrices which swap the $i$ and $i+1$ position. The rest of the axioms can be checked with linear algebra, but it is also of interest what building this group acts on. The complete construction can be found in section 4.3 of \cite{buildings}, but there is a way to associate a building to any vector space $V.$ For any vector space $V,$ we can define a building $\Delta(V)$ where the chambers are complete flags in $V,$ and the apartments roughly correspond to unordered bases of $V.$ This is consistient as the upper triangular matrices $B$ are exactly those that stabilize the standard flag, and the permutation matrices are those that preserve the standard unordered basis.

\section{Moufang Buildings and RGD Systems}
We saw in the previous section that groups acting strongly transitively on a building have a great deal of group theoretic structure. In this section we will explore additional restrictions which can be placed on these group actions to be able to draw even more consequences, and then apply the results to common examples including Kac-Moody groups.

For now, let $\Delta$ be a thick sperical building. We will later cover how to generalize the results in the non-spherical case. In the previous section we discussed properties of certain group actions on a building, but we gave no indication on how these actions arise. There is however one group which always acts nicely on a building, namely $\mathrm{Aut}_0(\Delta),$ the group of type preserving automorphisms of $\Delta.$ For any root $\alpha$ of $\Delta$ we can then define the root group $U_\alpha$ to be the subgroup of $\mathrm{Aut}_0(\Delta)$ which fixes $\alpha$ pointwise, and fixes $\st(P)$ pointwise for any panel of $\Delta$ in $\alpha\setminus \partial\alpha.$

Recall that $\alpha$ is a root of $\Delta$ if it is a chamber subcomplex of $\Delta$ which is a root in any apartment which contains it. Define $\mathcal{A}(\alpha)$ to be the set of apartments of $\Delta$ which contain $\alpha.$ If $P$ is a panel contained in $\partial\alpha$ we can also define $\mathcal{C}(P,\alpha)$ to be the the set of chambers in $P$ which do not lie in $\alpha.$ Remark 4.118 in \cite{buildings} says that for any chamber $D$ in $\mathcal{C}(P,\alpha)$ there is a unique apartment of $\Delta$ containing $D$ and $\alpha,$ and thus there is a canonical bijection from $\mathcal{C}(P,\alpha)$ to $\mathcal{A}(\alpha).$ We can also see from the definition that $U_\alpha$ will act on both $\mathcal{C}(P,\alpha)$ and $\mathcal{A}(\alpha).$ Lemma 7.25 in \cite{buildings} says that these actions are equivalent under the canonical bijection, and that these actions are also free as long as the Coxeter diagram of $W$ has no isolated nodes. 

We say that a building is \emph{Moufang} if the action of $U_\alpha$ on $\mathcal{A}(\alpha)$ on is transitive, and it is \emph{strictly Moufang} if the action is simply transitive. By the previous remarks, a Moufang building is guaranteed to be strictly Moufang as long as the Coxeter diagram has no isolated nodes. 

We would like some way to relate these root groups to strongly transitive actions from the previous section, and we do get this. Let $\Sigma$ be the fundamental apartment of $\Sigma$ and $\Phi$ its set of roots. Assume that $\Delta$ is Moufang and let $G=\langle U_\alpha|\alpha\in \Phi\rangle.$ Then Proposition 7.28 from \cite{buildings} says that the group $G$ will act strongly transitively on $\Delta$ with respect to the apartment system $\mathcal{A}=\{g\Sigma|g\in G\}.$ It can also be checked that $gU_\alpha g^{-1}=U_g\alpha$ for any $g\in \mathrm{Aut}(\Delta).$ If $\beta$ is any root of $\Delta$ then there is some $g\in G$ such that $g\beta\subset \Sigma$ since $G,$ acts strongly transitively, and thus we have $gU_\beta g^{-1}=U_{g\beta}\le G.$ This means $G=\langle U_{\alpha}|\alpha \text{ is a root of }\Delta\rangle$ and $G$ does not depend on the choice of $\Sigma$ as described in Remark 7.29 of \cite{buildings}.

In Chapter \ref{ch:buildings} we discussed links and the relationship between apartments of $\Delta$ and apartments of $\lk(A).$ Recall that for any simplex $A$ of $\Delta$ of cotype $J$ that $\lk(A)$ is a building of type $(W_J,J).$ Furthermore, if $\Sigma$ is an apartment of $\Delta$ which contains $A$ then $\Sigma\cap \lk(A)$ is an apartment of $\lk(A).$ Given the connection between roots and apartments, it is not surprising that the roots of $\lk(A)$ are given by $\lk(A)\cap \alpha$ for all roots $\alpha$ of $\Delta$ with $A\in \partial \alpha.$

Suppose that $A$ is a simplex of $\Delta$ and let $\Delta'=\lk(A).$ Suppose that $\alpha$ is a root of $\Delta$ with $A\in \partial \alpha$ and let $\alpha'$ be the corresponding root of $\Delta'.$ If $P'$ is a panel on $\partial \alpha'$ then $P=P'\cup A$ is a panel on $\partial \alpha.$ As described in section 7.3.2 of \cite{buildings}, we get a homomorphism $\rho:U_\alpha \to U_{\alpha'}$ given by the restriction of the action of $U_{\alpha}$ on $\Delta'.$ There is also a bijection between $\mathcal{C}(P',\alpha')$ and $\mathcal{C}(P,\alpha)$ given by $C'\mapsto C'\cup A.$ The consequence, if $\Delta$ is Moufang, then $\Delta'=\lk(A)$ is also Moufang, and if $\Delta$ is strictly Moufang and the Coxeter diagram for $\Delta'$ has no isolated nodes then $\Delta'$ is also strictly moufang and $\rho$ is an isomorphism. 

As with BN-Pairs, we would like to develop the theory that allows us to construct a Moufang building from group theoretic data which we will do in the next section.

\subsection{RGD Systems}
Throughout the section $(W,S)$ will be a spherical Coxeter system and $\Phi$ will be the set of roots of the Coxeter complex $\Sigma$ of type $(W,S).$ Let $(G,(U_\alpha)_{\alpha\in \Phi},T)$ be a triple consisting of a group $G,$ a family of subgroups $U_\alpha$ for each root of $\Phi,$ and another subgroup $T.$ Let $\Phi_{\pm}$ denote the set of positive (negative) roots, and $U_{\pm}=\langle U_\alpha|\alpha\in \Phi_{\pm}\rangle.$ We also know that for every $s\in S$ there is a root $\alpha_s=\{D\in \Sigma|d(C,D)<d(sC,D)\},$ and we will let $U_s=U_{\alpha_s}.$ 

We say that $(G,(U_{\alpha})_{\alpha\in \Phi},T)$ is an RGD system of type $(W,S)$ if the following conditions hold:
\begin{enumerate}
	\item[RGD0] $U_\alpha\neq \{1\}$ for all $\alpha\in \Phi$
	\item[RGD1] $[U_\alpha,U_\beta]\le U_{(\alpha,\beta)}=\langle U_\gamma|\gamma\in (\alpha,\beta)\rangle$ whenever $\alpha\neq \pm \beta$
	\item[RGD2] For every $s\in S,$ there is a function $m:U_s^*\to G$ such that $m(u)\in U_{-s}uU_{-s}$ and $m(u)U_\alpha m(u)^{-1}=U_{s\alpha}$ for all $\alpha\in \Phi.$ Furthermore, $m(u)^{-1}m(t)\in T$ for all $u,t\in U_s^*.$
	\item[RGD3] $U_{-s}\not\le U_+$ for all $s\in S.$
	\item[RGD4] $G=T\langle U_\alpha|\alpha\in \Phi\rangle$
	\item[RGD5] $T\le \cap_{\alpha\in \Phi}N_G(U_\alpha)$
\end{enumerate}

Based on our setup, it should be of no surprise that our first example of RGD systems comes from strictly Moufang buildings. Suppose $\Delta$ is a spherical, strictly Moufang building of type $(W,S).$ Fix an apartment $\Sigma$ and let $U_\alpha$ be the root group of $\alpha$ for each root of $\Phi.$ Let $G=\langle U_\alpha|\alpha\in \Phi\rangle$ and $T=\mathrm{Fix}_G(\Sigma).$ Then $(G,(U_\alpha)_{\alpha\in \Phi},T)$ is an RGD system of type $(W,S).$ 

The goal of this section is similar to that for BN-Pairs, which is to show that RGD systems are more or less equivalent to Moufang buildings in the right circumstances. Suppose that $(G,(U_\alpha)_{\alpha\in \Phi},T)$ is an RGD system of type $(W,S)$ where $W$ is spherical. Let $B_+=TU_+$ and $N=\langle T,\{m(u)|u\in U_s^*, s\in S\}\rangle.$ Theorem 7.115 in \cite{buildings} says that $B_+$ and $N$ form a BN-pair, and $(G,B,N,S)$ is a Tits system. Furthermore, $B_+\cap N=T$ and the Weyl group $N/(B_+\cap N)=N/T$ is isomorphic to $W.$ In particular, there is a building $\Delta$ of type $(W,S)$ on which $G$ will act stronly transitively with respect to some apartment system.

Theorem 7.166 in \cite{buildings} then also goes on to say that $\Delta$ is Moufang. Moreover, if the Coxeter diagram of $W$ has no isolated nodes then $\Delta$ is strictly Moufang. We can also choose an apartment $\Sigma_0$ of $\delta$ and identify the roots $\Phi_0$ of $\Sigma_0$ with $\Phi$ by $\alpha_0\mapsto \alpha$ such that for every $\alpha\in \Phi,$ the subgroup $U_\alpha$ is exactly the root group $U_{\alpha_0}.$ This theorem says that RGD systems and strictly Moufang buildings encode essentially the same information.

\subsection{Moufang Polygons}
Recall that the rank of a Building/Coxeter complex/Coxeter system is the size of $S.$ Then a \emph{Moufang Polygon} is a spherical, Moufang building of rank 2. If $\Delta$ is a Moufang polygon then the Weyld group $W$ has presentation $\langle s,t|s^2=t^2=(st)^{m(s,t)}=1\rangle$ and the sperical condition is equivalent to saying that $m(s,t)<\infty.$ As long as $m(s,t)>2$ we also know that the Coxeter diagram of $W$ is connected and so $\Delta$ will also be strictly Moufang. 

The Moufang condition is very restrictive and it turns out that rank 2, thick Moufang buildings only exist when $m(s,t)=3,4,6,8$ as shown in \textcolor{red}{somtheing.} 

Moufang Polygons are of interest because if we assume that $\Delta$ is a 2-spherical Moufang building with a Coxeter diagram that has no labels of 2, then every link of a co-dimension 2 simplex will be a Moufang polygon. The structure of these co-dimension 2 linkes will be very useful for understanding the group theory and geometry of $\Delta.$

\textcolor{red}{probably some more information should go here but I don't know a good reference right now}

\subsection{Non-Spherical RGD Systems and Twin Buildings}
To finish the chapter, we need to add some details to complete the discussion of Moufang buildings. Thus far, the reader may have noticed that we have only discussed the topic of Moufang buildings and RGD systems which are spherical. They may have also noticed that when defining RGD systems, we got a BN-pair by defining $B_+=T\langle U_\alpha|\alpha\in \Phi_+\rangle.$ It seems that we made an arbitrary choice to use $B_+$ instead of the completely symmetric choice of $B_-$ which is defined similarly. In this section, we will address both of these issues at the same time.

Suppose that $(W,S)$ is an arbitrary Coxeter system with roots $\Phi.$ Then we say that $(G,(U_\alpha)_{\alpha\in \Phi},T)$ is an RGD system if it satisfies all of RGD axioms outlined before with the replacement of (RGD2) by the condition that $[U_\alpha,U_\beta]\le U_{(\alpha,\beta)}$ whenever $\alpha$ and $\beta$ form a pre-nilpotent pair. Recall that if $W$ is spherical then $\alpha$ and $\beta$ form a pre-nilpotent pair if and only if $\alpha\neq \pm \beta.$ Thus this new (RGD2) axiom is equivalent to the old. 

Under these assumptions we can form a Tits system $(G,B_+,N,S)$ as before, and we can also form another Tits system $(G,B_-,N,S)$ which gives us two buildings $\Delta_+$ and $\Delta_-.$ We can also define an opposition relation between $\Delta_+$ and $\Delta_-$ so that the fundamental chambers $G/B_+$ and $G/B_-$ are opposite. The triple $(\Delta_+,\Delta_-,\mathrm{op})$ is called a \emph{twin building}. A full treatment of Twin buildings can be found in sections 5.8, 6.3, and 8.3 of \cite{buildings} which we will not repeat here, but we will cover some of the highlights. First of all, twin buildings are a generalization of spherical buildings. If $(\Delta_+,\Delta_-,\mathrm{op})$ is a twin building with spherical Weyl group then there is a canonical isomorphism between $\Delta_+$ and $\Delta_-$ such that the opposition relation coresponds exactly with opposition in the spherical sense.

Let $\Delta$ be the twin builing $(\Delta_+,\Delta_-,\mathrm{op}).$ Twin apartments of $\Delta$ consist of a pair of apartments $(\Sigma_+,\Sigma-)$ so that every chamber in each half is opposite to exactly 1 chamber of the other half. Twin roots of $\Delta$ consist of pairs of roots in twin apartments with appropriate opposition relations. The twin roots of of $\Delta$ are in exact correspondence with the roots of either half of the twin building. When we start with an RGD system, we also know that the group $G$ will act strongly transitively on both halves of the twin building. The groups $U_\alpha$ will act just as they did in the spherical case, except they are now acing on twin roots instead of standard roots.

We can define links in twin buildings as we did before, but for the most part we will only be using links in a single half of the twin building, so the theory is identical. There is of course much more theory about twin buildings than what was covered here, but we will mostly use information only in the spherical case and will cite any other results later as they are used.

In the next chapter we will develop more of the group theoretic consequences of the RGD axioms, and we will so how we can use them to answer questions about finite generation.

\subsection{Kac-Moody Groups}
We will take this section to motivate some of the topics in this chapter with classical examples. Perhaps the most well known example is the group $G=\mathrm{GL}_n(k)$ or $\mathrm{SL}_n(k)$ where $k$ is a field. Then $G$ has an RGD system of type $S_n$ where $S_n$ is the symmetric group on $n$ letters with the standard generating set $(i\;i+1).$ The roots of $W$ correspond to those in a classical root system of type $A_{n-1},$ and so we get a root group $U_\alpha=U_{ij}$ for each pair $1\le i\neq j\le n.$ Each root group $U_{ij}$ consists of matrices with 1's on the diagonal, and all 0's except for possibly in the $ij$ position. The associated building is the same one as described earlier in the chapter, which is Moufang. The BN-pair associated to the RGD system is the same as that described earlier as well.

There are also many more examples. Classical groups such as simplectic, unitary, and orthogonal groups all have RGD systems of spherical type, but we also would like to describe some examples which are not spherical. Let $G=\mathrm{SL}_n(k[t,t^{-1}]).$ A more complete treatment of this group can be found in \cite{sarith} but we will show some of the details here. The Weyl group of $G$ is the affine Coxeter group of type $\tilde{A}_{n-1}.$ The roots of $W$ correspond to $ij$ pairs with $i\neq j$ as well as an exponent $\ell$ of $t.$ Then we have a root group $U_{ij\ell}$ consisting of matrices with 1's on the diagonal, and some multiple of $t^{-\ell}$ in the $ij$ position. The subgroup $B_+$ is the subgroup $\mathrm{SL}_n(k[t])$ and similarly for $B_-.$ 

Perhaps the most motivating example to justify RGD systems is that of Kac-Moody groups as treated in \cite{kacmoody}. Kac-Moody groups are a natural extensions of Chevalley groups and have similar presentations by ``Steinberg relations''. We start with a triple $(\Lambda,(\alpha_i)_{i\in I},(h_i)_{i\in I})$ where $\Lambda$ is a free $\mathbb{Z}$ modules with $\alpha_i\in \Lambda$ and $h_i$ in the dual so that $(\langle \alpha_j,h_i\rangle)$ is a generalized Cartan matrix. For any field $k$ we get an RGD system $(\mathcal{G}_D(k),(\mathcal{U}_\alpha(k))_{\alpha\in \Phi},\mathcal{T}(k))$ where the root system $\Phi$ is that for the Weyl group associated to the generalized Cartan matrix. It is also worth noting that each root group $\mathcal{U}_\alpha(k)$ is isomorphic to the additive group of $k,$ and $\mathcal{T}(k)$ is a torus. These Kac-Moody groups also have some additional properties which we will use later.

\end{document}
