\documentclass[class=book, crop=false,12 pt]{standalone}
\usepackage[subpreambles=true]{standalone}
\usepackage{import}
\usepackage{/home/mark/Documents/gradschool/research/thesis/preamble}

%\newcommand{\leqnomode}{\tagsleft@true}

\begin{document}
\chapter{Known Results on Finite Generation}
\label{ch:known}

Throughout this chapter $(G,(U_\alpha)_{\alpha\in \Phi},T)$ will be an RGD system of type $(W,S)$ with the following assumptions:
\smallskip
\begin{equation}
	\label{assume}
	\tag{A} 
\begin{aligned}
	&W \text{ has rank 3, }S=\{s,t,u\},\: a=m(s,t),b=m(s,u),c=m(t,u)\text{ and }3\le a\le b\le c\\
	&[U_\alpha,U_\beta]=1\text{ when }\alpha,\beta \text{ are nested}
\end{aligned}
\end{equation}
\smallskip

Let $\Sigma$ be the Coxeter complex of $W$ with fundamental chamber $C,$ and $\Phi_+$ be the positive roots of $\Sigma.$ We will also let $U_+=\langle U_\alpha|\alpha\in \Phi_+\rangle$ be the subgroup of $G$ generated by the positive root groups. We will also note that properties of RGD systems tell us that $a,b,c\in \{2,3,4,6,8\}$ and thus by \eqref{assume} we know that $a,b,c\in \{3,4,6,8\}.$

%Throughout this section, $\G$ will be a Kac-Moody group with rank 3 Weyl group $W$ over a field $k.$ We will also assume that $W$ is defined by the Coxeter diagram with edge labels $a,b,c\in\{3,4,6\}$ with $a\le b\le c$ and $c\ge 4.$ This last condition ensures that $W$ is hyperbolic. Let $\Sigma$ be the Coxeter complex of $W.$ Let $\Phi^+$ be the positive roots of $\Sigma,$ and for any $\alpha\in \Phi^+$ we will let $U_\alpha$ be the root group associated to $\alpha.$

For any vertex $v$ of $\Sigma,$ there will be some walls of $\Sigma$ which pass through $v,$ and for each of these walls we have a unique \emph{positive} root. We will call these the \textbf{positive roots at $v$} and denote them by $\Phi_+^v.$ Recall that $\st(v)$ is defined as all the chambers containing $v$ as a vertex. If there are $n$ positive roots at $v$ then $|\st(v)|=2n.$ Furthermore, it is possible to label the positive roots at $v$ as $\alpha_1,\dots,\alpha_n$ in such a way that $\alpha_i\cap \alpha_j\subset \alpha_k$ for any $1\le i\le k\le j\le n.$ This ordering is unique up to a reversal of the form $\alpha_i\mapsto \alpha_{n+1-i}.$ This possible reversal will not matter in most cases and if it does then a choice of $\alpha_1$ will be specified. It does however allow us to unambigiously define $\alpha_1$ and $\alpha_n$ as the \textbf{simple} roots at $v.$ They are the unique positive roots at $v$ whose intersection is contained in all other positive roots at $v.$

Now we can define $U_v$ to be the subgroup of $G$ generated by all of the root groups of the positive roots at $v.$ That is
\[
U_v=\langle U_\alpha|\alpha \text{ is a positive root at }v\rangle=\langle U_\alpha|\alpha\in \Phi_+^v\rangle
\]
If $\alpha_1,\alpha_2,\dots,\alpha_n$ is a standard ordering of the positive roots at $v$ then we can simplify notation by letting $U_i=U_{\alpha_i}$ for all $\alpha_i$ through $v.$ Since $v$ is a simples of $\Sigma$ of co-dimension 2, we know from the theory of RGD systems that $U_v$ will also have the structure of a spherical, rank 2 RGD system as well. Let $U'_v=\langle U_1,U_n\rangle$ be the subgroup of $U_v$ generated by the simple root groups, where $|\st(v)|=2n.$ Then it is known that $U_v=U'_v=\langle U_1,U_n\rangle$ with the exception of a few cases which we will explictly state in the following Lemma.

\begin{lemma}
	\label{lem:index}
	Let $v$ be a vertex of $\Sigma$ with $|\st(v)|=2n$ and let $U'_v=\langle U_1,U_n\rangle$ where $U_1,U_n$ are the root groups of the simple roots at $v.$ Then the group $U_v$ is has the structure of a spherical, rank 2 RGD system and $U_v=U'_v$ unless $U_v$ is isomorphic to one of the following groups:
	\[
		C_2(2)\qquad G_2(2) \qquad G_2(3) \qquad {}^2F_4(2)
	\]
	In fact, we also know the index $[U_v:U'_v]$ in each of these cases which is summarized in the following table.
	\[
		\begin{array}{c|c}
			U_v&[U_v:U'_v]\\\hline
			C_2(2)&2\\
			G_2(2)&4\\
			G_2(3)&3\\
			{}^2F_4(2)&2
		\end{array}
	\]
	and $[U_v:U'_v]=1$ in all other cases.
\end{lemma}

We can see from the previous lemma that even when $U'_v\neq U_v,$ it is still a fairly large subgroup and in some cases it will even be normal. This will allow us to construct helpful homomorphisms later, but before we do so we will explicitly state the desired result.

\begin{lemma}
	\label{lem:normal}
	Suppose $v$ is a vertex of $\Sigma$ with $|\st(v)|=2n$ such that $[U_v:U'_v]\ge 2.$ If $U_v$ is isomorphic to $C_2(2), G_2(3),$ or ${}^2F_4(2)$ then $U'_v$ is a normal subgroup of $U_v.$ If $U_v\cong G_2(2)$ then $U'_v$ is not a normal subgroup of $U_v,$ but there is a standard labeling of the positive roots through $v$ so that $U''_v=\langle U_{1},U_{5},U_{6}\rangle$ is a normal subgroup of $U_v$ with $[U_v:U''_v]=2.$
\end{lemma}
\begin{proof}
	If $U_v\cong C_2(2)$ or ${}^2F_4(2)$ then $U'_v$ is a subgroup of index 2 and thus it is normal. If $U_v\cong G_2(3)$ then $U_v$ is a 3-group and thus $3$ is the smallest prime dividing $|U_v|$ and we know that $U'_v$ is normal in this case as well.

	Now suppose $U_v\cong G_2(2).$ \textcolor{red}{Need to add this proof later}
\end{proof}

Using Lemma \ref{lem:normal} and elementary group theory, we get the following result.
\begin{cor}
	\label{cor:phiv}
	Suppose $v$ is a vertex of $\Sigma$ with $|\st(v)|=2n$ such that $[U_v:U'_v]\ge 2.$ Then there is a cyclic group $H$ and a surjective group homomorphism $\phi_v:U_v\to H$ with the property that $\phi_v(U_1)=\phi_v(U_n)=\{1\}$ where $U_1$ and $U_n$ are the simple root groups at $v.$
\end{cor}
\begin{proof}
	If $[U_v:U'_v]\ge 2$ then $U_v$ must be isomorphic to one of $C_2(2),G_2(2),G_2(3),{}^2F_4(2).$ If $U_v\cong C_2(2),G_2(3),{}^2F_4(2)$ then we can apply Lemma \ref{lem:normal} to let $H=U_v/U'_v$ and $\phi_v$ be the quotient map which certainly will be surjective and send $U_1$ and $U_n$ to $\{1\}$ by the definition of $U'_v.$ The group $H$ is cyclic because it has prime order.

	If $U_v\cong G_2(2)$ then we know that $U'_v\subset U''_v=\langle U_1,U_5,U_6\rangle$ for an appropreate standard labeling, and we again apply Lemma \ref{lem:normal} to set $H=U_v/U''_v$ and $\phi_v$ as the quotient map. The group $H$ is again cyclic because it has prime order.
\end{proof}

The following corollary will show that we do not have very much wiggle room when defining $\phi_v,$ and thus if we can write any function which ``looks like'' $\phi_v$ then they must be esentially the same.
\begin{cor}
	\label{cor:uniquephiv}
	Suppose $v$ is a vertex of $\Sigma$ with $|\st(v)|=2n$ such that $[U_v:U'_v]\ge 2$ and let $\phi_v$ be defined as in the previous corollary. Then $\ker \phi_v$ is the unique, proper, normal subgroup of $U_v$ which contains $U_1$ and $U_n.$
\end{cor}
\begin{proof}
	If $U_v\cong C_2(2),G_2(3),{}^2F_4(2)$ then $U'_v$ is normal, it is generated by $U_1$ and $U_n,$, and it has prime index so there cannot be another proper subgroup containing $U'_v.$ By the construction of $\phi_v,$ we also know that $\ker \phi_v=U'_v$ so that $\ker\phi_v$ is the unique proper, normal subgroup of $U_v$ containing $U_1$ and $U_n.$

	If $U_v\cong G_2(2)$ then $\ker \phi_v=U''_v=\langle U_1,U_5,U_6\rangle$ under a standard labeling. If $N$ is any normal subgroup containing $U_1$ and $U_n$ then we can apply the commutator relations in $G_2(2)$ to get \textcolor{red}{add proof later}
\end{proof}

So far we have only considered each vertex $v$ and $U_v$ separately. But in the Coxeter complex $\Sigma,$ we have not only a collection of vertices, but an action of the group $W$ on the vertices which behaves nicely with properties like the type of a vertex. We will show that the $W$ action also interacts nicely with $U_v$ and $\phi_v$ in a similar way.

\begin{lemma}
	\label{lem:resp_order}
	Suppose $v$ is a vertex of $\Sigma$ of type $s,$ $|\st(v)|=2n,$ and $[U_v:U'_v]\ge 2.$ Suppose that $v'$ is any other vertex of $\Sigma$ of type $s.$ Then there is an element of $w\in W$ such that $v'=wv$ and there is an isomorphism between $U_v$ and $U_{v'}=U_{wv}.$ Furthermore, if $\alpha_1,\dots,\alpha_n$ is a standard ordering of the positive roots through $v,$ then we can choose $w$ so that $w\gamma$ is a positive root at $wv$ for every positive root $\gamma$ at $v$ and $\alpha'_1=w\alpha_1,\dots,\alpha'_n=w\alpha_n$ is a standard ordering of the positive roots at $wv.$
\end{lemma}
\begin{proof}
	Since the $W$ action on $\Sigma$ is transitive on vertices of the same type, it will suffice to show the result when $v$ is a vertex of the fundamental chamber $C.$ Let $D=\proj_{v'}(C)$ so that $d(D,C)$ is minimal among all chambers of $\st(v').$ Then we know that no walls through $v'$ can separate $D$ and $C,$ because crossing one of these walls would produce a chamber in $\st(v)$ which is closer to $C.$ Therefore, a root at $v'$ is positive if and only if it contains $D.$

Now choose $w\in W$ such that $D=wC.$ We claim that $w$ satisfies the desired properties. First of all, $v$ is a vertex of $C$ of type $s$ and thus $wv$ is a vertex of $wC=D$ of type $s.$ But we know that $v'$ is a vertex of $D$ of type $s$ by definition and thus $wv=v'$ as desired. Now suppose $\gamma$ is any positive root at $x.$ Then $C\in \gamma$ and thus $D=wC\in w\gamma$ and thus $w\gamma$ is positive at $wv=v'.$ Therefore, we know that $w$ induces a bijection between the positive roots at $v$ and the positive roots at $v'.$ Suppose $\alpha_1,\dots,\alpha_n$ is a standard ordering of the positive roots at $v.$ Then by definition we have $\alpha_i\cap\alpha_j\subset\alpha_k$ for all $1\le i\le k\le j\le n$ where $2n=|\st(v)|.$ If we apply the action of $w$ we get $w\alpha_i\cap w\alpha_j\subset w\alpha_k$ for all $1\le i\le k\le j\le n$ as well. But since the action of $w$ is a bijection on the positive roots, we know that each $w\alpha_i$ is also positive at $wv$ and thus $\alpha'_1=w\alpha_1,\dots,\alpha'_n=w\alpha_n$ is a standard ordering of the roots through $wv=v'$ as desired.

The last thing we must do is show there is a bijection between $U_v$ and $U_{v'}.$ The theory of RGD systems tells us that there is a subgroup $N\le G$ with the property that for any $w\in W,$ there is some $\tilde{w}\in N$ such that $\tilde{w}U_\alpha \tilde{w}^{-1}=U_{w\alpha}$ for all $\alpha\in \Phi.$ Choose such an $\tilde{w}$ for the $w$ defined above and let $f_w:G\to G$ be the isomorphism of conjugation by $\tilde{w}.$ Now suppose that $\alpha$ is any positive root through $v.$ Then $f_w(U_\alpha)=U_{w\alpha}$ and $w\alpha$ is a positive root through $wv=v'.$ Thus the map $f_w$ restricts to a group homomorphism $\bar{f}_w:U_v\to U_{v'}$ which is necessarily injective. 

Now suppose that $\alpha'$ is any positive root at $v'.$ The action of $w$ induces a bijection of the positive roots that $v$ to the positive roots at $v'$ so there is some positive root $\alpha$ at $v$ such that $w\alpha=\alpha'.$ But this means $\bar{f}_w(U_\alpha)=U_{\alpha'}.$ Since $U_{v'}$ is generated by the positive root groups at $v',$ this means $\bar{f}_w$ is surjective and we get an isomorphism betwen $U_v$ and $U_{v'}$ as desried.
\end{proof}

Before moving on it is worth clarifying that the type $s$ of the vertex $v$ in the previous lemma can by any type, not just the literal type $s$ in the definition of $W.$

The previous result can also be used to show that the $W$ action on $\Sigma$ also behaves nicely with respect to the homomorphisms $\phi_v$ when they exit.
\begin{cor}
	\label{cor:respect_phiv}
	Suppose $v$ is a vertex of $\Sigma$ with $|\st(v)|=2n$ and $[U_v:U'_v]\ge 2.$ If $v'$ is any other vertex of $\Sigma$ of the same type then there is an isomorphism between $U_v$ and $U_{v'}$ which sends $U'_v$ to $U'_{v'}$ and $\ker \phi_v$ to $\ker \phi_{v'}.$
\end{cor}
\begin{proof}
	Let $\bar{f}_w$ be the isomorphism defined by Lemma \ref{lem:resp_order} and let $\alpha_1\dots,\alpha_n$ be a standard ordering of the positive roots through $v.$ Then by Lemma \ref{lem:resp_order} again we know that there is a standard ordering $\alpha'_1,\dots,\alpha'_n$ of the positive roots through $v'$ such that $\bar{f}_w(U_{\alpha_i})=U_{\alpha'_i}$ for all $1\le i\le n.$ Since $U'_v$ and $U'_{v'}$ are generated by $\{U_{\alpha_1},U_{\alpha_n}\}$ and $\{U_{\alpha'_1},U_{\alpha'_n}\}$ respectively, we know that $\bar{f}_w$ must induce an isomorphism between $U'_v$ and $U'_{v'}$ as desired.

	By Corollary \ref{cor:uniquephiv}, $\ker \phi_v$ is the unique proper, normal subgroup of $U_v$ containing $U'_v.$ If we apply the isomorphism $\bar{f}_w$ once again we get that $\bar{f}_w(\ker \phi_v)$ is a proper, normal subgroup of $U_{v'}$ containing $\bar{f}_w(U'_v)=U'_{v'}$ and thus $\bar{f}_w(\ker \phi_v)=\ker \phi_{v'}$ by Corollary \ref{cor:uniquephiv}.
\end{proof}



\vspace{1 in}
The general theory gives us the following result
\begin{theorem}
	\label{knownfgresult}
	Let $\G$ be a Kac-Moody group over $k$ with rank 3 Weyl group $W$ as before. For any vertex $v$ of $\Sigma,$ let $U'_v=\langle U_1,U_n\rangle$ where $U_1,U_n$ are the simple roots at $v.$ If $U'_v=U_v$ for all $v\in \Sigma$ then $U$ is finitely generated.
\end{theorem}

Remark: In fact, we can make an even stronger statement. Let $\alpha_s$ be the positive root defined by the wall which separates $C$ and $sC$ and similarly define  $\alpha_t$ and $\alpha_u.$ If $U'_v=U_v$ for all $v\in \Sigma$ then $U$ is generated by $ U_{\alpha_s}, U_{\alpha_t},$ and  $U_{\alpha_u}.$

%I use this lemma later. This isn't organized yet but I wanted to have it so my reference aren't broken.
%\begin{lemma} Let $\alpha,\beta,\beta+\alpha,\beta+2\alpha$ be the positive roots of a root system of type $C_2$ and $U$ the unipotent subgroup of $C_2(\F{2}).$ Then $U=\langle U_\alpha,U_\beta,U_{\beta+\alpha}\rangle=\langle U_\alpha,U_\beta,U_{\beta+2\alpha}\rangle.$
%	\label{c2f2gen}
%\end{lemma}
%
%\begin{lemma}
%	\label{g2f3gen}
%Let $\alpha,\beta,\beta+\alpha,\beta+2\alpha,\beta+3\alpha,2\beta+3\alpha$ be the positive roots of a root system of type $G_2$ and $U$ the unipotent subgroup of $G_2(\F{3}).$ Then $U=\langle U_\alpha,U_\beta,U_{\beta+\alpha}\rangle=\langle U_\alpha,U_\beta,U_{\beta+3\alpha}\rangle.$
%\end{lemma}
%
%\begin{lemma}
%	\label{g2f2gen}
%	Let $\alpha,\beta,\beta+\alpha,\beta+2\alpha,\beta+3\alpha,2\beta+3\alpha$ be the positive roots of a root system of type $G_2$ and $U$ the unipotent subgroup of $G_2(\F{2}).$ Then $U=\langle U_\alpha,U_\beta,U_{\beta+\alpha}\rangle$ but $\langle U_\alpha,U_\beta,U_{\beta+3\alpha}\rangle\subsetneq U.$
%\end{lemma}
%
%\begin{lemma}
%	\label{g2f2unique}
%	If $\G=\G_2(\F{2})$ then there is a unique surjective homomorphism $\phi:U\to K$ where $K$ is the cyclic group of order 2, such that $\phi(U_\alpha)=\phi(U_\beta)=1.$
%\end{lemma}

\end{document}
