\documentclass[class=book, crop=false,12 pt]{standalone}
\usepackage[subpreambles=true]{standalone}
\usepackage{import}
\RequirePackage{/home/mark/Documents/gradschool/research/thesis/preamble}

%\newcommand{\leqnomode}{\tagsleft@true}

\begin{document}
\chapter{Known Results on Finite Generation}
\label{ch:known}

Throughout this chapter $(G,(U_\alpha)_{\alpha\in \Phi},T)$ will be an RGD system of type $(W,S).$ As discussed in the previous chapter, this means that there is a twin building $\Delta=(\Delta_+,\Delta_-,\mathrm{op})$ on which $G$ will act strongly transitively. Our main goal in the rest of the paper is to prove some results about finiteness properties of $G$ and its subgroups. Our main tools will be the geometry of the fundamental apartment $\Sigma,$ and links of co-dimension 2 simplices in $\Delta.$ In the next section we will develop the theory about roots which we will use in the main results. Since the roots of a twin apartment are in exact correspondence with the roots of a single half, we will consider $\Sigma$ as a standard apartment or Coxeter complex, and not a twin apartment.

\section{Local Roots and Root Groups}
Assume that they Weyl group $W$ of $G$ has rank 3 and that $W$ is 2-spherical. Then the fundamental apartment $\Sigma$ will be a Coxeter complex which is 2 dimensional, and thus co-dimension 2 simplices of $\Sigma$ will be points, or vertices. We will also assume that $m(s,t)\ge 3$ for all $s,t\in S$ so that $\Delta$ is strictly Moufang, and every link of a vertex will also by strictly Moufang.

For any vertex $v$ of $\Sigma,$ there will be some walls of $\Sigma$ which pass through $v,$ and for each of these walls we have a unique \emph{positive} root. We will call these the \textbf{positive roots at $v$} and denote them by $\Phi_+^v.$ Recall that $\st(v)$ is the set of all simplices with $v$ as a face, but we will view it as a chamber complex and only consider the chambers with $v$ as a face. If there are $n$ positive roots at $v$ then $|\st(v)|=2n.$ 

Furthermore, it is possible to label the positive roots at $v$ as $\alpha_1,\dots,\alpha_n$ in such a way that $\alpha_i\cap \alpha_j\subset \alpha_k$ for any $1\le i\le k\le j\le n.$ We will call this a \emph{standard labeling} or \emph{standard ordering} of the positive roots at $v.$ This ordering is unique up to a reversal of the form $\alpha_i\mapsto \alpha_{n+1-i}.$ In most cases this reversal will not matter, and when it does we will specify a choice of $\alpha_1.$ While this definition may seem strange, it is worth noting that a standard ordering will give an ordering which increases as we move clockwise or counterclockwise, depending on our choice of $\alpha_1.$ The standard labeling also has a nice interpretation for open intervals. If $\alpha_1,\dots,\alpha_n$ is a standard labeling of the roots through $v,$ and $i<j$ then $(\alpha_i,\alpha_j)=\{\alpha_k|i<k<j\}.$

If $v$ is a vertex of $\Delta$ then the Moufang property, and the assumption $m(s,t)\ge 3$ will imply that $\lk(v)$ is also a Moufang polygon $\Delta'.$ We also know that the root groups $U_\alpha$ will be isomorphic to the root groups of $\Delta'.$ We define the subgroup $U_v=\langle U_\alpha|\alpha\in \Phi_+^v\rangle,$ and note that it is the corresponding $U'_+$ for the building $\Delta'.$ We can also define a subgroup $U'_v=\langle U_1,U_n\rangle\le U_v$ which is the subgroup of $U_v$ generated by the root groups of simple roots at $v.$ It turns out that ``most'' of the time, $U'_v=U_v,$ which has deep consequences for $U_+.$

Recall that a spherical building has a notion of opposition when two chambers are as far apart as possible. We will say that a spherical building $\Delta$ satisfies condition (co) if for any chamber $C,$ the set $C^\text{op}=\{D\in \Delta|C\mathrm{ op }D\}$ is gallery connected. Lemma 3 in \cite{cop} tells us that for any vertex $v$ of $\Sigma,$ the index $[U_v:U'_v]$ is equal to the number of connected components, as chamber complexes, of the spherical building $\lk(v).$ In particular, $U_v=U'_v$ if and only if $\lk(v)$ satisfies condition (co). Citing the main result of \cite{cop} again we have the following Lemma

\begin{lemma}
	\label{lem:index}
	If $v$ is a vertex of $\Sigma,$ then $\lk(v)$ satisfies condition (co) unless $\lk(v)$ is the spherical building associated to one of the following finite Chevalley groups
	\[
		C_2(2)\qquad G_2(2) \qquad G_2(3) \qquad {}^2F_4(2)
	\]
	Moreover, the index $[U_v:U'_v]$ is summarized for all of the exceptional cases in the following table.
	\[
		\begin{array}{c|c}
			U_v&[U_v:U'_v]\\\hline
			C_2(2)&2\\
			G_2(2)&4\\
			G_2(3)&3\\
			{}^2F_4(2)&2
		\end{array}
	\]
\end{lemma}

As mentioned in the previous chapter, twin buildings are a generalization of spherical buildings and we define condition (co) in the same manner for twin buildings, where opposition is now the twin building opposition in the two halves of the twin building. Theorem 1.5 in \cite{globalco} says that a twin building will satisfy property (co) if all of it's rank 2 residues satisfy (co) when viewed as spherical buildings. As a result, it is enough to check that none of the links of co-dimension 2 vertices of $\Delta$ are one of the 4 types described above.

For any twin building $\Delta$ of type $(W,S),$ and choice of fundamental apartment and fundamental chamber $\Sigma$ and $C$ we have a canonical set of fundamental roots $\{\alpha_s\}_{s\in S}$ where $\alpha_s$ is the root which contains $C$ and not $sC.$ Then we have the subgroup $U'=\langle U_{\alpha_s}|s\in S\}$ and Lemma 3 of \cite{cop} says that $U'=U_+$ if and only if $\Delta$ satisfies (co). Note that if $\Delta$ has rank 2 then $U'$ is identical to that described above and we get the same result as before. This also gives the following Corollary

\begin{cor}
	\label{cor:cofg}
	Let $(G,(U_\alpha),T)$ is an RGD system of type $(W,S).$ If $W$ is 2-spherical, and the associated building $\Delta$ does not have any rank 2 residues associated to $C_2(2),G_2(2),G_2(3),$ or ${}^2F_4(2)$ then $U_+$ is finitely generated.
\end{cor}

Much of the theory of twin buildings relies on condition (co), and thus uses the assumption that no rank 2 residues are among the 4 exceptional types listed above. For example, if $\Delta$ satisfies (co) then $U_+$ is finitely generated. Our goal for the remainder will be to fill in some of this theory to include cases where $\Delta$ does not satisfy (co). Before we can do this, we will need to collect some more results about the 4 exceptional rank 2 buildings listed above.

The groups $C_2(2),G_2(2),$ and $G_2(3)$ are all finite Chevalley groups and so they have well known presentations. The group ${}^2F_4(2)$ is a twisted Chevalley group, but we will not have as much cause to work with this group specifically so we will not work with it as much. A full construction of Chevalley groups can be found in \cite{carter}, among other places, but we will record the specific presentations found in Corollary 5.2.3.

\begin{lemma}
	\label{lem:c22pres}
	If $v$ is a vertex of $\Sigma$ such that $\lk(v)$ is the building associated to $C_2(2),$ then there is a standard labeling $\alpha_1,\dots,\alpha_4$ of the positive roots at $v$ so that $U_v$ has the following presentation:
	\begin{align*}
		U_{\alpha_i}&=U_i=\{1,u_i\}\text{ for all }i\\
		U_+&=\langle u_i|1\le i\le 4,\;u_i^2=1,\;[u_1,u_4]=u_2u_3,\;[u_i,u_j]=1\text{ if }|i-j|<3\rangle
	\end{align*}
\end{lemma}

To get the presentations in the $G_2(2)$ and $G_2(3)$ cases, we can derive both presentations at the same time from the results in \cite{carter}. If $\lk(v)$ is the building associated to $G_2(k),$ then for each positive root $\alpha$ through $v,$ the group $U_\alpha$ is isomorphic to the additive group of the finite field of order $k.$ This means we can write $U_\alpha=\{x_\alpha(t)|t\in \mathbb{F}_k\}$ and $x_\alpha(t)x_\alpha(u)=x_\alpha(t+u).$ As a consequence, $x_\alpha(0)=1$ for all roots $\alpha.$ Then there is a standard labeling $\alpha_1,\dots,\alpha_6$ of the positive roots through $v,$ with $U_i=U_{\alpha_i}$ such that $U_v$ is generated by $U_i=\{x_i(t)|t\in\mathbb{F}_k\}$ subject to the relations
\begin{align*}
	[x_1(u),x_6(t)]&=x_5(\pm tu)x_3(\pm tu^2)x_2(\pm tu^3)x_4(\pm 2t^2u^3)\\
	[x_1(u),x_5(t)]&=x_3(\pm 2tu)x_2(\pm 3 tu^2)x_4(\pm 3t^2u)\\
	[x_2(u),x_6(t)]&=x_4(\pm tu)\\
	[x_1(u),x_3(t)]&=x_2(\pm 3tu)\\
	[x_3(u),x_5(t)]&=x_4(\pm 3tu)\\
	[x_i(u),x_j(t)]&=1 \quad \text{otherwise}
\end{align*}
There is some ambiguity in this presentation from the signs in the relations, but this will not be a problem as when $k=2$ the signs are irrelevant and when $k=3$ then sign change replaces a generator of $U_i$ by its inverse. This presentation applies to any Chevalley group of type $G_2,$ so next we will apply it to the two specific groups in question.

\begin{lemma}
	\label{lem:g22pres}
	Suppose $v$ is a vertex of $\Sigma$ so that $\lk(v)$ is the building associated to $G_2(2).$ Then there is a standard labeling of the positive roots through $v$ with $U_i=U_{\alpha_i}$ such that $U_v=\langle U_i|1\le i\le 6\rangle$ and a presentation is given by the following relations
	\begin{align*}
		U_i&=\{1,u_i\}\\
		[u_1,u_6]&=u_5u_3u_2\\
		[u_1,u_5]&=u_2u_4\\
		[u_2,u_6]&=u_4\\
		[u_1,u_3]&=u_2\\
		[u_3,u_5]&=u_4\\
		[u_i,u_j]&=1\quad \text{otherwise}
	\end{align*}
\end{lemma}

\begin{lemma}
	\label{lem:g23pres}
	Suppose $v$ is a vertex of $\Sigma$ so that $\lk(v)$ is the building associated to $G_2(3).$ Then there is a standard labeling of the positive roots through $v$ with $U_i=U_{\alpha_i}$ such that $U_v=\langle U_i|1\le i\le 6\rangle$ and a presentation is given by the following relations
	\begin{align*}
		U_i&=\{1,x_i(1),x_i(-1)\}\\
		[x_1(u),x_6(t)]&=x_5(c_1 tu)x_3(c_2 tu^2)x_2(c_3 tu)x_4(c_4 t^2u)\\
		[x_1(u),x_5(t)]&=x_3(c_5 tu)\\
		[x_2(u),x_6(t)]&=x_4(c_6 tu)\\
		[x_i(u),x_j(t)]&=1 \quad \text{otherwise}
	\end{align*}
where each $c_i\in \{\pm 1\}.$
\end{lemma}

So far, we know that for the 4 exceptional cases listed above we have $U'_v\neq U_v$ and we know the index. The next few results will be to collect properties about $U'_v$ and $U_v$ which we will use later when proving results about finite generation.

\begin{lemma}
	\label{lem:normal}
	Suppose $v$ is a vertex of $\Sigma$ with $|\st(v)|=2n$ such that $[U_v:U'_v]\ge 2.$ If $\lk(v)$ is the building associated to $C_2(2),G_2(3),$ or ${}^2F_4(2)$ then $U'_v$ is a normal subgroup of $U_v.$ If $\lk(v)$ is the building associated to $G_2(2)$ then $U'_v$ is not normal, but there is a standard labeling of the positive roots through $v$ such that $U''_v=\langle U_1,U_2,U_6\rangle$ is proper and normal.
\end{lemma}
\begin{proof}
	First suppose $\lk(v)$ is the building associated to either $C_2(2)$ or ${}^2F_4(2).$ Then $U'_v$ is an index 2 subgroup of $U_v$ and thus it is normal.

	Now suppose $\lk(v)$ is the building associated to $G_2(3).$ To show that $U'_v$ is normal it will suffice to show that $x_i(t)U_jx_i(-t)\in U'_v$ for $1\le i\le 6$ and $j\in\{1,6\}.$ Since most of the commutators are trivial, we can use the presentation in Lemma \ref{lem:g23pres} to calculate
	\begin{align*}
		[[x_1(u),x_6(t)],x_6(t')]&=[x_5(c_1 tu)x_3(c_2 tu^2)x_2(c_3 tu)x_4(c_4 t^2u),x_6(t')]\\
					 &=[x_2(c_3tu),x_6(t')]\\
					 &=x_4(c_3c_6tt'u)
	\end{align*}
	which means $U_4\subset U'_v.$ A similar computation shows that $[x_1(t'),[x_1(u),x_6(t)]]$ is a non-trivial element of $U_3$ so that $U_3\subset U'_v$ as well. 

	The presentation in Lemma \ref{lem:g23pres} tells us that $U_i$ and $U_j$ commute when $|i-j|\le 3.$ Therefore we have $x_i(t)U_1x_i(-t)=U_1\subset U'_v$ for all $i\ge 3,$ and similarly we have $x_i(t)U_6x_i(-t)=U_6$ for all $i\le 4.$ It is also clear that $x_1(t)U_6x_1(-t)\subset U'_v$ and $x_6(t)U_1x_6(-t)\subset U'_v.$ There are only two conjugates left to check and we can see that
	\[
		x_5(u)x_1(t)x_5(-u)=[x_5(u),x_1(t)]x_1(t)x_5(u)x_5(-u)\subset U_3U_1\subset U'_v
	\]
	and similarly we have
	\[
		x_2(u)x_6(t)x_2(-u)=[x_2(u),x_6(t)]x_6(t)x_2(u)x_2(-u)\subset U_4U_6\subset U'_v
	\]
	which shows $U'_v$ is normal as desired.


	Now suppose that $\lk(v)$ is the building associated to $G_2(2)$ and choose the standard labeling of the positive roots through $v$ so that we have the presentation given in Lemma \ref{lem:g22pres}. Let $f$ be a map from $\{u_i\}_{1\le i\le 6}$ to the group with two elements $\{\pm 1\}$ such that $f(u_i)=1$ if $i\in \{1,2,4,6\}$ and $f(u_i)=-1$ if $i\in \{3,5\}.$ If we check the presentation given in Lemma \ref{lem:g22pres} we can see that $f$ will extend to a well defined group homomorphism $f:U_v\to \{\pm 1\}$ which is surjective. Thus $\ker f$ has index 2 in $U_v$ and it contains $U''_v=\langle U_1,U_2,U_6\rangle$ by definition. 

	The group $U_v$ is generated by the groups $U_i$ and thus is generated by the elements $u_i$ for $1\le i \le 6.$ Since $u_2,u_6\in U''_v$ we also know that $[u_2,u_6]=u_4\in U''_v$ as well. By the presentation in Lemma \ref{lem:g22pres} this means that $[u_i,u_3],[u_j,u_5]\in U''_v$ for all $i,j.$ Using the fact that $xy=yx[y,x]^{-1}$ we can commute $u_3$ and $u_5$ past any element of $U''_v,$ picking up only other elements of $U''_v.$ Since $u_3$ and $u_5$ commute we can say that $U_v=U_5U_3U''_v.$ The cosets of $U''_v$ are $U''_v,u_5U''_v,u_3U''_v,u_5u_3U''_v,$ but since $u_5u_3=[u_1,u_6]u_2\in U''_v$ we get $U''_v=u_5u_3U''_v$ and $u_5U''_v=u_3U''_v$ so that $[U_v:U''_v]\le 2.$ Since $U''_v\subset \ker f$ and $[U_v:\ker f]=2,$ we must have $U''_v=\ker f$ so that $U''_v$ is proper an normal. Now it remains to show that $U'_v$ is not normal.

	If $U'_v$ was normal then there would be a surjective map $g:U_v\to U_v/U'_v=H$ where $|H|=4$ and therefore is abelian. This means that $g(u_2)=g([u_1,u_3])=[g(u_1),g(u_3)]=1$ and thus $u_2\in \ker g.$ But $u_1,u_6\in \ker g$ by definition and thus $U''_v\subset \ker f.$ This is a contradiction as $\ker f=U'_v$ by definition, and $U''_v$ strictly contains $U'_v,$ and thus $U'_v$ is not normal.

\end{proof}

Using Lemma \ref{lem:normal} and elementary group theory, we get the following result.
\begin{cor}
	\label{cor:phiv}
	Suppose $v$ is a vertex of $\Sigma$ with $|\st(v)|=2n$ such that $[U_v:U'_v]\ge 2.$ Then there is a non-trivial cyclic group $H$ and a surjective group homomorphism $\phi_v:U_v\to H$ with the property that $\phi_v(U_1)=\phi_v(U_n)=\{1\}$ where $U_1$ and $U_n$ are the simple root groups at $v.$
\end{cor}
\begin{proof}
	If $[U_v:U'_v]\ge 2$ then $\lk(v)$ must be isomorphic to the building associated to one of $C_2(2),G_2(2),G_2(3),{}^2F_4(2).$ If the associated group is one of $C_2(2),G_2(3),{}^2F_4(2)$ then we can apply Lemma \ref{lem:normal} to let $H=U_v/U'_v$ and $\phi_v$ be the quotient map which certainly will be surjective and send $U_1$ and $U_n$ to $\{1\}$ by the definition of $U'_v.$ The group $H$ is cyclic because it has prime order.

	If $\lk(v)$ is isomorphic to the building associated to $G_2(2)$ then we know that $U'_v\subset U''_v=\langle U_1,U_2,U_6\rangle$ for an appropriate standard labeling, and we again apply Lemma \ref{lem:normal} to set $H=U_v/U''_v$ and $\phi_v$ as the quotient map. The group $H$ has order equal to $[U_v:U''_v]$ which must be $2$ as $U''_v\neq U_v$ and $U''_v\neq U'_v,$ and thus $H$ is cyclic as desired.
\end{proof}

The following corollary will show that we do not have very much wiggle room when defining $\phi_v,$ and $\ker \phi_v$ is uniquely determined by the fact that $\phi_v$ sends $U_1$ and $U_n$ to the identity.
\begin{cor}
	\label{cor:uniquephiv}
	Suppose $v$ is a vertex of $\Sigma$ with $|\st(v)|=2n$ such that $[U_v:U'_v]\ge 2$ and let $\phi_v$ be defined as in the previous corollary. Then $\ker \phi_v$ is the unique, proper, normal subgroup of $U_v$ which contains $U_1$ and $U_n.$
\end{cor}
\begin{proof}
	First suppose that $\lk(v)$ is the building associated to $C_2(2),G_2(4),$ or ${}^2F_4(2).$ Then by the construction in Corollary \ref{cor:phiv} we know that $\ker \phi_v=U'_v$ and $U'_v$ contains $U_1,U_n$ by definition. In all of these cases the index $[U_v:U'_v]$ is prime and thus $U'_v$ is the unique proper, normal subgroup of $U_v$ which contains $U_1,U_n.$

	Now suppose that $\lk(v)$ is the building associated to $G_2(2).$ The proof of Corollary \ref{cor:phiv} shows that any normal subgroup of $U_v$ which contains $U_1$ and $U_6$ must also contain $U_2$ and thus $U''_v.$ Since $\ker \phi_v=U''_v$ and $[U_v:U''_v]=2,$ $\ker\phi_v$ is once again the unique proper normal subgroup of $U_v$ containing $U_1$ and $U_6.$
\end{proof}

Despite the fact that $U_v$ is not generated by $U_1$ and $U_n,$ it will be helpful to show which root groups will generate $U_v.$ This will be necessary later when we prove that $U_+$ is finitely generated in certain cases.

\begin{lemma}
	\label{lem:generators}
	Suppose $v$ is a vertex of $\Sigma$ such that $\lk(v)$ is the Moufang polygon associated to $C_2(2)$ or $G_2(3).$ If $\alpha_1,\dots,\alpha_n$ is a standard ordering of the positive roots through $v$ which gives the presentation as in Lemma \ref{lem:c22pres} and \ref{lem:g23pres}, then $U_v=\langle U_1,U_2,U_n\rangle=\langle U_1,U_{n-1},U_n\rangle.$
\end{lemma}
\begin{proof}
	Let $H=\langle U_1,U_{2},U_n\rangle$ and let $K=\langle U_1,U_{n-1},U_n\rangle.$ In both cases we have $U'_v\le H,K\le U_v,$ and since $[U_v:U'_v]$ is prime, we get $H=U'_v$ or $U_v$ and similarly for $K.$

	Now suppose $\lk(v)$ is associated to $C_2(2).$ Using the presentation we know that $u_1,u_2,u_4\in H$ and thus $u_2[u_1,u_4]=u_3\in H.$ Since $U_v$ is generated by $\{u_1,u_2,u_3,u_4\}$ we get $H=U_v.$ Similarly, $u_2=[u_1,u_4]u_3\in K$ so $K=U_v$ as well.

	Now suppose $\lk(v)$ is associated to $G_2(3).$ Since the presentation is more complicated in this case we can use a slightly different argument. By Corollary \ref{cor:phiv}, there is a surjective homomorphism $\phi_v:U_v\to C$ where $C$ is a non-trivial cyclic group such that $U'_v\subset \ker \phi_v.$ Since $C$ is cyclic we get $\phi_v(x_4(c_6))=\phi_v([x_2(1),x_6(2)])=[\phi_v(x_2(1)),\phi_v(x_6(1))]=1$ and thus $U_4\subset \ker \phi_v.$ A similar argument shows that $U_3\subset \ker \phi_v.$

	If $H=U'_v$ then get $U_2\subset \ker \phi_v$ as well. This means that \[x_5(c_1)=[x_1(1),x_6(1)]x_4(-c_4)x_2(-c_3)x_3(-c_2)\in \ker \phi_v\] and thus $U_5\subset \ker \phi_v.$ Since $\ker \phi_v$ contains $U_i$ for all $1\le i\le 6,$ it must be the trivial map which is a contradiction, as it is a surjection onto a non-trivial group. Thus $H\neq U'_v$ and $H=U_v$ as desired. A similar argument shows that $K=U_v$ and thus $U_v=\langle U_1,U_2,U_n\rangle=\langle U_1,U_{n-1},U_n\rangle$ as desired.
\end{proof}

So far we have only considered each vertex $v$ and $U_v$ separately. But in the Coxeter complex $\Sigma,$ we have not only a collection of vertices, but an action of the group $W$ on the vertices which behaves nicely with properties like the type of a vertex. We will show that the $W$ action also interacts nicely with $U_v$ and $\phi_v$ in a similar way.

\begin{lemma}
	\label{lem:resporder}
	Suppose $v$ is a vertex of $\Sigma$ of type $s,$ $|\st(v)|=2n,$ and $[U_v:U'_v]\ge 2.$ Also suppose that $w$ is an element of $W$ such that $w\gamma$ is a positive root at $wv$ for every positive root $\gamma$ at $v.$ Then there are standard labelings $\alpha_1,\dots,\alpha_n$ and $\alpha'_1,\dots,\alpha'_n$ of the positive roots through $v$ and $wv$ respectively such that $\alpha'_i=w\alpha_i$ for all $i.$ In particular, $w$ sends roots at $v$ which are simple to roots at $v'$ which are also simple. Furthermore, if $v'$ is any vertex of $\Sigma$ of type $s$ then there is a $w\in W$ such that $wv=v'$ and $w\gamma$ is a positive root at $v'$ for any positive $\gamma$ at $v.$
\end{lemma}
\begin{proof}
	Recall a standard labeling is on of the form $\alpha_1,\dots, \alpha_n$ where $\alpha_i\cap \alpha_j\subset \alpha_k$ for all $1\le i\le k\le j\le n.$ If $w$ sends all of the positive roots at $v$ to the positive roots at $wv$ then $w$ induces a bijection on the positive roots at $v$ and $wv.$ Now we can define a labeling of the positive roots at $wv$ by $\alpha'_i=w\alpha_i$ for all $i.$ It only remains to check that this is a standard labeling. If $1\le i\le k\le j\le n$ then $\alpha_i\cap \alpha_j\subset \alpha_k$ and thus $\alpha'_i\cap \alpha'_j=w\alpha_i\cap w\alpha_j\subset w\alpha_k=\alpha_k'$ so this is a standard labeling as desired.

Now it suffices to show that such a $w$ exists for any vertex $v'$ in $\Sigma.$ Since the $W$ action on $\Sigma$ is transitive on vertices of the same type, it will suffice to show the result when $v$ is a vertex of the fundamental chamber $C.$ Let $D=\proj_{v'}(C)$ so that $d(D,C)$ is minimal among all chambers of $\st(v').$ Then we know that no walls through $v'$ can separate $D$ and $C,$ because crossing one of these walls would produce a chamber in $\st(v)$ which is closer to $C.$ Therefore, a root at $v'$ is positive if and only if it contains $D.$

Now choose the unique $w\in W$ such that $D=wC.$ We claim that $w$ satisfies the desired properties. First of all, $v$ is a vertex of $C$ of type $s$ and thus $wv$ is a vertex of $wC=D$ of type $s.$ But we know that $v'$ is a vertex of $D$ of type $s$ by definition and thus $wv=v'$ as desired. Now suppose $\gamma$ is any positive root at $v.$ Then $C\in \gamma$ and thus $D=wC\in w\gamma$ and thus $C\in w\gamma$ so $w\gamma$ is positive at $wv=v'.$ Now this $w$ sends positive roots at $v$ to positive roots at $v'$ as desired.

\end{proof}

Before moving on it is worth clarifying that the type $s$ of the vertex $v$ in the previous lemma can by any type, not just the literal type $s$ in the definition of $W.$

The previous result can also be used to show that the $W$ action on $\Sigma$ also behaves nicely with respect to the group $U_v$ and the homomorphisms $\phi_v$ when they exit.
\begin{cor}
	\label{cor:respectphiv}
	Suppose $v$ is a vertex of $\Sigma$ with $|\st(v)|=2n$ and $[U_v:U'_v]\ge 2$ and $v'$ is any other vertex of $\Sigma$ of the same type. Then there is an isomorphism between $U_v$ and $U_{v'}$ which sends $U'_v$ to $U'_{v'}.$ Consequently, $[U_v:U'_v]=[U_{v'}:U'_{v'}],$ $\phi_v$ exists if and only if $\phi_{v'}$ exists, and if $\phi_v$ exists then this isomorphism sends $\ker \phi_v$ to $\ker \phi_{v'}.$ If $w$ is any element of $W$ such that $wv=v'$ and $w\gamma$ is positive for all positive $\gamma$ at $v,$ then this isomorphism can be defined by the property that $U_\gamma$ is sent to $U_{w\gamma}$ for every $\gamma$ at $v.$
\end{cor}
\begin{proof}
	Let $w$ be any element of $W$ with $wv=v'$ which sends positive roots at $v$ to positive roots at $v'.$ Such a $w$ is guaranteed to exist by Lemma \ref{lem:resporder}. By Proposition 8.54 in \cite{buildings} and the there is an element $\tilde{w}\in G$ such that $\tilde{w}U_\alpha (\tilde{w})^{-1}=U_{w\alpha}$ for all $\alpha\in \Phi.$ Let $f_w:G\to G$ be the isomorphism of conjugation by $\tilde{w}.$ Since $w\gamma$ is positive at $v'$ for every positive root $\gamma$ at $v$ we know that $f_w(U_\gamma)=U_{w\gamma}\subset U_{v'}$ and thus $f_w$ restricts to a homomorphism $\bar{f}_w:U_v\to U_{v'}$ which is necessarily injective. But $w$ also give a bijection on positive roots at $v$ and $v',$ and $U_{v'}$ is generated by positive root groups at $v'$ so $\bar{f}_w$ is surjective and thus an isomorphism. Now it remains to check it satisfies the rest of the properties. 

	Since $w$ preserves standard labelings at $v$ and $v'$ we know that it also preserves simple roots. Thus $\bar{f}_w(U_{\alpha_1})=U_{\alpha'_1}$ for a standard labeling, and similarly for $U_{\alpha_n}$ and $U_{\alpha'_n}.$ Since $U'_v=\langle U_{\alpha_1},U_{\alpha_n}\rangle$ and $U'_{v'}=\langle U_{\alpha'_1},U_{\alpha'_n}\rangle$ we can also see that $\bar{f}_w$ sends $U'_v$ to $U'_{v'}.$ Since $\bar{f}_w$ is an isomorphism it also preserves index so $[U_v:U'_v]=[U_{v'}:U'_{v'}].$

	For any vertex $v,$ the map $\phi_v$ exists if and only if $[U_v:U'_v]\ge 2$ and thus $\phi_v$ will exist exactly when $\phi_{v'}$ exists. By Corollary \ref{cor:uniquephiv} we know that $\ker \phi_v$ is a proper normal subgroup of $U_v$ containing $U'_v$ and thus $\bar{f}_w(\ker \phi_v)$ will be a proper, normal subgroup of $U_{v'}$ containing $U'_{v'}.$ By Corollary \ref{cor:uniquephiv} again this means $\bar{f}_w(\ker \phi_v)=\ker \phi_{v'}$ which completes the result.
\end{proof}

The main idea of our results will be to extend the map $\phi_v$ in a certain way to a map on all of $U_+,$ and the main difficulty in the proof will be to show that this extension is well defined. Perhaps the easiest way to prove this is to use a presentation of $U_+$ and the universal property which says if we define a map on generators, which sends all relations to the identity, then the map defines a homomorphism. The group $U_+$ does admit a nice presentation as shown in Theorem 8.84 of \cite{buildings}, which we will repeat here in the following lemma
\begin{lemma}
	\label{lem:upres}
	Let $(G,(U_\alpha)_{\alpha\in \Phi},T)$ be and RGD system. For each $\alpha\in \Phi$ choose a set $S_\alpha\subset U_\alpha,$ and a set of words $R_\alpha$ with letters in $S_\alpha$ so that $\langle S_\alpha|R_\alpha\rangle$ is a presentation of $U_\alpha.$ Then for each pre-nilpotent pair $\{\alpha,\beta\}$ and any $u_\alpha\in S_\alpha$ and $u_\beta\in S_\beta,$ we can write $[u_\alpha,u_\beta]=v$ where $v$ is a word in $\cup_{\gamma \in (\alpha,\beta)}S_\gamma.$ Furthermore, one obtains a presentation of $U_+$ by combining the relations $R_\alpha$ for all $\alpha$ as well as the commutator relations $[u_\alpha,u_\beta]=v$ where $\alpha,\beta$ range over all pre-nilpotent pairs, and $u_\alpha,u_\beta$ range over all of $S_\alpha$ and $S_\beta.$
\end{lemma}

There is another result which we will use extensively in the following chapters, but which is slightly different from what we have done so far. We state the following fact about the geometry of certain Coxeter complexes. 

\begin{lemma}[Triangle Condition]
	\label{lem:tri}
	Suppose $\Sigma$ is a Coxeter complex of type $(W,S)$ where $S=\{s,t,u\},$ $3\le m(s,t)\le m(s,u)\le m(t,u)<\infty,$ and $m(t,u)\ge 4.$ Let $\alpha_1,\alpha_2,\alpha_3$ be roots of $\Sigma$ such that $\partial\alpha_i\cap \partial\alpha_j\neq \emptyset$ for all $i,j$ but $\partial\alpha_1\cap \partial\alpha_2\cap \partial \alpha_3=\emptyset.$ If we assume that $\partial\alpha_i\cap \partial\alpha_j\subset \alpha_k$ for $i\neq j\neq k$ then $\alpha_1\cap \alpha_2\cap\alpha_3$ is a chamber of $\Sigma.$
\end{lemma}

The previous lemma essentially says that a ``Triangle'' formed by 3 walls of $\Sigma$ under the specified conditions must be a single chamber. One way we will use this lemma is by showing two walls cannot intersect, if the resulting triangle would contain more than one chamber.


In the next two chapters we will prove new results about finite generation in RGD systems when the associated building $\Delta$ has exceptional links as described in this chapter.

\end{document}
