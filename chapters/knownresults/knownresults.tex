\documentclass[class=book, crop=false,12 pt]{standalone}
\usepackage[subpreambles=true]{standalone}
\usepackage{import}
\usepackage{/home/mark/Documents/gradschool/research/thesis/preamble}

\begin{document}
\chapter{Known Results on Finite Generation}

Throughout this section, $\G$ will be a Kac-Moody group with rank 3 Weyl group $W$ over a field $k.$ We will also assume that $W$ is defined by the coxeter diagram with edge labels $a,b,c\in\{3,4,6\}$ with $a\le b\le c$ and $c\ge 4.$ This last condition ensures that $W$ is hyperbolic. Let $\Sigma$ be the Coxeter complex of $W.$ Let $\Phi^+$ be the positive roots of $\Sigma,$ and for any $\alpha\in \Phi^+$ we will let $\U_\alpha$ be the root group associated to $\alpha.$

For any vertex $v$ of $\sigma,$ there will be some walls of $\Sigma$ which pass through $v,$ and for each of these walls we have a unique \emph{positive} root. We will call these the \textbf{positive roots at v} and denote them by $\Phi_v.$ Recall that $\st(v)$ is defined as all the chambers containing $v$ as a vertex. If there are $n$ positive roots at $v$ then $|\st(v)|=2n.$ Furthermore, it is possible to label the positive roots at $v$ as $\alpha_1,\dots,\alpha_n$ in such a way that $\alpha_i\cap \alpha_j\subset \alpha_k$ for any $1\le i\le k\le j\le n.$ This ordering is unique upto a reversal of the form $\alpha_i\mapsto \alpha_{n+1-i}.$ This possible reversal will not matter in most cases and if it does then a choice of $\alpha_1$ will be specified. It does however allow us to unambigiously define $\alpha_1$ and $\alpha_n$ as the \textbf{simple} roots at $v.$ They are the unique positive roots at $v$ whose intersection is contained in all other positive roots at $v.$

Now we can define $\U_v$ to be the subgroup of $\G$ generated by all of the root groups of the positive roots at $v.$ That is
\[
\U_v=\langle \U_\alpha|\alpha \text{ is a positive root at }v\rangle=\langle \U_\alpha|\alpha\in \Phi_v\rangle
\]
Most of the time the group $\U_v$ is generated by $\U_1,\U_n$ which are the simple root groups at $v.$ However, there are some exceptions to this. Let $\U'_v=\langle \U_1,\U_n\rangle$ where $2n=|\st(v)|.$ Then we have the following results about the $\U_v$ which comes from the known theory about rank 2 Moufang Polygons.
\begin{lemma}
	\label{simpleindex}
	Let $v$ be a vertex of $\Sigma$ with $|\st(v)|=2n.$ Let $\U'_v=\langle \U_1,\U_n\rangle$ where $\U_1,\U_n$ are the root groups of the simple roots at $v.$ Then we can describe $[\U_v:\U'_v]$ with the following table
	\[
		\begin{array}{c|c|c}
			n&|k|&[\U_v:\U'_v]\\\hline
			4&2&2\\
			6&2&4\\
			6&3&3
		\end{array}
	\]
	and $[\U_v:\U'_v]=1$ in all other cases. In other words, $\U'=\U$ with the exception of the 3 cases above.
\end{lemma}

We can in fact say a little more than that when $|k|=2$ and $n=6.$
\begin{lemma}
	\label{order6f2index}
	Suppose that $\U$ is defined over $k=\F{2}$ and $v$ is a vertex of $\Sigma$ with $|\st(v)|=2n=12.$ Then it is possible to label the positive roots at $v$ as $\U_1,\dots,\U_6$ in such a way that $\U''_v=\langle \U_1,\U_5,\U_6\rangle$ has index $2$ in $\U_v.$
\end{lemma}

These two lemmas together give the following corollary.
\begin{cor}
	\label{phiv}
	Suppose $v$ is a vertex of $\Sigma$ with $|\st(v)|=2n$ and $\U_1,\U_n$ the simple roots at $v.$ Suppose that $[\U_v:\U'_v]\ge 2.$ Let $H$ be the cyclic group of order $|k|$ where $k$ is the field over which $\U$ is defined. Then there is a surjective group homomorphism, call it $\phi_v:\U_v\to H$ such that $\phi_v(\U_1)=\phi_v(\U_n)=\{1\}.$
\end{cor}
\begin{proof}
	Let $\U'_v=\langle \U_1,\U_n\rangle.$ Since $[\U_v,\U'_v]\neq 1$ we know we must be in one of the three exceptional cases above. If $n=4$ and $|k|=2$ then $[\U_v,\U'_v]=2$ and thus $\U'_v$ is a normal subgroup of $\U_v$ and the quotient has order $2.$ So we can define $\phi_v:\U_v \to H$ to be the quotient map $\U_vto \U_v/\U'_v.$

	If $n=6$ and $|k|=3$ then $[\U_v:\U'_v]=3.$ But $\U_v$ is a 3-group and thus $\U'_v$ is normal and the quotient has order 3, so we can construct $\phi_v$ as before.

	Now suppose $n=6$ and $|k|=2.$ Then by Lemma \ref{order6f2index}, we can define $\U''_v=\langle \U_1,\U_5,\U_6\rangle$ so that $[\U_v:\U''_v]=2$ and thus $\U''_v$ is normal and the quotient has order $2.$ In this case we can define $\phi_v$ to be the quotient map $\U_v\to \U_v/\U''_v.$
\end{proof}

The following corollary will show that we do not have very much wiggle room when defining $\phi_v,$ and thus if we can write any function which ``looks like'' $\phi_v$ then they must be esentially the same.
\begin{cor}
	\label{uniquephiv}
	Suppose $v$ is a vertex of $\Sigma$ with $|\st(v)|=2n$ and $\U_1,\U_n$ the simple root groups at $v.$ Let $\phi_v$ be defined as in the previous corollary. Then $\ker \phi_v$ is the unique, proper, normal subgroup of $\U_v$ which contains $\U_1$ and $\U_n.$
\end{cor}
\begin{proof}
	If $n=4$ and $|k|=2$ or $n=6$ and $|k|=3$ then the result is clear as $\U'_v=\langle \U_1,\U_n\rangle=\ker\phi_v$ is normal and has prime index, so there can be no other proper normal subgroups containing it.

	If $n=6$ and $|k|=2$ then $[\U_v:\U'_v]=4$ but $\U'_v$ is not a normal subgroup. It can be shown if $N$ is a normal subgroup containing $\U'_v$ then $\U_5\subset N$ as well, and thus $\U''_v\subset N.$ But $[\U_v:\U''_v]=2$ and thus $\U''_v$ is the only proper normal subgroup containing $\U_1,\U_n$ as desired.
\end{proof}
This isn't really a proof but I will fill in the details later. I was more just reminding myself of the arguments.


The general theory gives us the following result
\begin{theorem}
	\label{knownfgresult}
	Let $\G$ be a Kac-Moody group over $k$ with rank 3 Weyl group $W$ as before. For any vertex $v$ of $\Sigma,$ let $\U'_v=\langle \U_1,\U_n\rangle$ where $\U_1,\U_n$ are the simple roots at $v.$ If $\U'_v=\U_v$ for all $v\in \Sigma$ then $\U$ is finitely generated.
\end{theorem}

%
I use this lemma later. This isn't organized yet but I wanted to have it so my reference aren't broken.
\begin{lemma} Let $\alpha,\beta,\beta+\alpha,\beta+2\alpha$ be the positive roots of a root system of type $C_2$ and $\U$ the unipotent subgroup of $C_2(\F{2}).$ Then $\U=\langle \U_\alpha,\U_\beta,\U_{\beta+\alpha}\rangle=\langle \U_\alpha,\U_\beta,\U_{\beta+2\alpha}\rangle.$
	\label{c2f2gen}
\end{lemma}
%
%\begin{lemma}
%	\label{g2f3gen}
%Let $\alpha,\beta,\beta+\alpha,\beta+2\alpha,\beta+3\alpha,2\beta+3\alpha$ be the positive roots of a root system of type $G_2$ and $\U$ the unipotent subgroup of $G_2(\F{3}).$ Then $\U=\langle \U_\alpha,\U_\beta,\U_{\beta+\alpha}\rangle=\langle \U_\alpha,\U_\beta,\U_{\beta+3\alpha}\rangle.$
%\end{lemma}
%
%\begin{lemma}
%	\label{g2f2gen}
%	Let $\alpha,\beta,\beta+\alpha,\beta+2\alpha,\beta+3\alpha,2\beta+3\alpha$ be the positive roots of a root system of type $G_2$ and $\U$ the unipotent subgroup of $G_2(\F{2}).$ Then $\U=\langle \U_\alpha,\U_\beta,\U_{\beta+\alpha}\rangle$ but $\langle \U_\alpha,\U_\beta,\U_{\beta+3\alpha}\rangle\subsetneq \U.$
%\end{lemma}
%
%\begin{lemma}
%	\label{g2f2unique}
%	If $\G=\G_2(\F{2})$ then there is a unique surjective homomorphism $\phi:\U\to K$ where $K$ is the cyclic group of order 2, such that $\phi(\U_\alpha)=\phi(\U_\beta)=1.$
%\end{lemma}

\end{document}
