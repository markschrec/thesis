\documentclass[class=book, crop=false,12 pt]{standalone}
\usepackage[subpreambles=true]{standalone}
\usepackage{import}
\usepackage{/home/mark/Documents/gradschool/research/thesis/preamble}

\begin{document}

\chapter{Conditions for Infinite Generation}
\section{Extension of $\phi_v$}
%Throughout this section we will use the following assumptions. Let $\G$ be a Kac-Moody group over the finite field $k$ with rank 3 Weyl group $W.$ Assume that $W$ is defined by the Coxeter diagram with edge labes $a\le b\le c$ with $a,b,c\in \{3,4,6\}$ and $c\ge 4.$ Let $\Sigma$ be the Coxeter complex of $W$ and let $\Phi^+$ be the positive roots of $\Sigma.$ Let $C$ be the fundamental chamber of $\Sigma.$

Throughout this chapter $(G,(U_\alpha)_{\alpha\in \Phi},T)$ will be an RGD system of type $(W,S)$ with the following assumptions:
\smallskip
\begin{equation}
	\tag{A} 
\begin{aligned}
	&W \text{ has rank 3, }S=\{s,t,u\},\: a=m(s,t),b=m(s,u),c=m(t,u)\text{ and }3\le a\le b\le c\\
	&[U_\alpha,U_\beta]=1\text{ when }\alpha,\beta \text{ are nested}
\end{aligned}
\end{equation}
\smallskip

Let $\Sigma$ be the Coxeter complex of $W$ with fundamental chamber $C,$ and $\Phi_+$ be the positive roots of $\Sigma.$ We will also let $U_+=\langle U_\alpha|\alpha\in \Phi_+\rangle$ be the subgroup of $G$ generated by the positive root groups. We will also note that properties of RGD systems tell us that $a,b,c\in \{2,3,4,6,8\}$ and thus by \eqref{assume} we know that $a,b,c\in \{3,4,6,8\}.$


%Throughout this section we will use the following assumptions. Let $(G,\(U_\alpha\)_{\alpha\in \Phi},T)$ be an RGD system of type $(W,S).$ Let $\Sigma$ be the Coxeter complex of $W$ and recall that $\Phi$ is the set of roots of $\Sigma.$

We can also recall some terminology from the last chapter. We will say that $\alpha$ is a positive root at $v$ if $\alpha$ is positive and the wall $\partial\alpha$ passes through $v$ and we will denote the positive roots at $v$ as $\Phi^v_+.$ Then we can define $U_v=\langle U_\alpha|\alpha\in \Phi^v_+\rangle.$ We can also label the roots of $\Phi^v_+$ as $\alpha_1,\dots,\alpha_n,$ where $2n=|\st(v)|$ in $\Sigma,$ in such a way that $\alpha_i\cap \alpha_j\subset \alpha_k$ for $1\le i\le k\le j\le n.$ With this labeling we will call $\alpha_1,\alpha_n$ the simple roots at $v$ and we will note that they do not depend on the labeling. We will use this labeling many times throughout the section and we will refer to it as the standard labeling. This definition is a slight abuse as this labeling scheme is not unique, however, the only other possible labeling is given by flipping the order and sending $\alpha_i\mapsto \alpha_{n+1-i}.$ In practice, this ambiguity will not matter and so most of the time we can simply refer to the standard labeling without any further detail.

We say that two distinct positive roots $\alpha,\beta$ are  a \emph{pre-nilpotent} pair if $\alpha\cap \beta$ and $(-\alpha)\cap (-\beta)$ both contain a chamber. There is a very nice characterization of pre-nilpotent roots which we will use in the remainder of the chapter. Two roots $\alpha,\beta$ form a pre-nilpotent pair if and only if one of the following holds:
\[
	\text{(i) }\partial \alpha \cap \partial \beta \neq \emptyset\qquad \text{(ii) } \alpha,\beta \text{ are nested}
\]
where we say $\alpha,\beta$ are nested if $\alpha \subset \beta$ or vice versa. By definition, $\partial \alpha \cap \partial \beta=\emptyset$ if $\alpha,\beta$ are nested so only one of the previous conditions can be satisfied.

We will also briefly recall the definitions of open and closed intervals of roots. If $\alpha,\beta$ are two pre-nilpotent, positive roots then we define the closed interval
\[
	[\alpha,\beta]=\{\gamma\in \Phi_+|\alpha\cap\beta\subset \gamma\text{ and }(-\alpha)\cap (-\beta)\subset -\gamma\}
\]
and the open interval $(\alpha,\beta)=[\alpha,\beta]\setminus\{\alpha,\beta\}.$ In a similar manner as before, we will define $U_{(\alpha,\beta)}=\langle U_\gamma|\gamma\in (\alpha,\beta)\rangle.$

One feature of the standard labeling is that it allows us to describe some of these intervals in a very natural way. If $v$ is some vertex of $\Sigma$ and $\alpha_1,\dots,\alpha_n$ are the positive roots through $v$ with the standard labeling, then $[\alpha_i,\alpha_j]=\{\alpha_k|i\le k \le j\}$ whenever $i\le j.$ Similarly we get $(\alpha_i,\alpha_j)=\{\alpha_k|i<k<j\}$ whenever $i<j.$

%The group $U_+$ is generated by the set of all $U_\alpha$ for $\alpha\in \Phi_+$ but we can actually say a little bit more. Each $U_\alpha$ is isomorphic to the additive group of $k,$ so there is a finite set of relations $\mathcal{R}_\alpha$ which make $\langle U_\alpha|\mathcal{R}_\alpha\rangle$ into a presentation of $U_\alpha.$ The theory of RGD systems tells us that we can get a presentation of $U$ of the form
%\[
%	U=\left\langle U_\alpha\text{ for }\alpha\in \Phi_+|\{\mathcal{R}_\alpha\}_{\alpha\in \Phi_+},\left\{[u,u']=v|u\in U_\alpha,u'\in U_\beta, \alpha,\beta \text{ pre-nilpotent pair}\right\}\right\rangle
%\]
%where $v$ is a word in $U_{(\alpha,\beta)}$ which depends on $u$ and $u'.$  Furthermore, the theory of Kac-Moody groups tells us that if $u\in U_\alpha$ and $u'\in U_\beta$ with $\alpha,\beta$ nested, then $v=1.$

By definition, $U_+$ is generated by the $U_\alpha$ for all positive roots $\alpha.$ However we can say a little bit more about $U_+.$ Each $U_\alpha$ will have its own set of relations $\mathcal{R}_\alpha.$ The theory of RGD systems tells us that we have a presentation of $U_+$ of the following form
\[
	U_+=\langle U_\alpha,\alpha\in \Phi_+|\mathcal{R}_\alpha,\alpha\in \Phi_+,[u,u']=v,u\in U_\alpha,u'\in U_\beta,\{\alpha,\beta\} \text{ a pre-nilpotent pair}\rangle
\]
where $v$ is a word in $U_{(\alpha,\beta)}$ which depends on $u,u'.$ Furthermore, by condition \eqref{assume} we know that $[u,u']=1$ if $\alpha$ and $\beta$ are nested. Therefore, the only non-trivial commutator relations will occur when $\partial\alpha\cap \partial\beta\neq \emptyset.$

Let $U'_v=\langle U_1,U_n\rangle$ for any vertex $v\in \Sigma,$ where $U_1$ and $U_n$ are the simple roots at $v.$ By Theorem \ref{knownfgresult} we know that $U$ is finitely generated if $U'_v=U_v$ for all $v\in \Sigma.$ What we will show in the rest of the chapter is that if $U'_v\neq U_v$ for some $v\in \Sigma,$ then most of the time $U$ will not be finitely generated. Our general strategy will be as follows. If $v$ is some vertex of $\Sigma$ such that $U'_v\neq U_v$ then Corollary \ref{phiv} shows the existence of a surjective group homomorphism $\phi_v:U_v\to H$ where $H$ is a cyclic group of the appropriate order. If we can extend this map to all of $U_+$ in a certain way then we will be able to show certain root groups must be in any generating set of $U_+.$ If we can do this for enough $v$ then we will be able to show that $U_+$ is not finitely generated.

Our first lemma will define our notion of extending $\phi_v,$ and give a sufficient condition for this extension to exist.
\begin{lemma}
	Suppose that $v$ is a vertex of $\Sigma$ such that $U'_v=\langle U_1,U_n\rangle\neq U_v,$ where $U_1,U_n$ are the simple roots at $v.$ Then there is a surjective group homomorphism $\phi_v:U_v\to H$ with the property that $\phi_v(U_1)=\phi_v(U_n)=\{1\},$ where $H$ is a cyclic group. Also suppose that for any positive root $\gamma$ with $v\in \partial \gamma$ which is not simple at $v,$ that $\gamma$ is simple at $y$ for all $y\in \partial \gamma$ with $y\neq v.$ Then the map $\tilde{\phi}_v:\cup_{\gamma\in \Phi_+}U_\gamma\to H$ defined by
\[
	\tilde{\phi}_v(u)=\begin{cases} \phi_v(u)&\text{if }u\in U_\gamma\text{ and }v\text{ lies on }\partial\gamma\\
		1&\text{ otherwise}
	\end{cases}
\]
Extends uniquely to a well defined group homormoprhism $\tilde{\phi}_v:U_+\to H.$
	\label{existence}
\end{lemma}
\begin{proof}
	We know that the map $\phi_v$ exists by Corollary \ref{phiv}. We have a presentation for $U_+$ and we have defined $\tilde{\phi}_v$ on the generators of $U_+,$ so in order to check that it is well defined we will need to verify that the relations of $U_+$ are satisfied in the image.

	There are three types of relations in the presentation for $U_+.$ There are relations within the same root group so that $U_\alpha$ for all positive roots $\alpha.$ There are also relations between root groups of pre-nilpotent pairs where either the walls intersect or the roots are nested.

	Let $R_\alpha$ be a relation for $U_\alpha$ where $R_\alpha$ is considered as a word with letters in $U_\alpha.$ If $v$ lies on $\partial \alpha$ then $\tilde{\phi_v}(R_\alpha)=\phi_v(R_\alpha)=1$ since $\phi_v$ is a well defined homomorphism. Otherwise, every element of $U_\alpha$ is sent to 1 and thus $\tilde{\phi_v}(R_\alpha)=1$ as well so that $R_\alpha$ is mapped to the identity as desired.

	Now suppose that $\alpha$ and $\beta$ are any two positive roots. If $\alpha,\beta$ nested, then \eqref{assume} tells us that $[U_\alpha,U_\beta]=1.$ Since the codomain of $\tilde{\phi_v}$ is an abelian group, then any relation of the form $[x,y]=1$ will be satisfied by the image.


	Now suppose that $\partial \alpha$ and $\partial \beta$ meet at a point $y$ and consider any relation of the form $[u_\alpha,u_\beta]=w$ where $u_\alpha\in U_\alpha,$ $u_\beta\in U_\beta,$ and $w$ is a word in $U_{(\alpha,\beta)}\subset U_y.$ Again, note that the image of the left side of this equation will always be the identity as the codomain is still abelian. If $y=v$ then $U_y=U_v$ and thus $\tilde{\phi_v}(w)=\phi_v(w)=1$ because $\phi_v$ is well defined.

	Now suppose that $y\neq v.$ Then we can label the positive roots passing through $y$ as $\gamma_1,\cdots,\gamma_n$ in such a way that $(\gamma_i,\gamma_j)=\{\gamma_{r}|i<r<j\}$ whenever $i<j.$ In this case we can can say without loss of generality that $\alpha=\gamma_l$ and $\beta=\gamma_m$ with $l<m.$  There can be at most one root whose wall passes through $y$ and $v,$ which we will call $\gamma_k$ if it exists. If $\gamma_k$ does not exist, or $k\le l$ or $k\ge m$ then the root $\gamma_k$ is not contained in $(\alpha,\beta)$ and thus $\tilde{\phi_v}(U_\delta)=1$ for all $\delta\in (\alpha,\beta).$ This means $\tilde{\phi_v}(w)=1$ and the relation is satisfied.

	Now we suppose that $\gamma_k$ exists and $l<k<m.$ Then $\gamma_k$ is not simple at $y$ and thus $\gamma_k$ must be simple at $v$ by assumption. This means $\tilde{\phi_v}(U_{\gamma_k})=\phi_v(U_{\gamma_k})=1$ by the construction of $\phi_v.$ Since $\tilde{\phi_v}(U_{\gamma_i})=1$ for all $i\neq k$ by definition, this means that $\tilde{\phi_v}(w)=1$ showing the relation is satisfied and giving the desired result.
\end{proof}

Now Lemma \ref{existence} gives a sufficient condition for the existence of $\tilde{\phi_v}$ which is fairly easy to check. This will be the main tool we use in the remainder of the section. 

Recall our assumptions in \eqref{assume} that $(W,S)$ is a rank 3 Coxeter system with $S=\{s,t,u\}.$ We also assumed that $a=m(s,t),b=m(s,u),$ and $c=m(t,u)$ with $3\le a\le b \le c.$ Let $x$ be the vertex of $C$ of type $s$ and assume that $[U_x:U'_x]\ge 2.$ By the characterization of such $U_x$ we know that $c\ge 4.$ Our first step in the main proof will be to show that $\tilde{\phi}_x$ exists. We will do this by applying Lemma \ref{existence} and to do this we need to prove the following result about roots through $x.$

\begin{lemma}
\label{positiveeverywhereelse}
	Let $x$ be the vertex of $C$ of type $s.$ If $\gamma$ is any positive root at $x,$ and $y$ is any other vertex on $\partial\gamma,$ then $\gamma$ is simple at $y.$
\end{lemma}
\begin{proof} 
	Suppose that $\gamma$ is not simple at $y.$ Then we can label the positive roots at $y$ as $\delta_1,\dots,\delta_m$ in such a way that $\delta_i\cap \delta_j\subset \delta_k$ for $1\le i\le k\le j.$ In this case we have $\delta_1,\delta_m$ are simple at $y$ and $\gamma=\delta_r$ for some $1<r<m.$ But $x$ is a vertex of $C$ and $C\in \delta_1\cap \delta_m$ and thus $x\in \delta_1\cap \delta_m$ as well. We know that $x$ lies on $\partial \delta_r$ by assumption and thus $x$ is an element of $\partial \delta_r \cap \delta_1\cap \delta_m.$ But this is impossible as we can observe from the geometry of $\Sigma$ that $\partial \delta_i\cap \delta_1\cap \delta_m=\{y\}$ for all $1<i<m.$ Thus $\gamma$ is simple at $y$ as desired.
\end{proof}
Despite some of the technical details the previous result should be intuitively clear. The walls through $y$ will divide $\Sigma$ into $2m$ regions, and the region which contains $C$ will be bounded by the two simple roots. Since $x$ lies on $\partial \gamma,$ it is impossible for any other roots through $y$ to be any ``closer'' to $C$ and thus $\gamma$ must be simple at $y$ as we proved.
\begin{cor}
	Let $x$ be the vertex of $C$ of type $s,$ and assume that $[U_x:U'_x]\ge 2.$ Then the map $\tilde{\phi}_x$ as defined in Lemma $\ref{existence}$ is well defined.
\end{cor}
\begin{proof}
	Let $\gamma$ be any non-simple, positive root through $x$ and let $y$ be another vertex on $\partial \gamma.$ Then by the previous lemma, $\gamma$ is simple at $y$ and thus $\tilde{\phi_x}$ exists by Lemma \ref{existence}.
\end{proof}

The remainder of the section will be used to show that we can use $\tilde{\phi}_x$ and the $W$ action on $\Sigma$ to construct a large family of vertices for which $\tilde{\phi}_v$ exists.


We can label the roots through $x$ as $\alpha_1,\dots,\alpha_n$ so that $\alpha_1$ and $\alpha_n$ are the simple roots at $x.$ Also note that $n=c.$ The ordering on these roots is chosen so that $\alpha_i\cap \alpha_j\subset \alpha_k$ for any $1\le i \le k \le j\le n.$ This is equivalent to the condition that $(\alpha_i,\alpha_j)=\{\alpha_k|i<k<j\}$ for any $i<j.$

We can describe any root in terms of a pair of adjacent chambers. We can also identify $\mathcal{C}(\Sigma)$ with $W$ where the chamber $wC$ is associated to $w.$ If we use this identification then we can describe the roots as follows
\begin{align*}
	\alpha_1&=\{D\in \Sigma|d(D,C)<d(D,tC)\}=\{w\in W|\ell(w)<\ell(tw)\}\\
	\alpha_n&=\{D\in \Sigma|d(D,C)<d(D,uC)\}=\{w\in W|\ell(w)<\ell(uw)\}
\end{align*}
In a similar way we can define two more roots
\begin{align*}
	\beta&=\{D\in \Sigma|d(D,tC)<d(D,tsC)\}=\{w\in W|\ell(tw)<\ell(stw)\}\\
	\beta'&=\{D\in \Sigma|d(D,uC)<d(D,usC)\}=\{w\in W|\ell(uw)<\ell(suw)\}
\end{align*}
\begin{figure}[h]
	\label{defineD}
	\begin{center}
	\resizebox{3.5in}{!}{\subimport{diagrams/}{defineD.tex}}
	\caption{The Roots $\alpha_1,\alpha_n,\beta,\beta'$ with the region $\D$ in yellow.}
\end{center}
\end{figure}

Now we can define $\mathcal{D}=\alpha_1\cap \alpha_n \cap \beta\cap \beta'.$ These roots are chosen and $\D$ is defined in such a way to give the following lemma:

\begin{lemma}
	\label{containD}
	Let $x$ be the vertex of $C$ of type $s$ and assume $W=\langle s,t,u|s^2=t^2=u^2=(st)^a=(su)^b=(tu)^c=1\rangle.$ Let $\D=\alpha_1\cap \alpha_n\cap \beta\cap \beta'$ where $\alpha_1,\alpha_n,\beta,\beta'$ are roots of $\Sigma$ defined by
\begin{align*}
	\alpha_1&=\{D\in \Sigma|d(D,C)<d(D,tC)\}=\{w\in W|\ell(w)<\ell(tw)\}\\
	\alpha_n&=\{D\in \Sigma|d(D,C)<d(D,uC)\}=\{w\in W|\ell(w)<\ell(uw)\}\\
	\beta&=\{D\in \Sigma|d(D,tC)<d(D,tsC)\}=\{w\in W|\ell(tw)<\ell(stw)\}\\
	\beta'&=\{D\in \Sigma|d(D,uC)<d(D,usC)\}=\{w\in W|\ell(uw)<\ell(suw)\}
\end{align*}
If $\gamma$ is a positive root at $x$ which is not simple at $x,$ and $\delta$ is any other positive root such that $\partial\gamma\cap \partial\delta\neq \emptyset,$ then $\D\subset \gamma\cap \delta.$
\end{lemma}
\begin{proof}
	By assumption, $\gamma$ is a positive root through $x$ so $\gamma=\alpha_i$ for some $i.$ Furthermore, we assumed that $\gamma$ was not simple which means $2\le i \le n-1.$ Since $\alpha_1$ and $\alpha_n$ are simple at $x$ we can see that $\mathcal{D}\subset \alpha_1\cap \alpha_n\subset \alpha_i=\gamma.$ Thus it will suffice to prove that $\mathcal{D}\subset \delta.$

	Let $y=\partial\gamma \cap \partial \delta.$ If $y=x$ then $\delta$ is also a root which passes through $x$ and so $\delta=\alpha_j$ for some $j\neq i.$ Then as before we get $\alpha_1\cap \alpha_n\subset \alpha_j=\delta$ and thus $\mathcal{D}\subset \delta$ so that $\mathcal{D}\subset \gamma\cap \delta$ as desired.

	Now suppose that $\partial \gamma\cap \partial \delta=y\neq x.$ From the local geometry of $\Sigma$ around $x$ we can see the following facts. For any $\alpha_i$ with $2\le i\le n-1$ we know that $\partial\alpha_i\cap \alpha_1\cap \alpha_n=\{x\}$ and $\partial\alpha_i\subset \alpha_1\cup\alpha_n.$ Thus the point $y$ will lie in exactly one of $\alpha_1$ or $\alpha_n.$

	First suppose that $y\in \alpha_n$ so that $y\not\in \alpha_1.$ If $\partial\alpha_1\cap \partial\delta=\emptyset$ then there are exactly 3 possibilities. Either $\alpha_1\subset \delta,$ $\delta\subset\alpha_1,$ or $-\delta\subset \alpha_1.$ But the last two possibilites would contradict our assumption that $y\not\in \alpha_1$ and thus we get $\alpha_1\subset \delta$ and thus $\D\subset \alpha_1\subset \gamma\cap\delta$ as desired.

	Alternatively, assume that $\partial\alpha_1\cap \partial\delta=y'.$ Then the points $x,y,y'$ will form a triangle with sides on walls of $\Sigma.$ Then by the triangle condition, these three vertices must form a chamber, call it $E.$ The points $x,y$ lie on $\partial\gamma=\partial\alpha_i$ and the points $x,y'$ lie on $\partial\alpha_1.$ Since $y$ and $y'$ are adjacent this means that either $\gamma=\alpha_2$ or $\gamma=\alpha_n.$ The latter is a contradiction of our assumptions and thus $\gamma=\alpha_2.$ We know that $y$ and $y'$ are adjacent and $y\in \alpha_n.$ Since neither $y$ or $y'$ lies on $\partial\alpha_n$ this means that $y'\in \alpha_n$ as well.

	We know that $E$ is a chamber in $\st(x)$ with a side on $\partial \alpha_1$ and $\partial\alpha_2.$ let $D=tC$ and $D'$ be the chamber opposite $D$ in $\st(x).$ Then either $E=D$ or $E=D'.$ By definition, $\alpha_1$ is the only wall separating $C$ and $tC$ which means $D=tC\in \alpha_n.$ If $E=D'$ then $D'\in \alpha_n$ since $x,y,y'$ all lie in $\alpha_n.$ But this is a contradiction as $\alpha_n$ cannot contain two opposite chambers in $\st(x).$ Thus $E=D=tC$ and $\delta=\beta$ by definition. Thus $\D\subset \beta=\delta$ and $\D\subset \gamma\cap \delta$ as desired.

	If we assume insteach that $y\in\alpha_1$ so that $y\not\in \alpha_n$ then identical arguments show that $\delta=\beta'$ and we can again conclude that $\D\subset \gamma\cap \delta$ as desired.

%Now suppose $y\neq x$ and also assume that $x$ and $y$ are not adjacent. Suppose that $\partial \delta$ meets $\partial \alpha_1$ at a point $y'.$ Then the points $x,y,y'$ form a triangle, whose sides lie on the walls $\partial \gamma,$ $\partial \delta,$ and $\partial \alpha_1.$ The triangle condition then implies that $xyy'$ must be a chamber of $\Sigma,$ which is a contradiction since $x$ and $y$ are not adjacent. Thus $\partial \delta$ does not meet $\partial \alpha_1$ and a similar argument shows that $\partial \delta$ does not meet $\alpha_n.$ 
%
%From the geometry of the Coxeter complex, we can observe that for any $1<i<n$ we have $\partial \alpha_i \cap \alpha_1\cap \alpha_n=x.$ Since $y\neq x$ this means that $y$ does not lie in $\alpha_1\cap \alpha_n.$ We can assume without loss of generality that $y$ does not lie in $\alpha_1.$ We know that $\alpha_1$ and $\delta$ are two positive roots whose walls do not meet, and thus there are exactly three possibilites. Either $\alpha_1\subset \delta,$ $\delta\subset\alpha_1$ or $-\delta\subset \alpha_1.$ The later two cases are impossible as both would imply that $y\in \alpha_1$ which contradicts our assumption. Thus we have $\alpha_1\subset\delta$ which gives $\mathcal{D}\subset\alpha_1\subset \delta$ as desired.
%
%Now we suppose again that $y\neq x$ but $x$ and $y$ are adjacent. Then once again there are two possibilities. If $\partial \delta$ does not meet $\partial \alpha_1$ or $\partial \alpha_n$ then the identical argument from the last paragraph shows that $\mathcal{D}\subset \delta.$
%
%So now we suppose that $\partial \delta$ does meet $\partial \alpha_1$ or $\partial \alpha_n.$ If $\partial \delta$ meets $\partial \alpha_1$ at a point $y',$ then the vertices $xyy'$ will form a triangle which must be a chamber call it $C'.$ This chamber contains, $x,$ and has 2 sides on $\partial \alpha_1$ and $\partial \gamma$ respectively. Since $\gamma$ is not simple, an observation of the chambers around $x$ in Figure \ref{defineD} shows that $\gamma$ must be $\alpha_2$ $C'$ must be $tC$ in which case $\delta=\beta$ by definition. Thus we get $\mathcal{D}\subset \beta =\gamma$ which proves the result.
%
%If $\partial \delta$ meets $\alpha_n$ then an identical argument shows that $\gamma=\beta'$ which also proves the result.
%
\end{proof}


We are now ready to construct a large family of vertices $\{v\}$ for which $\tilde{\phi_v}$ will exist. The idea is as follows. If we take any chamber in $\mathcal{D}$ and treat it as a new ``$C$'' then $\tilde{\phi_x}$ would exist for this ``$C.$'' So what we do is apply elements of $W$ which map the chambers of $\mathcal{D}$ to $C,$ and use these choices of $w$ to get new vertices $v.$

Since the construction of these $\tilde{\phi_v}$ depends on properties of simple roots, we want to know the simplicity behaves nicely with the action of $W.$ To this end we have the following lemma.

\begin{lemma}
	Suppose $v$ is a vertex of $\Sigma$ with simple roots $\gamma,\gamma'$ at $v.$ If $w$ is an element of $w$ such that $w\delta$ is a positive root for all positive $\delta$ at $v,$ then $w\gamma$ and $w\gamma'$ are the simple roots at $wv.$ \label{preservesimple}
	\Huge I don't know if I need this any more, check the lemma from Chapter \ref{ch:known} \normalsize
\end{lemma}
\begin{proof}
	Let $\delta$ be a positive root at $wv.$ Since $w$ induces an isomorphism of simplical complexes, and it sends positive roots at $v$ to positive roots at $wv,$ it must also send negative roots at $v$ to negative roots at $wv.$ So $w^{-1}\delta$ is a root at $v,$ and $w(w^{-1}\delta)=\delta$ is positive, so $w^{-1}\delta$ is also positive. Thus by definition of simple, we have $\gamma \cap \gamma'\subset w^{-1}\delta.$ But we can now apply $w$ to get $w\gamma \cap w\gamma' \subset \delta.$ Since the choice of $\delta$ was arbitrary we must have $w\gamma$ and $w\gamma'$ are simple as desired.
\end{proof}

We can now use the previous lemma to actually construct $\tilde{\phi_v}$ for a certain collection of vertices $v.$



\begin{lemma}
	\label{Dexists}
	Let $x$ be the vertex of $C$ of type $s,$ and assume $U'_x\neq U_x.$ If $w^{-1}x$ is a vertex in $\mathcal{D}=\alpha_1\cap \alpha_n\cap \beta \cap \beta'$ then $\tilde{\phi}_{wx}$ exists.
\end{lemma}
\begin{proof}
	Let $D=\mathrm{Proj}_{w^{-1}x}(C)$ and define $w'$ so that $D=(w')^{-1}C.$ By the definition of projections, $w^{-1}x$ is a vertex of $D$ of type $s,$ but $(w')^{-1}x$ is also a vertex of $D$ of type $s,$ and thus $(w')^{-1}x=w^{-1}x.$ Therefore, we can replace $w$ with $w'$ which we will still call $w$ for notational simplicity. Again, the definition of projections means that $D$ is the closest vertex to $C$ which has a vertex of $w^{-1}x.$ Since $\mathcal{D}$ is convex, and $w^{-1}x$ and $C$ both lie in $\mathcal{D},$ we also know that $D=\mathrm{Proj}_{w^{-1}x}(C)$ lies in $\mathcal{D}$ as well. By a similar argument we know that $\mathrm{Proj}_{x}(D)$ must lie in $\mathcal{D}\subset \alpha_1\cap \alpha_n$ and thus $\mathrm{Proj}_{x}(D)=C.$ Now define $E=wC$ and note that the action of $W$ respects projections and thus we have
	\[
		E=wC=\mathrm{Proj}_{wx}{wD}=\mathrm{Proj}_{wx}{C} \qquad C=wD=\mathrm{Proj}_{w(w^{-1}x)}{wC}=\mathrm{Proj}_{x}{E}
	\]
In particular, if $\gamma$ is any positive root through $wx$ then $E\in \gamma$ by the convexity of $\gamma.$

Our goal is to apply Lemma \ref{existence} at the vertex $wx.$ Now suppose that $\gamma$ is a non-simple, positive root through $wx$ and $y$ is another vertex on $\partial \gamma.$ We must show that $\gamma$ is simple at $y.$ Since $\gamma$ is positive through $wx$ we know that $C,E\in \gamma.$ If we apply $w^{-1}$ then we get the following facts. We know that $w^{-1}\gamma$ is a root such that $\partial (w^{-1}\gamma)$ passes through $w^{-1}wx=x.$ We also know that $w^{-1}C=D,w^{-1}E=C\in w^{-1}\gamma$ so that $w^{-1}\gamma$ is also a positive root.

The first claim is that $w^{-1}\gamma$ is not simple at $x.$ Suppose that $\delta$ is any positive root at $wx.$ Then $E\subset \delta$ and so applying $w^{-1}$ we get that $w^{-1}E=C\subset w^{-1}\delta.$ Thus $w^{-1}$ sends positive roots at $wx$ to positive roots at $x.$ By Lemma \ref{preservesimple} this means that $w^{-1}$ sends simple roots at $wx$ to simple roots at $x.$ Since $\gamma$ is not simple at $wx$ this means that $w^{-1}\gamma$ is not simple at $x.$

So $w^{-1}\gamma$ is a non-simple positive root at $x,$ and since $y$ lies on $\partial \gamma$ we also know that $w^{-1}y$ lies on $w^{-1}(\partial \gamma).$ If we apply Lemma \ref{positiveeverywhereelse} we can see that $w^{-1}\gamma$ must be simple at $w^{-1}y.$ 
\begin{figure}[h]
	\label{mappicture}
\resizebox{6.5 in}{!}{\subimport{diagrams/}{mappicture.tex}}
\caption{The effect of $w$ and $w^{-1}$ on the chambers and roots.}
\end{figure}


Recall that $D\in \mathcal{D}$ by assumption. Now suppose that $\delta$ is any positive root at $w^{-1}y.$ Then by Lemma \ref{containD} we know that $D\in \mathcal{D}\subset \delta.$ If we apply $w$ then we get $C=wD\in w\delta$ and $w\delta$ is a root through $y.$ Thus $w\delta$ is a positive root through $y$ and therefore $w$ sends positive roots through $w^{-1}y$ to positive roots through $y.$ Again we can apply Lemma \ref{preservesimple} to say that $w$ must also send simple roots through $w^{-1}y$ to simple roots through $y.$ But $w^{-1}\gamma$ was a simple root through $w^{-1}y$ and thus $\gamma$ is simple at $y$ as desired.

We know that $wx$ and $x$ are both of type $s.$ We assumed that $[U_x:U'_x]\ge 2$ and thus $[U_{wx}:U'_{wx}]\ge 2$ as well. We have also shown that for any positive root at $wx,$ which is not simple at $wx,$ and any point $y\neq wx$ on $\partial \gamma$ that $\gamma$ is simple at $y.$ Thus we can apply Lemma \ref{existence} to say that $\tilde{\phi}_{wx}$ exists as desired.
\end{proof}

Now we have shown that vertices of $\D$ in some way correspond to $\tilde{\phi_v}.$ If our goal is to find infinitely many such $v$ then there is still some work to be done. For instance, we do not yet know if the region $\D$ contains infinitely many chambers, or even if it does, if all the vertices of $D$ lie on finitely many walls. We will show in the next section that these issues are not a problem in most cases.


\section{When $\D$ is infinite}
Our first task will be two show that the region $\D$ contains infinitely many unique vertices. Intuitively, this will happen if the walls for $\beta$ and $\beta'$ do not meet, and we will first give a sufficient (and necessary) condition for this.

Recall that $W$ is defined by the edge labels $a=m(s,t),b=m(s,u),c=m(t,u)$ with $a\le b \le c.$ For the remainder of the section we will also add the assumption that $b\ge 4.$ This assumption will allow us to show that the region $\D$ contains infinitely many vertices.



\begin{lemma}
	\label{infmany}
	Let $W$ as before with diagram labels $3\le a\le b\le c,$ and $b\ge 4.$ Also let $w_k=(tus)^k$ for all $k\ge 0.$ Then the vertices $(w_k)^{-1}x$ are all distinct from one another, and they all lie in $\D.$
\end{lemma}
\begin{proof}
	Note that $(w_k)^{-1}=(sut)^k$ for all $k.$ First we will show that $(w_k)^{-1}x\in \D$ for all $k.$ Since $x$ is a vertes of $C$ we know that $(w_k)^{-1}x$ is a vertex of $(w_k)^{-1}C$ and thus it will suffice to show $(w_k)^{-1}C$ is contained in $\D$ for all $k.$ Since the roots $\alpha_1,\alpha_n,\beta,\beta'$ can be identified with their corresponding subsets of $W,$ we can use the length function to check containment in these roots.

Now we recall the two $M$ operations on words in a Coxeter group are as follows:
\begin{enumerate}
	\item Delete a subword $ss$ for some $s\in S$
	\item Replace a subword of the form $stst\cdots st(s)$ by a subword of the form $tsts\cdots ts(t)$ where each of these strings has length $m(s,t).$
\end{enumerate}
Also recall that any word in a Coxeter group can be reduced to its minium length by repeated application of these operations, and any two reduced words can be converted each other by application of operations of type 2. Therefore, in order to check that the lenght relations are satisfied, it will be enough to show that we can never perform an $M$ operation of type 1 as this is the only way to reduce length.

It is immediate from the definition that $\ell((w_k)^{-1})=3k$ for all $k.$ We can also see that $\ell(t(w_k)^{-1})=3k+1$ and thus $(w_k)^{-1}\in \alpha_1$ for all $k.$ Similarly, $u(w_k)^{-1}=u(sutsut\cdots),$ and no reduction operations can be done as we assumed $m(s,u)\ge 4.$ Thus $\ell(u(w_k)^{-1})=3k+1$ which means $(w_k)^{-1}\in \alpha_n$ as well.

	Now consider the element $st(w_k)^{-1}.$ If we write this element out in terms of the generators and apply the only possible Coxeter relations we get
	\begin{align*}
		st(w_k)^{-1}&=st(sutsut\cdots)\\
		     &=(sts)(utsuts\cdots)\\
		     &=(tst)(utsuts\cdots)\\
		     &=(ts)(tut)(sutsut\cdots)
	\end{align*}
	and none of these can be reduced as $m(t,u)\ge 4.$ Note that the commutation relation $sts=tst$ may not be possible if $m(s,t)\ge 4,$ but it is the only relation possible in $st(w_k)^{-1}$ and even if it does exists then it does not allow $st(w_k)^{-1}$ to be reduced in length. We previously showed $\ell(t(w_k)^{-1})=3k+1$ and now we see $\ell(st(w_k)^{-1})=3k+2$ and so $(w_k)^{-1}\in \beta.$

	Now we can consider $su(w_k)^{-1}$ in a similar manner. Writing $su(w_k)^{-1}$ out as a word in the generators and applying Coxeter relations gives us
\begin{align*}
	su(w_k)^{-1}&=su(sutsut\cdots)\\
	     &=(susu)(tsutsu\cdots)\\
	     &=(usus)(tsutsu\cdots)\\
	     &=(usu)(sts)(utsuts\cdots)\\
	     &=(usu)(tst)(utsuts\cdots)
\end{align*}
Note once again that not all of these relations may be possible if $m(s,u)=6$ or $m(s,t)\ge 4.$ However, these are the only possible relations, and since $su(w_k)^{-1}$ cannot be reduced under these assumptions, it cannot be reduced at all. Thus $\ell(su(w_k)^{-1})=3k+2$ which means $su(w_k)^{-1}\in \beta'$ as well.

Now it only remains to show that $v_m\neq v_n$ for $m\neq n.$ Suppose $(w_m)^{-1}x=(w_n)^{-1}x$ for $m>n.$ Then we would have $x=w_m(w_n)^{-1}x=w_{m-n}.$ Thus it will suffice to show $w_kx\neq x$ for any $k\ge 1.$ But we know that $\mathrm{stab}_{W}(x)=\langle u,t \rangle$ which does not contain $w_k$ for any $k\ge 1$ and thus $(w_k)^{-1}x\neq x$ so that $(w_m)^{-1}x\neq (w_n)^{-1}x$ as desired.

\end{proof}

We now know that each of the $(w_k)^{-1}x$ is distinct and each of them lies in $\D.$ By Lemma \ref{Dexists} we know that $\tilde{\phi}_{w_kx}$ exists for each $k\ge 0.$ Our idea is still to use each of these vertices to give a root which must be contained in any generating set. However, there is still one possible issue. If almost all of these vertices lie on the same wall, then an inclusion of that root in a generating set could satisfy infinitely many of the $k$ at once, which would not allow us to prove infinite generation. So it remains to show that not only are the vertices $w_nx$ distinct, but also no two lie on the same wall.

\begin{lemma}
	\label{samewall}
	Let $w_k=(tus)^k$ for all $k\ge 0$ and $x$ the vertex of $C$ of type $s.$ If $W$ as in the rest of this section then $w_mx$ and $w_nx$ do not lie on the same wall of $\Sigma$ if $m>n\ge 0.$
\end{lemma}
\begin{proof}
	Suppose $w_mx$ and $w_nx$ do lie on the same wall with $m>n.$ Then we also know that $w_nw^{-1}_mx=w_{n-m}x$ and $x$ will lie on the same wall. Since $m>n$ we can let $k=m-n$ and thus it will suffice to show that $(w_k)^{-1}x$ and $x$ do not lie on the same wall for any $k\ge 1.$
	
We know from Lemma \ref{infmany} that $(w_k)^{-1}x\in \D.$ Thus if $(w_k^{-1})x$ and $x$ lie on the same wall, it must be a wall through $x$ and thus it must be $\partial\alpha_i$ for some $i.$ We know that $(w_k^{-1})x\in \alpha_1\cap \alpha_n$ since $\D\subset \alpha_1\cap \alpha_n$ by defintion. But we can also recall that $\partial\alpha_j\cap \alpha_1\cap \alpha_n=\{x\}$ for $2\le j\le n-1.$ Thus we have $i=1$ or $i=n$ so that $(w_k^{-1})x$ either lies on $\partial\alpha_1$ or $\partial\alpha_n.$  Therefore, we either have $u(w_k)^{-1}x=(w_k)^{-1}x$ or $t(w_k)^{-1}x=(w_k)^{-1}x$ which implies that either $w_kuw_k^{-1}$ or $w_ktw_k^{-1}$ is contained in $\mathrm{stab}_W(x)=\langle u,t \rangle.$ However, by a similar argument as before, we can simply write out these elements and show that they cannot be reduced. The only possible relations we have are
\begin{align*}
	w_ktw_k^{-1}&=(\cdots tustus)t(sutsut\cdots)\\
		    &=(\cdots tustu)(sts)(utsut\cdots)\\
		    &=(\cdots tustu)(tst)(utsut\cdots)\qquad m(t,u)\ge 4
\end{align*}
or
\begin{align*}
	w_kuw_k^{-1}&=(\cdots stustus)u(sutsuts\cdots)\\
		    &=(\cdots stust)(ususu)(tsuts\cdots)\\
		    &=(\cdots stust)(sus)(tsuts\cdots)\\
		    &=(\cdots stu)(sts)u(sts)(uts\cdots)\\
		    &=(\cdots stu)(tst)u(tst)(uts\cdots)\\
\end{align*}
Similarly as before, even these relations are only possible if $m(s,u)=4,$ but even in that case we cannot eliminate every instance of $s$ in $w_kuw_k^{-1}.$ In both cases we can see that there is no further reduction possible and thus neither of these conjugates can possibly lie in $\langle u,t \rangle.$ Thus the $w_nx$ all lie on distinct walls as desired.
\end{proof}

We now have all the ingredients and are ready to prove the main theorem.

\begin{theorem}
	\label{notfinitelygenerated}
	Let $(G,(U_\alpha)_{\alpha\in \Phi},T)$ be an RGD system of type $(W,S).$ Assume $W$ is defined by a Coxeter diagram with edge labels $3\le a\le b\le c$ and also assume that $b\ge 4.$ Let $U_+=\langle U_\alpha|\alpha\in \Phi_+\rangle$ and suppose that $[U_x:U'_x]\ge 2$ where $x$ is the vertex of $C$ of type $s.$ Then $U_+$ is not finitely generated.
\end{theorem}
\begin{proof}
Suppose that $U_+$ is finitely generated. Then there is some finite set of roots $\beta_1,\dots,\beta_m$ such that $U_+=\langle U_{\beta_i}|1\le i\le m\rangle.$ Now no two of the vertices $(tus)^{-k}x$ lie on the same wall and thus we can choose $k$ so that $v=(tus)^{-k}x$ does not lie on $\partial \beta_i$ for any $i.$ By Lemma \ref{infmany} and Lemma \ref{Dexists} we know that $\tilde{\phi}_v$ exists, and by definition it is a surjective map from $U_+\to H$ where $H$ is a cyclic group. However, we can also see by definition that $\tilde{\phi_v}(U_{\beta_i})=1$ for all $i,$ since none of these walls meet $v.$ But this means $\tilde{\phi_v}$ sends all of the generators of $U_+$ to the identity and thus it must be the trivial map which is a contradiction. Thus $U_+$ is not finitely generated as desired.
\end{proof}

A remark worth noting is that the previous proof actually shows something a bit stronger. Since $H$ is abelian, the map $\tilde{\phi}_v$ will factor through the abelianization $(U_+)_{\text{ab}}.$ Then the same arguments as before also show that $(U_+)_{\text{ab}}$ cannot be finitely generated either.


\end{document}
