\documentclass[class=book, crop=false,12 pt]{standalone}
\usepackage[subpreambles=true]{standalone}
\usepackage{import}
\usepackage{/home/mark/Documents/gradschool/research/thesis/preamble}

%\newcommand{\leqnomode}{\tagsleft@true}

\begin{document}
\chapter{Buildings and BN-Pairs}
\label{ch:building}
In Chapter \ref{ch:coxeter} we saw that for a Coxeter system $(W,S),$ we can define a simplical complex $\Sigma$ which will encapsulate the group theoretic structure of $W$ in its geometry. This allows us to understand, $W$ very well, but is somewhat limited as Coxeter groups are very specific. In this chapter we will see how we can generalize some of these notions to other simplical complexes, and then use geometry to study groups which act on them.

\begin{defn}
	\label{defn:building}
	A \emph{building} is a simplical complex $\Delta$ which can be expressed as a union of subcomplexes $\Sigma,$ called Apartments, such that
	\begin{itemize}
		\item[(B0)] Every apartment $\Sigma$ is a Coxeter complex
		\item[(B1)] For any two simplices $A,B\in \Delta,$ there is an apartment containing $A$ and $B.$
		\item[(B2)] For any two apartments $\Sigma,\Sigma',$ there is an isomorphism from $\Sigma$ to $\Sigma'$ which fixes $\Sigma\cap \Sigma'$ pointwise.
	\end{itemize}
\end{defn}

We are using much of the same notation and terminology as \cite{buildings} but we have changed (B2). When introducing the theory of buildings, we can weaken (B2) to another property which is actually equivalent. However, for our purposes it will be easier to simply state the stronger result as an axiom.

As buildings are defined as unions of Coxeter complex, it should come as no surprise that many of the properties of Chapter \ref{ch:coxeter} will still hold, possibly with some slight modification. In fact, a Coxeter complex $\Sigma$ is an example of a building with a single apartment, so nearly every result about buildings in general will also hold for Coxeter complexes.

First of all, we will note that every maximal simplex of $\Delta$ will have the same dimension as any two maximal simplices will lie in some apartment $\Sigma,$ and appartments, which are Coxeter complexes, have the property that every maximal simplex has the same dimension. As with any simplical complex, we will say the dimenion of $\Delta$ is the dimension of a maximal simplex. We will call these maximal simplices Chambers, and we will call co-dimension 1 simplices panels.

As with Coxeter complexes, we will say that two chambers are adjacent if they share a panel. One key difference between buildings and Coxeter complexes is that in a Coxeter complex, exactly two chambers will be adjacent on every panel, where in a building, there can be any number of chambers sharing the same panel, possibly infinitely many. As in the previous chapter, a sequence of chambers $D_0,\dots,D_n$ is called a gallery if $D_i$ and $D_{i+1}$ are adjacent for all $i.$ A building $\Delta$ will be gallery connected as any two chambers will be contained in an apartment, and apartments are gallery connected. We can use galleries to define a metric on the set of chambers of $\Delta,$ where $d(D,E)$ is the length of a minimal gallery from $D$ to $E.$ Even though we know that any two chambers can be connected through an apartment, there is no guarantee a priori that such a gallery would be minimal, or that a minimal gallery can even be contained in a single apartment. However, the following lemma from \cite{buildings} shows that we can focus our attention to apartments.

\begin{lemma}
	\label{lem:dist}
	Suppose $\Delta$ is a building with chambers $D$ and $E.$ If $\Sigma$ is an apartment of $\Delta$ which contains $D$ and $E,$ then any minimal gallery connecting $D$ and $E$ in $\Sigma$ will also be minimal in $\Delta.$
\end{lemma}

A consequence of the previous lemma is that when trying to determine the distance between any two chambers of $\Delta,$ it is enough consider any appartment between the two chambers. When working with Coxeter complexes, we had a stronger notion of adjacency coming from the type function on $\Sigma.$ We were able to say that two chambers $D$ and $E$ were $s$-adjacent if they shared a panel of cotype $s.$ It turns out that we can construct a type function for buildings as well. We will state the result found in \cite{buildings}
\begin{theorem}
	\label{thm:type}
	If $\Delta$ is a building of rank $n,$ and $S$ is a set of size $n,$ then there is a type function $\tau$ which takes values in $S.$
\end{theorem}

We will not include the proof of this theorem, but the idea is as follows. If we fix a chamber $C$ then each apartment containing $C$ will have a type function with values in $S.$ By some permutation of $S,$ we can choose all of these type functions so that they agree on $C.$ Then we glue these type functions together to get a type function on the union of apartments containing $C,$ where compatibility is ensured by (B2). But (B1) ensures that the union of apartments containing $C$ is all of $\Delta$ so we have a well defined type function. This also allows us to define the types and cotypes of simplices, and refer to $s$-adjacent chambers as we did with Coxeter complexes. It also means if we have any gallery $D_0,\dots,D_k,$ they we can define the type of this gallery to be a tuple $(s_1,\dots,s_k)$ such that $D_{i-1}$ and $D_i$ are $s_i$-adjacent for all $i.$

Using $S$-adjacency, and gallery types, we can introduce the notion of residues on $\Delta.$ Assume $\Delta$ has a type function taking values in $S,$ $J\subset S,$ and $D$ is a chamber of $\Delta.$ Then we can define the $J$-residue of $\Delta$ containing $D,$ denoted $\mathcal{R}_J(D),$ to be the chamber sub-complex of $\Delta$ where the chambers are those which can be connected to $D$ through galleries consisting of only $J$-adjacencies. More precicely, a chamber $E$ is in $\mathcal{R}_J(D)$ if and only if there is a gallyer $D=D_0,\dots,D_k=E$ of type $(s_1,\dots,s_k)$ where $s_i\in J$ for all $i.$ There are two ideas which should be discussed before developing more of the general theory started in Coxeter complexes.

If $\Delta$ is a building, then each apartment $\Sigma$ is a Coxeter complex for some Coxeter group $W.$ Since every apartment is isomorphic, and Coxeter complexes are isomorphic if and only if their associated Coxeter groups are isomorphic, we can assign to each building $\Delta$ a well defined (up to isomorphism) Coxeter group $W.$ In this case we say that $W$ is the Weyl group of $\Delta,$ and $\Delta$ is a building of type $W.$ It is also worth mentioning that $W$ can be recovered purely from the combinatorial information of $\Delta.$ If $\tau$ is a type function on $\Delta$ taking values in $S,$ then we can define a Coxeter group $W$ generated by $S$ with $m(s,t)=\mathrm{diam}(\mathcal{R}_J(D))$ where $D$ is any chamber of $\Delta$ and $J=\{s,t\}.$ It can be shown that every apartment of $\Delta$ will be isomorphic to $\Sigma_W$ for this $W,$ and thus $\Delta$ is a building of type $W.$

Finally, we will discuss the multiple ways in which we can treat buildings, a topic that we have mostly glossed over thus far. Our definition of a buildings involved with simpliclial compexes, where we view lower dimensional simplices as being contained in chambers. However, residues give another point of view. For example, suppose we have a panel $P$ of $\Delta$ of cotype $s,$ and a chamber $D$ containing $P.$ Then $\st(P)$ will be all of the simplices of $\Delta$ joinable to $P,$ and the chambers of $\st(P)$ will be exactly those containing $P.$ But if two chambers both contain $P,$ then they are $s$-adjacent, and they also lie in the same $\{s\}$-residue of $\Delta.$ A similar idea holds for simplices of lower rank, and motivates the following theorem.
\begin{theorem}
	\label{thm:sim-cham}
	Suppose $\Delta$ is a building. Then the poset of residues of $\Delta,$ ordered by reverse inclusion, defines a simplical complex which is isomorphic to $\Delta.$
\end{theorem}
The previous theorem allows us to ignore lower dimensional simplices all together, and instead focus on chambers and $J$-residues. In practice, we will not devote completely to one approach or the other, but use whichever is more convinient at the time. The biggest difference between the two aproaches is language. For example, in the simplical viewpoint we think of a panel as a co-dimension 1 simplex, and chambers contain a panel, where in the residue viewpoint, we think of a panel as a $J$ residue where $|J|=1,$ and we say that a panel contains a chamber. This mixing of terminology will not be confusing in context however, as it will be clear what approach is being used at any given time.


\section{Links, Projections, and Roots}
Throughout this section, assume that $\Delta$ is a building with a type function taking values in $S.$ In this section we will examine some of the ideas introduced in the previous chapter.

Suppose that $J\subset S$ and $A$ is a simplex of $\Delta$ of cotype $J.$ As with Coxeter complexes, we define $\st(A)$ to be the set of all simplices which are joinable to $A,$ and $\lk(A)$ is the set of all simplices in $\st(A)$ which are disjoint from $A.$ As aluded to before, the set of all chambers in $\st(A)$ will form a $J$-residue of $\Delta.$ It is also shown in \cite{buildings} that $\lk(A)$ is a chamber complex as well, and it is also a building of type $W_J.$ We also get a nice description of the apartments of $\lk(A).$ Suppose $\mathcal{A}$ is a set of apartments for $\Delta.$ Then $\{\Sigma\cap \lk(A)|A\in \Sigma\}$ is a set of apartments for $\lk(A).$ Furthermore, we know that $\Sigma\cap \lk(A)=\lk_{\Sigma}(A)$ where $\lk_{\Sigma}(A)$ simply denotes the link in the apartment $\Sigma.$

Now suppose that $A$ is a simplex of cotype $J$ and $D$ is any chamber of $\Delta.$ Then there is a unique chamber $E\in \st(A)$ such that $d(D,E)\le d(D,E')$ for all chambers $E'\in \st(A).$ In this case we call $E$ the projection of $D$ onto $A$ and denote it $\proj_A(D).$ If we use the chamber complex point of view then we say $\proj_R(D)$ where $R$ is the $J$-residue coresponding to $A.$ The projection still posesses the gate property so that $d(D,E')=d(D,E)+d(E,E')$ for all chambers $E'\in \st(A).$ Since distances and minimal galleries can be computed in suitable apartments, it is of no surprise that projections can also be computed in apartments. To be more precise, if $\Sigma$ is an apartment of $\Delta$ containing $A$ and $D,$ then $\proj_A^\Delta(D)=\proj_A^\Sigma(D).$ 

A subset of $\mathcal{M}$ of $\Delta$ is called convex if for every simplex $A$ and chamber $D$ of $M,$ we have that $\proj_A(D)\in M.$ The condition that $A$ is contained in $\mathcal{M}$ is replaced by the assumption that the residue $R$ meets $\mathcal{M}$ in the chamber complex viewpoint. Similar to the case for Coxeter complexes, the condition that $\mathcal{M}$ is convex is eqivalent to ensuring that for any chambers $D,E$ of $\mathcal{M},$ any minimal gallery connecting $D$ and $E$ will be completely contained in $\mathcal{M}.$ Through some of the comments made earlier, we have more or less shown that residues and apartments of a building are both convex subcomplexes. 

Recall that in a Coxeter complex, for every pair of adjacent chambers $D,D',$ we define the root $\alpha_{D,D'}$ to be the chambers which are closer to $D$ than to $D'.$ In a a building $\Delta,$ a subset $\alpha$ is called a root if it is a root of some apartment $\Sigma$ of $\Delta.$

We have mentioned it before but it is worth reiterating, most of the properties and definitions for buildings can be defined in terms of apartments. For this reason, you will see throughout the remainder our our work that we will rarely reference the building $\Delta$ at all, but will instead choose appropriate apartments and work there. This also makes our lives easier as panels contain only 2 chambers, the the interaction between $W$ and $\Sigma$ is much more straightforward than that between $W$ and $\Delta.$ 

\section{Spherical Buildings}
We say a building $\Delta$ is spherical if the Weyl group $W$ is spherical, or equivalently if each apartment $\Sigma$ is a spherical Coxeter complex. As stated before, this means that $W$ has a unique element of maximal length and the diameter of any apartment will also be this length. We say that two chambers $C$ and $D$ are opposite if $d(C,D)$ is maximal, and we write $C \mathrm{ op }D.$ If $C$ and $C'$ are opposite chambers of $\Delta,$ then there is a unique apartment containing $C$ and $C',$ and this apartment is the minimal convex subset of $\Delta$ containing $C$ and $C'.$ 

Results about spherical buildings will be especially useful when consider 2-spherical buildings, where $m(s,t)<\infty$ for all $s,t\in S.$ In this case, every codimension 2 link will be a spherical building and we can use facts about opposition to study local properties of $\Delta.$

\end{document}
