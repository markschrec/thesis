\documentclass[class=book, crop=false,12 pt]{standalone}
\usepackage[subpreambles=true]{standalone}
\usepackage{import}
\RequirePackage{/home/mark/Documents/gradschool/research/thesis/preamble}

%\newcommand{\leqnomode}{\tagsleft@true}

\begin{document}
\chapter{Buildings}
\label{ch:buildings}
In Chapter \ref{ch:Coxeter} we saw that for a Coxeter system $(W,S),$ we can define a simplicial complex $\Sigma$ which will encapsulate the group theoretic structure of $W$ in its geometry. The theory of finite reflection groups has also long been part of the study of classical groups, as the associated Weyl group will always be a spherical Coxeter group. The theory of buildings, as developed largely by Tits and Bruhat in the 50's and 60's, looks to extend the information given by the Weyl group $W$ in a natural way. We do not however have the time to discuss the full history of the building axioms, so we begin with a definition which we will study for the rest of the chapter.

\begin{defn}
	\label{defn:building}
	A \emph{building} is a simplicial complex $\Delta$ which can be expressed as a union of sub-complexes $\Sigma,$ called Apartments, such that
	\begin{itemize}
		\item[(B0)] Every apartment $\Sigma$ is a Coxeter complex
		\item[(B1)] For any two simplices $A,B\in \Delta,$ there is an apartment containing $A$ and $B.$
		\item[(B2)] For any two apartments $\Sigma,\Sigma',$ there is an isomorphism from $\Sigma$ to $\Sigma'$ which fixes $\Sigma\cap \Sigma'$ pointwise.
	\end{itemize}
\end{defn}

We are using much of the same notation and terminology as \cite{buildings} but we have changed (B2). It is possible to replace (B2) by an apparently weaker axiom which is actually equivalent. However, we choose here to just list the stronger result as an axiom as it will be easier to work with. When looking at the axioms, (B1) and (B2) seem to be fairly natural connectedness and coherence relations, but (B0) seems to be somewhat arbitrary. Besides the appearance of Coxeter groups in classical groups, it is unclear why we should require sub-complexes to be Coxeter complexes. At the end of the chapter we will see that Coxeter complexes are in fact the natural (and only) choice for these apartments.

As buildings are defined as unions of Coxeter complex, it should come as no surprise that many of the properties of Chapter \ref{ch:Coxeter} will still hold, possibly with some slight modification. In fact, a Coxeter complex $\Sigma$ is an example of a building with a single apartment, so nearly every result about buildings in general will also hold for Coxeter complexes.

First of all, we will note that every maximal simplex of $\Delta$ will have the same dimension as any two maximal simplices will lie in some apartment $\Sigma,$ and apartments, which are Coxeter complexes, have the property that every maximal simplex has the same dimension. As with any simplicial complex, we will say the dimension of $\Delta$ is the dimension of a maximal simplex. We will call these maximal simplices Chambers, and we will call co-dimension 1 simplices panels, using the same terminology as in the Coxeter complex.


As with Coxeter complexes, we will say that two chambers are adjacent if they share a panel as a common face. One key difference between buildings and Coxeter complexes is that in a Coxeter complex, exactly two chambers will be adjacent on every panel, where in a building, there can be any number of chambers sharing the same panel, possibly infinitely many. 
We say a building is \emph{thick} if each panel is a face of at least 3 chambers. We say a building is \emph{thin} if each panel is the face of exactly 2 chambers. As alluded to before, thin buildings are exactly Coxeter complexes. The definition of a thick building may again seem arbitrary, but we will see throughout the rest of the paper that many strong results related to buildings rely on thickness. It also turns out the most buildings arising naturally from the study of interesting groups, will be thick, and the number of chambers containing a given panel will have a nice interpretation as well.

As in the previous chapter, a sequence of chambers $D_0,\dots,D_n$ is called a gallery if $D_i$ and $D_{i+1}$ are adjacent for all $i.$ A building $\Delta$ will be gallery connected as any two chambers will be contained in an apartment, and apartments are gallery connected. We can use galleries to define a metric on the set of chambers of $\Delta,$ where $d(D,E)$ is the length of a minimal gallery from $D$ to $E.$ Even though we know that any two chambers can be connected through an apartment, there is no guarantee a priori that such a gallery would be minimal, or that a minimal gallery can even be contained in a single apartment. However, Corollary 4.34 from \cite{buildings} gives us the following lemma.

\begin{lemma}
	\label{lem:dist}
	Suppose $\Delta$ is a building with chambers $D$ and $E.$ If $\Sigma$ is an apartment of $\Delta$ which contains $D$ and $E,$ then any minimal gallery connecting $D$ and $E$ in $\Sigma$ will also be minimal in $\Delta.$
\end{lemma}

A consequence of the previous lemma is that when trying to determine the distance between any two chambers of $\Delta,$ it is enough consider any apartment between the two chambers. When working with Coxeter complexes, we had a stronger notion of adjacency coming from the type function on $\Sigma.$ We were able to say that two chambers $D$ and $E$ were $s$-adjacent if they shared a panel of cotype $s.$ It turns out that we can construct a type function for buildings as well. By Proposition 4.6 in \cite{buildings} we get the following theorem.
\begin{theorem}
	\label{thm:type}
	If $\Delta$ is a building of rank $n,$ and $S$ is a set of size $n,$ then there is a type function $\tau$ which takes values in $S.$ Furthermore, it is possible to choose isomorphisms satisfying (B2) which are type preserving.
\end{theorem}

We will not include the proof of this theorem, but the idea is as follows. If we fix a chamber $C$ then each apartment containing $C$ will have a type function with values in $S.$ By some permutation of $S,$ we can choose all of these type functions so that they agree on $C.$ Then we glue these type functions together to get a type function on the union of apartments containing $C,$ where compatibility is ensured by (B2). But (B1) ensures that the union of apartments containing $C$ is all of $\Delta$ so we have a well defined type function. This also allows us to define the types and cotypes of simplices, and refer to $s$-adjacent chambers as we did with Coxeter complexes. It also means if we have any gallery $D_0,\dots,D_k,$ they we can define the type of this gallery to be a tuple $(s_1,\dots,s_k)$ such that $D_{i-1}$ and $D_i$ are $s_i$-adjacent for all $i.$

Using $S$-adjacency, and gallery types, we can introduce the notion of residues on $\Delta.$ Assume $\Delta$ has a type function taking values in $S,$ $J\subset S,$ and $D$ is a chamber of $\Delta.$ Then we can define the $J$-residue of $\Delta$ containing $D,$ denoted $\mathcal{R}_J(D),$ to be the chamber sub-complex of $\Delta$ where the chambers are those which can be connected to $D$ through galleries consisting of only $J$-adjacencies. More precisely, a chamber $E$ is in $\mathcal{R}_J(D)$ if and only if there is a gallery $D=D_0,\dots,D_k=E$ of type $(s_1,\dots,s_k)$ where $s_i\in J$ for all $i.$ There are two ideas which should be discussed before developing more of the general theory started in Coxeter complexes.

If $\Delta$ is a building, then each apartment $\Sigma$ is a Coxeter complex for some Coxeter group $W.$ Since every apartment is isomorphic, and Coxeter complexes are isomorphic if and only if their associated Coxeter groups are isomorphic, we can assign to each building $\Delta$ a well defined (up to isomorphism) Coxeter group $W.$ In this case we say that $W$ is the Weyl group of $\Delta,$ and $\Delta$ is a building of type $W.$ It is also worth mentioning that $W$ can be recovered purely from the combinatorial information of $\Delta.$ If $\tau$ is a type function on $\Delta$ taking values in $S,$ then we can define a Coxeter group $W$ generated by $S$ with $m(s,t)=\mathrm{diam}(\mathcal{R}_J(D))$ where $D$ is any chamber of $\Delta$ and $J=\{s,t\}.$ A consequence of Corollary 4.36 of \cite{buildings} is that each apartment $\Sigma$ will be isomorphic to the Coxeter complex of this $W.$ It also allows for the definition of a Coxeter system $(W,S)$ where $S$ is the range of the type function on $\Delta.$

Finally, we will discuss the multiple ways in which we can treat buildings, a topic that we have mostly glossed over thus far. Our definition of a buildings involved with simplicial complexes, where we view lower dimensional simplices as being contained in chambers. In the previous chapter we often just considered the chambers of $\Sigma,$ and we even alluded to the fact that this was ``enough''. We are now ready to formalize what we mean. If $D$ and $E$ are chambers in the same $J$-residue, then $D$ and $E$ will share a face of cotype $J$ and vice versa. This gives a natural correspondence between residues and stars of simplices.

For example, let's consider Panels. Simplicially, a panel is a simplex of co-dimension 1 and cotype $\{s\}$ for some $s\in S.$ If $D$ is a chamber containing a panel $P$ of cotype $s,$ then $\mathcal{R}_s(D)$ will be all of the chambers which are $s$-adjacent to $D.$ This also corresponds exactly to the chambers in $\st(P),$ and thus we can associate to each rank 1 residue, a panel $P.$ The connection works for simplices of all ranks, and Lemma 5.88 of \cite{buildings} gives the following result.

\begin{theorem}
	\label{thm:sim-cham}
	Suppose $\Delta$ is a building. Let $\mathcal{C}$ be the set of residues of $\Delta,$ ordered by reverse inclusion. Then $\mathcal{C}$ can be viewed as a simplicial complex, and it is canonically isomorphic to $\Delta.$ The isomorphism is given by $\mathcal{R}_J(D)\mapsto \cap_{E\in \mathcal{R}_J(D)}E.$
\end{theorem}

The previous theorem allows us to ignore lower dimensional simplices all together, and instead focus on chambers and $J$-residues. In practice, we will not devote completely to one approach or the other, but use whichever is more convenient at the time. The biggest difference between the two approaches is language. For example, in the simplicial viewpoint we think of a panel as a co-dimension 1 simplex, and chambers contain a panel, where in the residue viewpoint, we think of a panel as a $J$ residue where $|J|=1,$ and we say that a panel contains a chamber. This mixing of terminology will not be confusing in context however, as it will be clear what approach is being used at any given time.


\section{Links, Projections, and Roots}
Throughout this section, assume that $\Delta$ is a building with a type function taking values in $S.$ In this section we will examine some of the ideas introduced in the previous chapter.

Suppose that $J\subset S$ and $A$ is a simplex of $\Delta$ of cotype $J.$ As with Coxeter complexes, we define $\st(A)$ to be the set of all simplices which are joinable to $A,$ and $\lk(A)$ is the set of all simplices in $\st(A)$ which are disjoint from $A.$ As alluded to before, the set of all chambers in $\st(A)$ will form a $J$-residue of $\Delta.$ Proposition 4.9 in \cite{buildings} gives us the following analog of Proposition \ref{prop:link}, If $A$ is a simplex of cotype $J,$ then $\lk(A)$ is a building of type $(W_J,J),$ and the type function on $\lk(A)$ is the restriction of the type function on $\Delta.$ The same proposition also says if $\mathcal{A}$ is a set of apartments for $\Delta,$ then $\{\lk(A)\cap \Sigma|A\in \Sigma\}_{\Sigma\in \mathcal{A}}$ is a set of apartments for $\lk(A).$ 

Now suppose that $A$ is a simplex of cotype $J$ and $D$ is any chamber of $\Delta.$ By Proposition 4.95 in \cite{buildings}, there is a unique chamber $E\in \st(A)$ such that $d(D,E)\le d(D,E')$ for all chambers $E'\in \st(A).$ In this case we call $E$ the projection of $D$ onto $A$ and denote it $\proj_A(D).$ If we use the chamber complex point of view then we say $\proj_R(D)$ where $R$ is the $J$-residue corresponding to $A.$ The projection still possesses the gate property so that $d(D,E')=d(D,E)+d(E,E')$ for all chambers $E'\in \st(A).$ Proposition 4.95 also says that projections can be computed in any apartment containing $A$ and $D.$ More precisely, $\proj_A^\Sigma (D)=\proj_A^\Delta(D)$ where the projections are taken in an apartment $\Sigma$ and $\Delta$ respectively.

A subset of $\mathcal{M}$ of $\Delta$ is called convex if for every simplex $A$ and chamber $D$ of $M,$ we have that $\proj_A(D)\in M.$ The condition that $A$ is contained in $\mathcal{M}$ is replaced by the assumption that the residue $R$ meets $\mathcal{M}$ in the chamber complex viewpoint. Similar to the case for Coxeter complexes, the condition that $\mathcal{M}$ is convex is equivalent to ensuring that for any chambers $D,E$ of $\mathcal{M},$ any minimal gallery connecting $D$ and $E$ will be completely contained in $\mathcal{M},$ by Proposition 4.115 in \cite{buildings}. As projections  can be computed in any apartment, some of our previous comments show that apartments and residues are both convex chamber sub-complexes of $\Delta.$

Recall that in a Coxeter complex, for every pair of adjacent chambers $D,D',$ we define the root $\alpha_{D,D'}$ to be the chambers which are closer to $D$ than to $D'.$ In a building $\Delta,$ a subset $\alpha$ is called a root if it is a root of some apartment, and thus every apartment $\Sigma$ containing $\alpha.$

We have mentioned it before but it is worth reiterating, most of the properties and definitions for buildings can be defined in terms of apartments. For this reason, you will see throughout the remainder our our work that we will rarely reference the building $\Delta$ at all, but will instead choose appropriate apartments and work there. This will be especially be the case once we introduce group actions in the next chapter, which allow us to view every apartment as some translate of a fixed choice of apartment $\Sigma.$

\section{Spherical Buildings}
We say a building $\Delta$ is spherical if the Weyl group $W$ is spherical, or equivalently if each apartment $\Sigma$ is a spherical Coxeter complex. As stated before, this means that $W$ has a unique element of maximal length and the diameter of any apartment will also be this length. We say that two chambers $C$ and $D$ are opposite if $d(C,D)$ is maximal, and we write $C \mathrm{ op }D.$ By Theorem 4.70 of \cite{buildings}, we know that for each pair of opposite chambers, there is a unique apartment containing the pair, and that apartment is the minimal convex chamber complex containing the pair. 

Results about spherical buildings will be especially useful when consider 2-spherical buildings, where $m(s,t)<\infty$ for all $s,t\in S.$ In this case, every co-dimension 2 link will be a spherical building and we can use facts about opposition to study local properties of $\Delta.$ In particular every rank 2 residue will be a generalized $m$-gon, which is defined as a rank 2 building.

\section{Thick Buildings}
Earlier in the chapter we said that a building was thin if every panel is a face of exactly two chambers, and it is thick if each panel is the face of at least 3 chambers. The definition of thin seems natural as Coxeter complexes are thin buildings, and in fact the only thin buildings. In this section we will seek to justify the thick condition. In the next chapter we will see how we can associate to any vector space $V,$ a canonical building $\Delta(V).$ This discussion is better saved for the following, but we will introduce one fact here. In $\Delta(V),$ each panel is the face of exactly $|k|+1$ chambers, where $k$ is the base field for $V.$ Since $|k|+1\ge 3$ for any field $k,$ this building will be a thick building. The buildings associated to other classical groups will also be thick, as will nearly all buildings which arise naturally.

Another surprising result is the relation between thickness and the building axioms. Theorem 4.131 of \cite{buildings} says that if $\Delta$ is any thick chamber complex, with a collection of thin chamber sub-complexes $\{\Sigma\}$ which satisfy axioms (B1) and (B2), that $\Delta$ will be a building, and each $\Sigma$ will be a Coxeter complex. Not only does this show the strength of the thick condition, but it also shows that Coxeter complexes are in some sense the ``correct'' type of chamber complexes to consider.

\end{document}
